\documentclass[
english,
smallborders
]{i6prcsht}
\usepackage{i6common}
\usepackage{i6lecture}

\usepackage{todonotes}
\usepackage[utf8]{inputenc}
\usepackage{textcomp}
\usepackage{pdfpages}
\usepackage{csquotes}
\usepackage{awesomebox}

\hyphenation{Stud-On}

\begin{document}

\title{Exercise Sheet 2: \\ Data Analysis and Preprocessing}
\maketitle
\vspace*{-2cm}

\section*{About this Exercise Sheet}

This exercise sheet focuses on the content of the lectures \textit{3. Getting to Know Your Data} (Exercise 1) and \textit{4. Data Preprocessing} (Exercises 2 and 3). The goal of this exercise sheet is to familiarize students with the content through a practical data science example.

The exercise sheet is designed to span three weeks and will thus be covered in three exercise sessions. The plan is to tackle one exercise per session. However, if there is not enough time for an exercise in any session, it can be continued in the following session. Similarly, if there is extra time, work can commence on the next exercise.

To accommodate this flexible model, the sample solution will be published only after the three weeks have elapsed.

\section*{Preparation}

Before participating in the exercise, you must prepare the following:

\begin{enumerate}
	\item \textbf{Install Python and pip on your computer}
	      
	      \begin{itemize}
		      \item Detailed instructions can be found in \texttt{1-Introduction-Python-Pandas.pdf}.
	      \end{itemize}
	      
	\item \textbf{Download provided additional files}
	      
	      \begin{itemize}
		      \item Download \texttt{Additional-Files-Student.zip} from StudOn
		      \item Extract it to a folder of your choice.
	      \end{itemize}
	      
	\item \textbf{Install required Python packages}
	      
	      \begin{itemize}
		      \item Open a terminal and navigate to the folder where you extracted the files.
		      \item Run the command \texttt{pip install -r requirements.txt} within the extracted additional files folder to install the required Python packages.
	      \end{itemize}
	      
	      
\end{enumerate}

\section*{Exercise 1: Getting to Know Your Data}

This exercise comprises practical data science tasks and thus utilizes a Jupyter Notebook:

\begin{enumerate}
	\item Open \texttt{Getting-to-Know-Your-Data.ipynb}.
	\item Take a look at the tasks (blue boxes) in the notebook and try to solve them.
\end{enumerate}

If you are unfamiliar with how to open a Jupyter Notebook, please refer to Exercise 1 of \texttt{1-Introduction-Python-Pandas.pdf}.

If you have worked through \texttt{1-Introduction-Python-Pandas.pdf} some of the initial tasks may seem familiar to you. However, this is a good repetition to consolidate the methods that you will need more often during the semester.

\begin{solution}
	The solution to the exercise can be found in \texttt{Additional-Files-Solution.zip}.
\end{solution}

\section*{Exercise 2: Data Cleaning and Integration}

This exercise comprises practical data science tasks and thus utilizes a Jupyter Notebook:

\begin{enumerate}
	\item Open \texttt{Data-Cleaning-and-Integration.ipynb}.
	\item Take a look at the tasks (blue boxes) in the notebook and try to solve them.
\end{enumerate}

If you are unfamiliar with how to open a Jupyter Notebook, please refer to Exercise 1 of \texttt{1-Introduction-Python-Pandas.pdf}.

\begin{solution}
	The solution to the exercise can be found in \texttt{Additional-Files-Solution.zip}.
\end{solution}

\section*{Exercise 3: Data Reduction, Transformation, and Discretization}

This exercise comprises practical data science tasks and thus utilizes a Jupyter Notebook:

\begin{enumerate}
	\item Open \texttt{Data-Reduction-Transformation-and-Discretization.ipynb}.
	\item Take a look at the tasks (blue boxes) in the notebook and try to solve them.
\end{enumerate}

If you are unfamiliar with how to open a Jupyter Notebook, please refer to Exercise 1 of \texttt{1-Introduction-Python-Pandas.pdf}.

\begin{solution}
	The solution to the exercise can be found in \texttt{Additional-Files-Solution.zip}.
\end{solution}

\end{document}

\end{document}
