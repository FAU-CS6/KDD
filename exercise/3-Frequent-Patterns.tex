\documentclass[
english,
smallborders
]{i6prcsht}
\usepackage{i6common}
\usepackage{i6lecture}

\usepackage{todonotes}
\usepackage[utf8]{inputenc}
\usepackage{textcomp}
\usepackage{pdfpages}
\usepackage{csquotes}
\usepackage{awesomebox}
\usepackage{makecell}
\usepackage{graphicx}
\usepackage{multicol}

\hyphenation{Stud-On}

\begin{document}

\title{Exercise Sheet 3: \\ Frequent Patterns}
\maketitle
\vspace*{-2cm}

\section*{About this Exercise Sheet}

This exercise sheet focuses on the content of lecture \textit{6. Mining Frequent Patterns, Associations and Correlations}.

It includes both a practical data science exercise (Exercise 1) and theoretical exercises on Apriori (Exercise 2) and FP-growth (Exercise 3).

The exercise sheet is designed for a two-week period, during which the tasks can be completed flexibly (Exercise 1 is planned for the first exercise session, and Exercises 2 and 3 for the second session).

The sample solution will be published after the two weeks have elapsed.

\section*{Preparation}

Before participating in the exercise, you must prepare the following:

\begin{enumerate}
	\item \textbf{Install Python and pip on your computer}
	      
	      \begin{itemize}
		      \item Detailed instructions can be found in \texttt{1-Introduction-Python-Pandas.pdf}.
	      \end{itemize}
	      
	\item \textbf{Download provided additional files}
	      
	      \begin{itemize}
		      \item Download \texttt{Additional-Files-Student.zip} from StudOn
		      \item Extract it to a folder of your choice.
	      \end{itemize}
	      
	\item \textbf{Install required Python packages}
	      
	      \begin{itemize}
		      \item Open a terminal and navigate to the folder where you extracted the files.
		      \item Run the command \texttt{pip install -r requirements.txt} within the extracted additional files folder to install the required Python packages.
	      \end{itemize}
	      
	      
\end{enumerate}

\section*{Exercise 1: Mining Frequent Patterns}

This exercise comprises practical data science tasks and thus utilizes a Jupyter Notebook:

\begin{enumerate}
	\item Open \texttt{Mining-Frequent-Patterns.ipynb}.
	\item Take a look at the tasks (blue boxes) in the notebook and try to solve them.
\end{enumerate}

If you are unfamiliar with how to open a Jupyter Notebook, please refer to Exercise 1 of \texttt{1-Introduction-Python-Pandas.pdf}.

\begin{solution}
	The solution to the exercise can be found in \texttt{Additional-Files-Solution.zip}.
\end{solution}

\section*{Exercise 2: Apriori}

Given is a \textbf{transactional dataset}:

\begin{center}
	\begin{tabular}{|c|l|}
		\hline
		\textbf{ID} & \textbf{Transaction}       \\
		\hline
		1           & Apple, Banana, Cherry      \\
		\hline
		2           & Banana, Cherry             \\
		\hline
		3           & Cherry, Apple              \\
		\hline
		4           & Dragonfruit, Apple, Banana \\
		\hline
		5           & Apple, Dragonfruit         \\
		\hline
	\end{tabular}
\end{center}

Use \textbf{Apriori} to find all frequent itemsets for a \textbf{minimum support count} of \textbf{2}.

Write down \textbf{all} intermediate steps.

\begin{solution}
	\begin{enumerate}
		\item \textbf{Count the occurrences of each 1-itemset}:
		      
		      Each item that occurs in the dataset is a 1-itemset:
		      
		      \begin{itemize}
			      \item Apple: 4
			      \item Banana: 3
			      \item Cherry: 3
			      \item Dragonfruit: 2
		      \end{itemize}
		      
		\item \textbf{Prune non-frequent 1-itemsets}:
		      
		      All 1-itemsets have a support count of at least 2. Therefore, all 1-itemsets are frequent.
		      
		\item \textbf{Generate length-2 candidate itemsets}:
		      
		      The candidate itemsets are generated by combining all the frequent 1-itemsets:
		      
		      \begin{itemize}
			      \item Apple, Banana
			      \item Apple, Cherry
			      \item Apple, Dragonfruit
			      \item Banana, Cherry
			      \item Banana, Dragonfruit
			      \item Cherry, Dragonfruit
		      \end{itemize}
		      
		\item \textbf{Count the occurrences of each length-2 candidate itemset}:
		      
		      \begin{itemize}
			      \item Apple, Banana: 2
			      \item Apple, Cherry: 2
			      \item Apple, Dragonfruit: 2
			      \item Banana, Cherry: 2
			      \item Banana, Dragonfruit: 1
			      \item Cherry, Dragonfruit: 0
		      \end{itemize}
		      
		\item \textbf{Prune non-frequent length-2 candidate itemsets}:
		      
		      The length-2 candidate itemsets that have a support count of at least 2 are:
		      
		      \begin{itemize}
			      \item Apple, Banana
			      \item Apple, Cherry
			      \item Apple, Dragonfruit
			      \item Banana, Cherry
		      \end{itemize}
		      
		\item \textbf{Generate length-3 candidate itemsets}:
		      
		      The candidate itemsets are generated by combining all the frequent length-2 itemsets:
		      
		      \begin{itemize}
			      \item Apple, Banana, Cherry
		      \end{itemize}
		      
		      This length-3 itemset contains the frequent length-2 itemsets \glqq Apple, Banana\grqq\ and \glqq Banana, Cherry\grqq , and \glqq Apple, Cherry\grqq\ and is the only valid length-3 candidate.
		      
		      \textit{\textbf{Common Mistake:} A common mistake is that \glqq Apple, Banana, Dragonfruit\grqq , \glqq Apple, Cherry, Dragonfruit\grqq , and \glqq Banana, Cherry, Dragonfruit\grqq\ are generated as length-3 candidates. These 3-itemsets each contain at least one non-frequent 2-itemset (e.g \glqq Apple, Banana, Dragonfruit\grqq\ contains \glqq Banana, Dragonfruit\grqq ) and are therefore not valid length-3 candidates.}
		      
		\item \textbf{Count the occurrences of the length-3 candidate itemset}:
		      
		      \begin{itemize}
			      \item Apple, Banana, Cherry: 1
		      \end{itemize}
		      
		\item \textbf{Prune non-frequent length-3 candidate itemsets}:
		      
		      \glqq Apple, Banana, Cherry\grqq\ has a support count of 1, which is below the minimum support count of 2. Therefore, there are no frequent length-3 itemsets.
		      
		\item \textbf{Generate length-4 candidate itemsets}:
		      
		      There are no frequent length-3 itemsets, so there are no valid length-4 candidates.
		      
		\item \textbf{Termination}:
		      
		      The algorithm terminates because there are no length-4 candidates.
		      
	\end{enumerate}
	
	\textbf{Result:}
	
	\vspace*{-0.2cm}
	
	The frequent itemsets for a minimum support count of 2 are:
	
	\begin{multicols}{4}
		\begin{enumerate}
			\item Apple
			\item Banana
			\item Cherry
			\item Dragonf.
			\item Apple, Banana
			\item Apple, Cherry
			\item Apple, Dragonf.
			\item Banana, Cherry
		\end{enumerate}
	\end{multicols}
\end{solution}

\section*{Exercise 3: FP-growth}

Given is a \textbf{transactional dataset}:

\begin{center}
	\begin{tabular}{|c|l|}
		\hline
		\textbf{ID} & \textbf{Transaction} \\
		\hline
		1           & Apple, Banana        \\
		\hline
		2           & Banana, Cherry       \\
		\hline
		3           & Cherry, Apple        \\
		\hline
		4           & Apple, Banana        \\
		\hline
		5           & Apple, Dragonfruit   \\
		\hline
	\end{tabular}
\end{center}

Use \textbf{FP-growth} to find all frequent itemsets for a \textbf{minimum support count} of \textbf{2}.

Write down \textbf{all} intermediate steps. This \textbf{includes} the header table for each FP-tree.

\begin{solution}
	\newcommand{\fptreerootnode}{
		\begin{tikzpicture}
			\node[draw, fill=white, minimum height = 0.65cm, minimum width=2.25cm] at (0,-1) (Label) {$\{\}$};
		\end{tikzpicture}
	}
	
	\newcommand{\fptreenodewithoccurences}[2]{
		\begin{tikzpicture}
			\node[draw, fill=white, minimum height = 0.65cm, minimum width=1.75cm] at (0,-1) (Label) {$#1$};
			\node[draw, fill=white, minimum height = 0.65cm, minimum width=0.5cm, right = 0cm of Label] {$#2$};
		\end{tikzpicture}
	}
	
	\begin{enumerate}
		\item \textbf{Count the occurrences of each 1-itemset}:
		      
		      Each item that occurs in the dataset is a 1-itemset:
		      
		      \begin{itemize}
			      \item Apple: 4
			      \item Banana: 3
			      \item Cherry: 2
			      \item Dragonfruit: 1
		      \end{itemize}
		      
		\item \textbf{Prune non-frequent 1-itemsets}:
		      
		      The 1-itemsets that have a support count of at least 2 are:
		      
		      \begin{itemize}
			      \item Apple: 4
			      \item Banana: 3
			      \item Cherry: 2
		      \end{itemize}
		      
		\item \textbf{Create the f-list for our dataset}:
		      
		      The f-list is created by sorting the 1-itemsets in descending order of their support count:
		      
		      \begin{itemize}
			      \item Apple $\rightarrow$ Banana $\rightarrow$ Cherry
		      \end{itemize}
		      
		\item \textbf{Order the items in the transactions according to the f-list}:
		      
		      Additionally, non frequent items are removed from the transactions:
		      
		      \begin{center}
			      \begin{tabular}{|c|l|}
				      \hline
				      \textbf{ID} & \textbf{Transaction} \\
				      \hline
				      1           & Apple, Banana        \\
				      \hline
				      2           & Banana, Cherry       \\
				      \hline
				      3           & Apple, Cherry        \\
				      \hline
				      4           & Apple, Banana        \\
				      \hline
				      5           & Apple                \\
				      \hline
			      \end{tabular}
		      \end{center}
		      
		      \newpage
		      
		\item \textbf{Create the initial FP-tree}:
		      
		      The initial FP-tree is created by inserting the items of each transaction into the tree:
		      
		      \begin{enumerate}
			      \item \textbf{Insert the first transaction (Apple, Banana)}:
			            
			            \begin{center}
				            \scalebox{0.85}{
					            \begin{tikzpicture}[scale=3]
						            \useasboundingbox (-2,0) rectangle (5,-1.25);
						            
						            \node[minimum height = 0.65cm, minimum width=2cm] at (0,0) (0) {\fptreerootnode};
						            
						            \node[fill=white, minimum height = 0.65cm, minimum width=2.25cm] at (0,-0.5) (a1) {\fptreenodewithoccurences{$Apple$}{$1$}};
						            \node[fill=white, minimum height = 0.65cm, minimum width=2.25cm] at (0,-1) (a1b1) {\fptreenodewithoccurences{$Banana$}{$1$}};
						            
						            \draw (0) -- (a1) -- (a1b1);
						            
						            % Header table
						            \node[anchor=west, text width=5cm] at (2,-0.5) (Header Table) {
							            \begin{center}
								            \textbf{Header table:}
								            
								            \vspace{0.45cm}
								            
								            \begin{tabular}{|c|c|c|}
									            \hline
									            \textbf{Item} & \textbf{Freq.} & \textbf{Nodes} \\
									            \hline
									            Apple         & 1              & 1              \\
									            \hline
									            Banana        & 1              & 1              \\
									            \hline
									            Cherry        & 0              & 0              \\
									            \hline
								            \end{tabular}
							            \end{center}
						            };
						            
					            \end{tikzpicture}
				            }
			            \end{center}
			            
			      \item \textbf{Insert the second transaction (Banana, Cherry)}:
			            
			            \begin{center}
				            \scalebox{0.85}{
					            \begin{tikzpicture}[scale=3]
						            \useasboundingbox (-2,0) rectangle (5,-1.25);
						            
						            \node[minimum height = 0.65cm, minimum width=2cm] at (0,0) (0) {\fptreerootnode};
						            
						            \node[fill=white, minimum height = 0.65cm, minimum width=2.25cm] at (-0.5,-0.5) (a1) {\fptreenodewithoccurences{$Apple$}{$1$}};
						            \node[fill=white, minimum height = 0.65cm, minimum width=2.25cm] at (-0.5,-1) (a1b1) {\fptreenodewithoccurences{$Banana$}{$1$}};
						            \node[fill=white, minimum height = 0.65cm, minimum width=2.25cm] at (0.5,-0.5) (b1) {\fptreenodewithoccurences{$Banana$}{$1$}};
						            \node[fill=white, minimum height = 0.65cm, minimum width=2.25cm] at (0.5,-1) (b1c1) {\fptreenodewithoccurences{$Cherry$}{$1$}};
						            
						            \draw (0) -- (a1) -- (a1b1);
						            \draw (0) -- (b1) -- (b1c1);
						            
						            % Header table
						            \node[anchor=west, text width=5cm] at (2,-0.5) (Header Table) {
							            \begin{center}
								            \textbf{Header table:}
								            
								            \vspace{0.45cm}
								            
								            \begin{tabular}{|c|c|c|}
									            \hline
									            \textbf{Item} & \textbf{Freq.} & \textbf{Nodes} \\
									            \hline
									            Apple         & 1              & 1              \\
									            \hline
									            Banana        & 2              & 2              \\
									            \hline
									            Cherry        & 1              & 1              \\
									            \hline
								            \end{tabular}
							            \end{center}
						            };
					            \end{tikzpicture}
				            }
			            \end{center}
			            
			      \item \textbf{Insert the third transaction (Apple, Cherry)}:
			            
			            \begin{center}
				            \scalebox{0.85}{
					            \begin{tikzpicture}[scale=3]
						            \useasboundingbox (-2,0) rectangle (5,-1.25);
						            
						            \node[minimum height = 0.65cm, minimum width=2cm] at (0,0) (0) {\fptreerootnode};
						            
						            \node[fill=white, minimum height = 0.65cm, minimum width=2.25cm] at (-0.5,-0.5) (a2) {\fptreenodewithoccurences{$Apple$}{$2$}};
						            \node[fill=white, minimum height = 0.65cm, minimum width=2.25cm] at (-1,-1) (a2b1) {\fptreenodewithoccurences{$Banana$}{$1$}};
						            \node[fill=white, minimum height = 0.65cm, minimum width=2.25cm] at (0,-1) (a2c1) {\fptreenodewithoccurences{$Cherry$}{$1$}};
						            \node[fill=white, minimum height = 0.65cm, minimum width=2.25cm] at (1,-0.5) (b1) {\fptreenodewithoccurences{$Banana$}{$1$}};
						            \node[fill=white, minimum height = 0.65cm, minimum width=2.25cm] at (1,-1) (b1c1) {\fptreenodewithoccurences{$Cherry$}{$1$}};
						            
						            \draw (0) -- (a2) -- (a2b1);
						            \draw (a2) -- (a2c1);
						            \draw (0) -- (b1) -- (b1c1);
						            
						            % Header table
						            \node[anchor=west, text width=5cm] at (2,-0.5) (Header Table) {
							            \begin{center}
								            \textbf{Header table:}
								            
								            \vspace{0.45cm}
								            
								            \begin{tabular}{|c|c|c|}
									            \hline
									            \textbf{Item} & \textbf{Freq.} & \textbf{Nodes} \\
									            \hline
									            Apple         & 2              & 1              \\
									            \hline
									            Banana        & 2              & 2              \\
									            \hline
									            Cherry        & 2              & 2              \\
									            \hline
								            \end{tabular}
							            \end{center}
						            };
					            \end{tikzpicture}
				            }
			            \end{center}
			            
			      \item \textbf{Insert the fourth transaction (Apple, Banana)}:
			            
			            \begin{center}
				            \scalebox{0.85}{
					            \begin{tikzpicture}[scale=3]
						            \useasboundingbox (-2,0) rectangle (5,-1.25);
						            
						            \node[minimum height = 0.65cm, minimum width=2cm] at (0,0) (0) {\fptreerootnode};
						            
						            \node[fill=white, minimum height = 0.65cm, minimum width=2.25cm] at (-0.5,-0.5) (a3) {\fptreenodewithoccurences{$Apple$}{$3$}};
						            \node[fill=white, minimum height = 0.65cm, minimum width=2.25cm] at (-1,-1) (a3b2) {\fptreenodewithoccurences{$Banana$}{$2$}};
						            \node[fill=white, minimum height = 0.65cm, minimum width=2.25cm] at (0,-1) (a3c1) {\fptreenodewithoccurences{$Cherry$}{$1$}};
						            \node[fill=white, minimum height = 0.65cm, minimum width=2.25cm] at (1,-0.5) (b1) {\fptreenodewithoccurences{$Banana$}{$1$}};
						            \node[fill=white, minimum height = 0.65cm, minimum width=2.25cm] at (1,-1) (b1c1) {\fptreenodewithoccurences{$Cherry$}{$1$}};
						            
						            \draw (0) -- (a3) -- (a3b2);
						            \draw (a3) -- (a3c1);
						            \draw (0) -- (b1) -- (b1c1);
						            
						            % Header table
						            \node[anchor=west, text width=5cm] at (2,-0.5) (Header Table) {
							            \begin{center}
								            \textbf{Header table:}
								            
								            \vspace{0.45cm}
								            
								            \begin{tabular}{|c|c|c|}
									            \hline
									            \textbf{Item} & \textbf{Freq.} & \textbf{Nodes} \\
									            \hline
									            Apple         & 3              & 1              \\
									            \hline
									            Banana        & 3              & 2              \\
									            \hline
									            Cherry        & 3              & 3              \\
									            \hline
								            \end{tabular}
							            \end{center}
						            };
					            \end{tikzpicture}
				            }
			            \end{center}
			            
			      \item \textbf{Insert the fifth transaction (Apple)}:
			            
			            \begin{center}
				            \scalebox{0.85}{
					            \begin{tikzpicture}[scale=3]
						            \useasboundingbox (-2,0) rectangle (5,-1.25);
						            
						            \node[minimum height = 0.65cm, minimum width=2cm] at (0,0) (0) {\fptreerootnode};
						            
						            \node[fill=white, minimum height = 0.65cm, minimum width=2.25cm] at (-0.5,-0.5) (a4) {\fptreenodewithoccurences{$Apple$}{$4$}};
						            \node[fill=white, minimum height = 0.65cm, minimum width=2.25cm] at (-1,-1) (a4b2) {\fptreenodewithoccurences{$Banana$}{$2$}};
						            \node[fill=white, minimum height = 0.65cm, minimum width=2.25cm] at (0,-1) (a4c1) {\fptreenodewithoccurences{$Cherry$}{$1$}};
						            \node[fill=white, minimum height = 0.65cm, minimum width=2.25cm] at (1,-0.5) (b1) {\fptreenodewithoccurences{$Banana$}{$1$}};
						            \node[fill=white, minimum height = 0.65cm, minimum width=2.25cm] at (1,-1) (b1c1) {\fptreenodewithoccurences{$Cherry$}{$1$}};
						            
						            \draw (0) -- (a4) -- (a4b2);
						            \draw (a4) -- (a4c1);
						            \draw (0) -- (b1) -- (b1c1);
						            
						            % Header table
						            \node[anchor=west, text width=5cm] at (2,-0.5) (Header Table) {
							            \begin{center}
								            \textbf{Header table:}
								            
								            \vspace{0.45cm}
								            
								            \begin{tabular}{|c|c|c|}
									            \hline
									            \textbf{Item} & \textbf{Freq.} & \textbf{Nodes} \\
									            \hline
									            Apple         & 4              & 1              \\
									            \hline
									            Banana        & 3              & 2              \\
									            \hline
									            Cherry        & 2              & 2              \\
									            \hline
								            \end{tabular}
							            \end{center}
						            };
					            \end{tikzpicture}
				            }
			            \end{center}
			            
		      \end{enumerate}
		      
		\item \textbf{Determine the conditional pattern base for each frequent item in the header tables of the FP-tree}:
		      
		      
		      \begin{enumerate}
			      \item \textbf{Conditional pattern base for Apple}:
			            
			            Apple is the direct child of the root node, so the conditional pattern base for Apple is empty.
			            
			      \item \textbf{Conditional pattern base for Banana}:
			            
			            \begin{itemize}
				            \item Apple: 2
			            \end{itemize}
			            
			      \item \textbf{Conditional pattern base for Cherry}:
			            
			            \begin{itemize}
				            \item Apple: 1
				            \item Banana: 1
			            \end{itemize}
			            
		      \end{enumerate}
		      
		\item \textbf{Create the conditional FP-trees}:
		      
		      \begin{enumerate}
			      \item \textbf{Conditional FP-tree for Banana}:
			            
			            \begin{center}
				            \scalebox{0.85}{
					            \begin{tikzpicture}[scale=3]
						            \useasboundingbox (-1,0) rectangle (5,-0.6);
						            
						            \node[minimum height = 0.65cm, minimum width=2cm] at (0,0) (0) {\fptreerootnode};
						            
						            \node[fill=white, minimum height = 0.65cm, minimum width=2.25cm] at (0,-0.5) (a2) {\fptreenodewithoccurences{$Apple$}{$2$}};
						            
						            \draw (0) -- (a2);
						            
						            % Header table
						            \node[anchor=west, text width=8cm] at (2,-0.25) (Header Table) {
							            \begin{center}
								            \textbf{Header table:}
								            
								            \vspace{0.45cm}
								            
								            \begin{tabular}{|c|c|c|}
									            \hline
									            \textbf{Item}   & \textbf{Freq.} & \textbf{Nodes} \\
									            \hline
									            (Banana,) Apple & 2              & 1              \\
									            \hline
								            \end{tabular}
							            \end{center}
						            };
					            \end{tikzpicture}
				            }
			            \end{center}
			            
			      \item \textbf{Conditional FP-tree for Cherry}:
			            
			            \begin{enumerate}
				            \item \textbf{Insert \glqq Apple: 1\grqq}:
				                  
				                  \begin{center}
					                  \scalebox{0.85}{
						                  \begin{tikzpicture}[scale=3]
							                  \useasboundingbox (-1,0) rectangle (5,-0.6);
							                  
							                  \node[minimum height = 0.65cm, minimum width=2cm] at (0,0) (0) {\fptreerootnode};
							                  
							                  \node[fill=white, minimum height = 0.65cm, minimum width=2.25cm] at (-0,-0.5) (a1) {\fptreenodewithoccurences{$Apple$}{$1$}};
							                  
							                  \draw (0) -- (a1);
							                  
							                  % Header table
							                  \node[anchor=west, text width=8cm] at (2,-0.2) (Header Table) {
								                  \begin{center}
									                  \textbf{Header table:}
									                  
									                  \vspace{0.45cm}
									                  
									                  \begin{tabular}{|c|c|c|}
										                  \hline
										                  \textbf{Item}    & \textbf{Freq.} & \textbf{Nodes} \\
										                  \hline
										                  (Cherry,) Apple  & 1              & 1              \\
										                  \hline
										                  (Cherry,) Banana & 0              & 0              \\
										                  \hline
									                  \end{tabular}
								                  \end{center}
							                  };
						                  \end{tikzpicture}
					                  }
				                  \end{center}
				                  
				            \item \textbf{Insert \glqq Banana: 1\grqq}:
				                  
				                  \begin{center}
					                  \scalebox{0.85}{
						                  \begin{tikzpicture}[scale=3]
							                  \useasboundingbox (-1,0) rectangle (5,-0.6);
							                  
							                  \node[minimum height = 0.65cm, minimum width=2cm] at (0,0) (0) {\fptreerootnode};
							                  
							                  \node[fill=white, minimum height = 0.65cm, minimum width=2.25cm] at (-0.5,-0.5) (a1) {\fptreenodewithoccurences{$Apple$}{$1$}};
							                  \node[fill=white, minimum height = 0.65cm, minimum width=2.25cm] at (0.5,-0.5) (b1) {\fptreenodewithoccurences{$Banana$}{$1$}};
							                  
							                  \draw (0) -- (a1);
							                  \draw (0) -- (b1);
							                  
							                  % Header table
							                  \node[anchor=west, text width=8cm] at (2,-0.2) (Header Table) {
								                  \begin{center}
									                  \textbf{Header table:}
									                  
									                  \vspace{0.45cm}
									                  
									                  \begin{tabular}{|c|c|c|}
										                  \hline
										                  \textbf{Item}    & \textbf{Freq.} & \textbf{Nodes} \\
										                  \hline
										                  (Cherry,) Apple  & 1              & 1              \\
										                  \hline
										                  (Cherry,) Banana & 1              & 1              \\
										                  \hline
									                  \end{tabular}
								                  \end{center}
							                  };
						                  \end{tikzpicture}
					                  }
				                  \end{center}
			            \end{enumerate}
		      \end{enumerate}
		      
		\item \textbf{Determine the conditional pattern base for each frequent itemset in the header tables of the conditional FP-trees}:
		      
		      
		      \begin{enumerate}
			      \item \textbf{Conditional pattern base for \glqq Banana, Apple\grqq }:
			            
			            Apple is the direct child of the root node, so the conditional pattern base is empty.
		      \end{enumerate}
		      
		      
		\item \textbf{Termination}:
		      
		      The algorithm terminates because there are no more conditional FP-trees to create.
		      
	\end{enumerate}
	
	\textbf{Result:}
	
	\vspace*{-0.2cm}
	
	The frequent itemsets for a minimum support count of 2 are:
	
	\begin{multicols}{4}
		\begin{enumerate}
			\item Apple
			\item Banana
			\item Cherry
			\item Banana, Apple
		\end{enumerate}
	\end{multicols}
	
\end{solution}

\end{document}
