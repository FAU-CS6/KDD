\documentclass[
english,
smallborders
]{i6prcsht}
\usepackage{i6common}
\usepackage{i6lecture}

\usepackage{todonotes}
\usepackage[utf8]{inputenc}
\usepackage{textcomp}
\usepackage{pdfpages}
\usepackage{csquotes}

\hyphenation{Stud-On}

\begin{document}

\title{Exercise Sheet 1: \\ Introduction to Python and pandas}
\maketitle
\vspace*{-2cm}

\section*{About this Exercise Sheet}

This exercise sheet is a gentle introduction to the technical tools that we will use repeatedly in the exercise throughout the semester: Python and pandas.

In contrast to all other exercises, we only recommend participating in this exercise if you have no previous experience with Python and/or pandas or if you do not feel confident using them.

\section*{Preparation}

Before participating in the exercise, you must prepare the following:

\begin{enumerate}
	\item \textbf{Install Python and pip on your computer}
	      
	      \begin{itemize}
		      \item Install Python 3.8 or higher on your computer.
		            A good guide on the installation process can be found at \url{https://realpython.com/installing-python/}.
		      \item If your Python installation doesn't come with pip (the package installer for Python), install pip on your computer.
		            You can find more information on the installation process at \url{https://pip.pypa.io/en/stable/installation/}.
	      \end{itemize}
	      
	\item \textbf{Download provided additional files}
	      
	      \begin{itemize}
		      \item Download \texttt{Additional-Files-Student.zip} from StudOn
		      \item Extract it to a folder of your choice.
	      \end{itemize}
	      
\end{enumerate}

\section*{Exercise 1: Getting started}

Before we can start with the actual exercise, we have to perform some basic steps. These will be similar for all Python based exercises:

\begin{enumerate}
	\item \textbf{Install required Python packages}
	      
	      \begin{itemize}
		      \item Open a terminal and navigate to the folder where you extracted the files.
		      \item Run the command \texttt{pip install -r requirements.txt} within the extracted additional files
		            folder to install the required Python packages.
	      \end{itemize}
	      
	\item \textbf{Start the Jupyter Notebook server}
	      
	      \begin{itemize}
		      \item Run the command \texttt{jupyter notebook} within the extracted additional files
		            folder to start the Jupyter Notebook server\footnote{If you have problems starting the Jupyter Notebook server, try \texttt{python -m notebook} as an alternative command.}.
		      \item There should be a new tab in your browser with the Jupyter Notebook interface.
	      \end{itemize}
\end{enumerate}

\section*{Exercise 2: Get to know Python}

This exercise utilizes a Jupyter Notebook:

\begin{enumerate}
	\item Open \texttt{Python.ipynb} in the Jupyter Notebook interface.
	\item Take a look at the tasks (blue boxes) in the notebook and try to solve them.
\end{enumerate}

\begin{solution}
	The solution to the exercise can be found in \texttt{Additional-Files-Solution.zip}.
\end{solution}

\section*{Exercise 3: Get to know pandas}

This exercise utilizes a Jupyter Notebook:

\begin{enumerate}
	\item Open \texttt{pandas.ipynb} in the Jupyter Notebook interface.
	\item Take a look at the tasks (blue boxes) in the notebook and try to solve them.
\end{enumerate}

\begin{solution}
	The solution to the exercise can be found in \texttt{Additional-Files-Solution.zip}.
\end{solution}

\end{document}

\end{document}
