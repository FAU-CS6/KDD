\documentclass[
english,
smallborders
]{i6prcsht}
\usepackage{i6common}
\usepackage{i6lecture}

\usepackage{todonotes}
\usepackage[utf8]{inputenc}
\usepackage{textcomp}
\usepackage{pdfpages}
\usepackage{csquotes}
\usepackage{awesomebox}
\usepackage{makecell}
\usepackage{graphicx}
\usepackage{multicol}
\usepackage{amsmath}
\usepackage[]{mdframed}
\usepackage{array}
\usepackage{multirow}


\hyphenation{Stud-On}

\newcommand{\OfSpecificValue}[3]{_{\text{\tiny #1#2#3}}}
\newcommand{\OfAttribute}[1]{_{\text{\tiny #1}}}
\newcommand{\PriorProbability}[2]{P(\texttt{#1}=\text{"#2"})}
\newcommand{\Likelihood}[4]{P(\texttt{#1}=\text{"#2"} | \texttt{#3}=\text{"#4"})}
\newcommand{\LikelihoodTuple}[3]{P(#1 | \texttt{#2}=\text{"#3"})}
\newcommand{\BayesNumerator}[3]{P(#1 | \texttt{#2}=\text{"#3"}) \cdot P(\texttt{#2}=\text{"#3"})}
\newcommand{\PosterioriProbability}[3]{P(\texttt{#2}=\text{"#3"} | #1)}
\newcommand{\ResultClass}[2]{\texttt{#1}=\text{"#2"}}

\begin{document}

\title{Exercise Sheet 4: \\ Classification}
\maketitle
\vspace*{-2cm}

\section*{About this Exercise Sheet}

This exercise sheet focuses on the content of lecture \textit{7. Classification}.

It includes TODO.

The exercise sheet is designed for a three-week period, during which the tasks can be completed flexibly.

The sample solution will be published after the three weeks have elapsed.

\section*{Preparation}

Before participating in the exercise, you must prepare the following:

\begin{enumerate}
	\item \textbf{Install Python and pip on your computer}

	      \begin{itemize}
		      \item Detailed instructions can be found in \texttt{1-Introduction-Python-Pandas.pdf}.
	      \end{itemize}

	\item \textbf{Download provided additional files}

	      \begin{itemize}
		      \item Download \texttt{Additional-Files-Student.zip} from StudOn
		      \item Extract it to a folder of your choice.
	      \end{itemize}

	\item \textbf{Install required Python packages}

	      \begin{itemize}
		      \item Open a terminal and navigate to the folder where you extracted the files.
		      \item Run the command \texttt{pip install -r requirements.txt} within the extracted additional files folder to install the required Python packages.
	      \end{itemize}


\end{enumerate}

\section*{Exercise 1: Decision Trees}

Given is a dataset $D$:

\begin{center}
	\begin{tabular}{|c|c|c|c|c|c|}
		\hline
		\textbf{Age} & \textbf{Major} & \textbf{Participation} & \textbf{Passed} \\ \hline
		23           & CS             & High                   & Yes             \\ \hline
		23           & DS             & Low                    & No              \\ \hline
		26           & DS             & High                   & Yes             \\ \hline
		24           & DS             & Medium                 & Yes             \\ \hline
		26           & DS             & Medium                 & No              \\ \hline
		26           & DS             & Low                    & No              \\ \hline
	\end{tabular}
\end{center}

$D$ is containing a continuous attribute (\textit{Age}) and two categorical attributes (\textit{Major} and \textit{Participation}) which can be used to predict the target attribute \textit{Passed}.

\subsection*{Task 1: Information Gain}

Use the algorithm for \textbf{Decision Tree Induction} known from the lecture to build a decision tree for dataset $D$. The decision tree should be built using \textbf{Information Gain} as the attribute selection method.

Write down \textbf{all} intermediate steps.

\begin{solution}
	\begin{enumerate}
		\item \textbf{Create the root node:}

		      To create the root node, we need to calculate the Information Gain for each attribute and select the one with the highest Information Gain.

		      \begin{enumerate}
			      \item \textbf{Calculate the Entropy of the target attribute \textit{Passed}:}
			            \begin{alignat*}{2}
				            \text{Info}(D) & = -\sum_{i=1}^{m}p_i \log_2(p_i)                                                                                                                          \\
				                           & = -p\OfSpecificValue{Passed}{=}{Yes} \log_2(p\OfSpecificValue{Passed}{=}{Yes})- p\OfSpecificValue{Passed}{=}{No} \log_2(p\OfSpecificValue{Passed}{=}{No}) \\
				                           & = -\frac{3}{6} \log_2\left(\frac{3}{6}\right) - \frac{3}{6} \log_2\left(\frac{3}{6}\right)                                                                \\
				                           & = 1                                                                                                                                                       \\
			            \end{alignat*}

			      \item \textbf{Calculate the Information Gain for all attributes:}

			            \begin{enumerate}
				            \item \textbf{Attribute \textit{Age}:}

				                  \textit{Age} is a continuous attribute. To calculate the Information Gain, we need to find the best split point.

				                  \begin{enumerate}
					                  \item \textbf{Split point $23,5$:}

					                        \begin{alignat*}{2}
						                        \text{Info}\OfAttribute{Age}(D) & = \sum_{j=1}^v \frac{|D\OfAttribute{Age$,j$}|}{|D\OfAttribute{Age}|} \text{Info}(D_{A\OfAttribute{Age$,j$}})                                                                                                                            \\
						                                                        & = \frac{|D\OfSpecificValue{Age}{$\leq$}{2,5}|}{|D\OfAttribute{Age}|} \text{Info}(D\OfSpecificValue{Age}{$\leq$}{2,5}) + \frac{|D\OfSpecificValue{Age}{$>$}{2,5}|}{|D\OfAttribute{Age}|} \text{Info}(D\OfSpecificValue{Age}{$>$}{2,5})   \\
						                                                        & = \frac{2}{6} \left(-\frac{1}{2} \log_2\left(\frac{1}{2}\right) - \frac{1}{2} \log_2\left(\frac{1}{2}\right)\right) + \frac{4}{6} \left(-\frac{2}{4} \log_2\left(\frac{2}{4}\right) - \frac{2}{4} \log_2\left(\frac{2}{4}\right)\right) \\
						                                                        & = \frac{2}{6} \cdot 1 + \frac{4}{6} \cdot 1                                                                                                                                                                                             \\
						                                                        & = 1                                                                                                                                                                                                                                     \\
						                        \text{Gain}\OfAttribute{Age}    & = \text{Info}(D) - \text{Info}\OfAttribute{Age}(D)                                                                                                                                                                                      \\
						                                                        & = 1 - 1                                                                                                                                                                                                                                 \\
						                                                        & = 0                                                                                                                                                                                                                                     \\
					                        \end{alignat*}

					                  \item \textbf{Split point $25,5$:}

					                        \begin{alignat*}{2}
						                        \text{Info}\OfAttribute{Age}(D) & = \sum_{j=1}^v \frac{|D\OfAttribute{Age$,j$}|}{|D\OfAttribute{Age}|} \text{Info}(D_{A\OfAttribute{Age$,j$}})                                                                                                                            \\
						                                                        & = \frac{|D\OfSpecificValue{Age}{$\leq$}{5}|}{|D\OfAttribute{Age}|} \text{Info}(D\OfSpecificValue{Age}{$\leq$}{5}) + \frac{|D\OfSpecificValue{Age}{$>$}{5}|}{|D\OfAttribute{Age}|} \text{Info}(D\OfSpecificValue{Age}{$>$}{5})           \\
						                                                        & = \frac{3}{6} \left(-\frac{2}{3} \log_2\left(\frac{2}{3}\right) - \frac{1}{3} \log_2\left(\frac{1}{3}\right)\right) + \frac{3}{6} \left(-\frac{1}{3} \log_2\left(\frac{1}{3}\right) - \frac{2}{3} \log_2\left(\frac{2}{3}\right)\right) \\
						                                                        & = \frac{3}{6} \cdot 0,9183 + \frac{3}{6} \cdot 0,9183                                                                                                                                                                                   \\
						                                                        & = 0,9183                                                                                                                                                                                                                                \\
						                        \text{Gain}\OfAttribute{Age}    & = \text{Info}(D) - \text{Info}\OfAttribute{Age}(D)                                                                                                                                                                                      \\
						                                                        & = 1 - 0,9183                                                                                                                                                                                                                            \\
						                                                        & = 0,0817                                                                                                                                                                                                                                \\
					                        \end{alignat*}

				                  \end{enumerate}

				                  Therefore, the Information Gain for the attribute \textit{Age} is $0,817$ (if we split at $25,5$).

				            \item \textbf{Attribute \textit{Major}:}

				                  \textit{Major} is a categorical attribute with two possible values: \textit{CS} and \textit{DS}.

				                  Since it is a categorical attribute and we are using the Information Gain, there is no need to determine a splitting criterion.

				                  \begin{alignat*}{2}
					                  \text{Info}\OfAttribute{Major}(D) & = \sum_{j=1}^v \frac{|D\OfAttribute{Major$,j$}|}{|D\OfAttribute{Major}|} \text{Info}(D_{A\OfAttribute{Major$,j$}})                                                                                                                      \\
					                                                    & = \frac{|D\OfSpecificValue{Major}{=}{CS}|}{|D\OfAttribute{Major}|} \text{Info}(D\OfSpecificValue{Major}{=}{CS}) + \frac{|D\OfSpecificValue{Major}{=}{DS}|}{|D\OfAttribute{Major}|} \text{Info}(D\OfSpecificValue{Major}{=}{DS})         \\
					                                                    & = \frac{1}{6} \left(-\frac{1}{1} \log_2\left(\frac{1}{1}\right) - \frac{0}{1} \log_2\left(\frac{0}{1}\right)\right) + \frac{5}{6} \left(-\frac{2}{5} \log_2\left(\frac{2}{5}\right) - \frac{3}{5} \log_2\left(\frac{3}{5}\right)\right) \\
					                                                    & = \frac{1}{6} \left(- 0 - 0 \cdot \text{undefined}\right) + \frac{5}{6} \left(0,9710\right) \text{\tiny \textit{\hspace{0,2cm}\textbf{Hint:} Multiplication by 0 always results in 0}}                                                  \\
					                                                    & = \frac{1}{6} \left(- 0 - 0 \right) + \frac{5}{6} \left(0,9710\right)                                                                                                                                                                   \\
					                                                    & = 0,8090                                                                                                                                                                                                                                \\
					                  \text{Gain}\OfAttribute{Major}    & = \text{Info}(D) - \text{Info}\OfAttribute{Major}(D)                                                                                                                                                                                    \\
					                                                    & = 1 - 0,8090                                                                                                                                                                                                                            \\
					                                                    & = 0,1910                                                                                                                                                                                                                                \\
				                  \end{alignat*}

				            \item \textbf{Attribute \textit{Participation}:}

				                  \textit{Participation} is a categorical attribute with three possible values: \textit{High}, \textit{Medium} and \textit{Low}.

				                  Since it is a categorical attribute and we are using the Information Gain, there is no need to determine a splitting criterion.

				                  \begin{alignat*}{3}
					                  \text{Info}\OfAttribute{Parti.}(D) & = \sum_{j=1}^v \frac{|D\OfAttribute{Parti.$,j$}|}{|D\OfAttribute{Parti.}|} \text{Info}(D_{A\OfAttribute{Parti.$,j$}})                                                                                                                             \\
					                                                     & = \frac{|D\OfSpecificValue{Parti.}{=}{High}|}{|D\OfAttribute{Parti.}|} \text{Info}(D\OfSpecificValue{Parti.}{=}{High}) + \frac{|D\OfSpecificValue{Parti.}{=}{Medium}|}{|D\OfAttribute{Parti.}|} \text{Info}(D\OfSpecificValue{Parti.}{=}{Medium}) \\
					                                                     & + \frac{|D\OfSpecificValue{Parti.}{=}{Low}|}{|D\OfAttribute{Parti.}|} \text{Info}(D\OfSpecificValue{Parti.}{=}{Low})                                                                                                                              \\
					                                                     & = \frac{2}{6} \left(-\frac{2}{2} \log_2\left(\frac{2}{2}\right) - \frac{0}{2} \log_2\left(\frac{0}{2}\right)\right) + \frac{2}{6} \left(-\frac{1}{2} \log_2\left(\frac{1}{2}\right) - \frac{1}{2} \log_2\left(\frac{1}{2}\right)\right)           \\
					                                                     & + \frac{2}{6} \left(-\frac{0}{2} \log_2\left(\frac{0}{2}\right) - \frac{2}{2} \log_2\left(\frac{2}{2}\right)\right)                                                                                                                               \\
					                                                     & = \frac{2}{6} \cdot 0 + \frac{2}{6} \cdot 1 + \frac{2}{6} \cdot 0                                                                                                                                                                                 \\
					                                                     & =  0,3333                                                                                                                                                                                                                                         \\
					                  \text{Gain}\OfAttribute{Parti.}    & = \text{Info}(D) - \text{Info}\OfAttribute{Parti.}(D)                                                                                                                                                                                             \\
					                                                     & = 1 - 0,3333                                                                                                                                                                                                                                      \\
					                                                     & = 0,6667                                                                                                                                                                                                                                          \\
				                  \end{alignat*}
			            \end{enumerate}

			      \item \textbf{Create the node based on the highest Information Gain:}

			            The attribute with the highest Information Gain is \textit{Participation} with a value of $0,6667$. It will therefore become the splitting attribute for the root node.

			            The resulting tree will look like this:

			            \begin{center}
				            \begin{tikzpicture}[
						            >=latex,
						            thick,
						            node/.style={
								            draw,
								            rounded corners=.25em,
								            text depth=0.2em
							            },
						            leaf/.style={
								            draw,
								            rounded corners=.7em,
								            text depth=0.2em
							            },
						            branch/.style={
								            fill=white,
								            font=\ttfamily\scriptsize,
								            rounded corners=.7em,
								            text depth=0.2em
							            }
					            ]
					            \useasboundingbox (-6,0.5) rectangle (6,-6);

					            \node[node, anchor=center] at (0,0) (parti) {Participation?};

					            \node[draw=none,anchor=center] at (-4,-2.5) (high) {
						            \resizebox{6.5cm}{!}{%
							            \begin{tabular}{|c|c|c|c|c|}
								            \hline
								            \textbf{Age} & \textbf{Major} & \textbf{Participation} & \textbf{Passed} \\ \hline
								            23           & CS             & High                   & Yes             \\ \hline
								            26           & DS             & High                   & Yes             \\ \hline
							            \end{tabular}
						            }
					            };

					            \draw[rounded corners=5pt] (parti.south) -- (0,-1) -- node[above, draw=none] {high} (-4,-1) -- (high.north);

					            \node[draw=none,anchor=center] at (0,-4.25)  (medium) {
						            \resizebox{6.5cm}{!}{%
							            \begin{tabular}{|c|c|c|c|c|}
								            \hline
								            \textbf{Age} & \textbf{Major} & \textbf{Participation} & \textbf{Passed} \\ \hline
								            24           & DS             & Medium                 & Yes             \\ \hline
								            26           & DS             & Medium                 & No              \\ \hline
							            \end{tabular}
						            }
					            };

					            \draw[rounded corners=5pt] (parti.south) -- (0,-1) -- node[above, draw=none, fill=white] {medium} (0,-2.5) -- (medium.north);

					            \node[draw=none,anchor=center] at (4,-2.5)  (low) {
						            \resizebox{6.5cm}{!}{%
							            \begin{tabular}{|c|c|c|c|c|}
								            \hline
								            \textbf{Age} & \textbf{Major} & \textbf{Participation} & \textbf{Passed} \\ \hline
								            23           & DS             & Low                    & No              \\ \hline
								            26           & DS             & Low                    & No              \\ \hline
							            \end{tabular}
						            }
					            };

					            \draw[rounded corners=5pt] (parti.south) -- (0,-1) -- node[above, draw=none] {low} (4,-1) -- (low.north);

				            \end{tikzpicture}
			            \end{center}
		      \end{enumerate}

		\item \textbf{Visit each branch:}

		      \begin{enumerate}
			      \item \textbf{Branch \textit{High}:}

			            All samples in the partial dataset of the branch \textit{High} have the same value for the target attribute \textit{Passed}. Therefore, the branch becomes a leaf node.

			            \begin{center}
				            \begin{tikzpicture}[
						            >=latex,
						            thick,
						            node/.style={
								            draw,
								            rounded corners=.25em,
								            text depth=0.2em
							            },
						            leaf/.style={
								            draw,
								            rounded corners=.7em,
								            text depth=0.2em
							            },
						            branch/.style={
								            fill=white,
								            font=\ttfamily\scriptsize,
								            rounded corners=.7em,
								            text depth=0.2em
							            }
					            ]
					            \useasboundingbox (-6,0.5) rectangle (6,-6);

					            \node[node, anchor=center] at (0,0) (parti) {Participation?};

					            \node[leaf] at (-4,-2.5) (high) {Yes};

					            \draw[rounded corners=5pt] (parti.south) -- (0,-1) -- node[above, draw=none] {high} (-4,-1) -- (high.north);

					            \node[draw=none,anchor=center] at (0,-4.25)  (medium) {
						            \resizebox{6.5cm}{!}{%
							            \begin{tabular}{|c|c|c|c|c|}
								            \hline
								            \textbf{Age} & \textbf{Major} & \textbf{Participation} & \textbf{Passed} \\ \hline
								            24           & DS             & Medium                 & Yes             \\ \hline
								            26           & DS             & Medium                 & No              \\ \hline
							            \end{tabular}
						            }
					            };

					            \draw[rounded corners=5pt] (parti.south) -- (0,-1) -- node[above, draw=none, fill=white] {medium} (0,-2.5) -- (medium.north);

					            \node[draw=none,anchor=center] at (4,-2.5)  (low) {
						            \resizebox{6.5cm}{!}{%
							            \begin{tabular}{|c|c|c|c|c|}
								            \hline
								            \textbf{Age} & \textbf{Major} & \textbf{Participation} & \textbf{Passed} \\ \hline
								            23           & DS             & Low                    & No              \\ \hline
								            26           & DS             & Low                    & No              \\ \hline
							            \end{tabular}
						            }
					            };

					            \draw[rounded corners=5pt] (parti.south) -- (0,-1) -- node[above, draw=none] {low} (4,-1) -- (low.north);

				            \end{tikzpicture}
			            \end{center}

			      \item \textbf{Branch \textit{Medium}:}

			            The partial dataset of the branch \textit{Medium} contains samples with different values for the target attribute \textit{Passed}. Therefore, we need to create a new node.

			            \begin{enumerate}
				            \item \textbf{Calculate the Entropy of the target attribute \textit{Passed}:}

				                  \begin{alignat*}{2}
					                  \text{Info}(D\OfSpecificValue{Parti}{=}{Medium}) & = -\sum_{i=1}^{m}p_i \log_2(p_i)                                                                                                                          \\
					                                                                   & = -p\OfSpecificValue{Passed}{=}{Yes} \log_2(p\OfSpecificValue{Passed}{=}{Yes})- p\OfSpecificValue{Passed}{=}{No} \log_2(p\OfSpecificValue{Passed}{=}{No}) \\
					                                                                   & = -\frac{1}{2} \log_2\left(\frac{1}{2}\right) - \frac{1}{2} \log_2\left(\frac{1}{2}\right)                                                                \\
					                                                                   & = 1                                                                                                                                                       \\
				                  \end{alignat*}

				            \item \textbf{Calculate the Information Gain for all attributes that are not yet a node:}

				                  \begin{enumerate}
					                  \item \textbf{Attribute \textit{Age}:}

					                        \textit{Age} is still a continuous attribute. However since there are only two different values in the partial dataset, we only have one split point.

					                        \begin{alignat*}{2}
						                        \text{Info}\OfAttribute{Age}(D\OfSpecificValue{Parti}{=}{Medium}) & = \sum_{j=1}^v \frac{|D\OfAttribute{Age$,j$}|}{|D\OfAttribute{Age}|} \text{Info}(D_{A\OfAttribute{Age$,j$}})                                                                                                                            \\
						                                                                                          & = \frac{|D\OfSpecificValue{Age}{$\leq$}{25}|}{|D\OfAttribute{Age}|} \text{Info}(D\OfSpecificValue{Age}{$\leq$}{25}) + \frac{|D\OfSpecificValue{Age}{$>$}{25}|}{|D\OfAttribute{Age}|} \text{Info}(D\OfSpecificValue{Age}{$>$}{25})       \\
						                                                                                          & = \frac{1}{2} \left(-\frac{1}{1} \log_2\left(\frac{1}{1}\right) - \frac{0}{1} \log_2\left(\frac{0}{1}\right)\right) + \frac{1}{2} \left(-\frac{1}{1} \log_2\left(\frac{1}{1}\right) - \frac{0}{1} \log_2\left(\frac{0}{1}\right)\right) \\
						                                                                                          & = \frac{1}{2} \cdot 0 + \frac{1}{2} \cdot 0                                                                                                                                                                                             \\
						                                                                                          & = 0                                                                                                                                                                                                                                     \\
						                        \text{Gain}\OfAttribute{Age}                                      & = \text{Info}(D\OfSpecificValue{Parti}{=}{Medium}) - \text{Info}\OfAttribute{Age}(D\OfSpecificValue{Parti}{=}{Medium})                                                                                                                  \\
						                                                                                          & = 1 - 0                                                                                                                                                                                                                                 \\
						                                                                                          & = 1                                                                                                                                                                                                                                     \\
					                        \end{alignat*}

					                  \item \textbf{Attribute \textit{Major}:}

					                        \textit{Major} is still a categorical attribute. This time we only have one value in the partial dataset. Therefore, the Information Gain is $0$:

					                        \begin{alignat*}{2}
						                        \text{Info}\OfAttribute{Major}(D\OfSpecificValue{Parti}{=}{Medium}) & = \sum_{j=1}^v \frac{|D\OfAttribute{Major$,j$}|}{|D\OfAttribute{Major}|} \text{Info}(D_{A\OfAttribute{Major$,j$}})       \\
						                                                                                            & = \frac{|D\OfSpecificValue{Major}{=}{DS}|}{|D\OfAttribute{Major}|} \text{Info}(D\OfSpecificValue{Major}{=}{DS})          \\
						                                                                                            & = \frac{2}{2} \left(-\frac{1}{2} \log_2\left(\frac{1}{2}\right) - \frac{1}{2} \log_2\left(\frac{1}{2}\right)\right)      \\
						                                                                                            & = \frac{2}{2} \left(1\right)                                                                                             \\
						                                                                                            & = 1                                                                                                                      \\
						                        \text{Gain}\OfAttribute{Major}                                      & = \text{Info}(D\OfSpecificValue{Parti}{=}{Medium}) - \text{Info}\OfAttribute{Major}(D\OfSpecificValue{Parti}{=}{Medium}) \\
						                                                                                            & = 1 - 1                                                                                                                  \\
						                                                                                            & = 0                                                                                                                      \\
					                        \end{alignat*}
				                  \end{enumerate}

				            \item \textbf{Create the node based on the highest Information Gain:}

				                  The attribute with the highest Information Gain is \textit{Age} with a value of $1$. It will therefore become the splitting attribute for the node.

				                  The resulting tree will look like this:

				                  \begin{center}
					                  \begin{tikzpicture}[
							                  >=latex,
							                  thick,
							                  node/.style={
									                  draw,
									                  rounded corners=.25em,
									                  text depth=0.2em
								                  },
							                  leaf/.style={
									                  draw,
									                  rounded corners=.7em,
									                  text depth=0.2em
								                  },
							                  branch/.style={
									                  fill=white,
									                  font=\ttfamily\scriptsize,
									                  rounded corners=.7em,
									                  text depth=0.2em
								                  }
						                  ]
						                  \useasboundingbox (-6,0.5) rectangle (6,-7);

						                  \node[node, anchor=center] at (0,0) (parti) {Participation?};

						                  \node[leaf] at (-4,-2.5) (high) {Yes};

						                  \draw[rounded corners=5pt] (parti.south) -- (0,-1) -- node[above, draw=none] {high} (-4,-1) -- (high.north);

						                  \node[node,anchor=center] at (0,-4.25)  (medium) {Age?};

						                  \draw[rounded corners=5pt] (parti.south) -- (0,-1) -- node[above, draw=none, fill=white] {medium} (0,-2.5) -- (medium.north);

						                  \node[draw=none,anchor=center] at (4,-2.5)  (low) {
							                  \resizebox{6.5cm}{!}{%
								                  \begin{tabular}{|c|c|c|c|c|}
									                  \hline
									                  \textbf{Age} & \textbf{Major} & \textbf{Participation} & \textbf{Passed} \\ \hline
									                  23           & DS             & Low                    & No              \\ \hline
									                  26           & DS             & Low                    & No              \\ \hline
								                  \end{tabular}
							                  }
						                  };

						                  \draw[rounded corners=5pt] (parti.south) -- (0,-1) -- node[above, draw=none] {low} (4,-1) -- (low.north);

						                  \node[leaf] at (-2,-6) (yes) {Yes};
						                  \node[leaf] at (2,-6) (no) {No};

						                  \draw[rounded corners=5pt] (medium.south) -- (0,-5) -- node[left, draw=none, fill=white] {$\leq 25$} (-1,-5) -- (-2,-5) -- (yes.north);

						                  \draw[rounded corners=5pt] (medium.south) -- (0,-5) -- node[right, draw=none, fill=white] {$> 25$} (1,-5) -- (2,-5) -- (no.north);
					                  \end{tikzpicture}
				                  \end{center}

			            \end{enumerate}
			      \item \textbf{Branch \textit{Low}:}

			            All samples in the partial dataset of the branch \textit{Low} have the same value for the target attribute \textit{Passed}. Therefore, the branch becomes a leaf node.

			            \begin{center}
				            \begin{tikzpicture}[
						            >=latex,
						            thick,
						            node/.style={
								            draw,
								            rounded corners=.25em,
								            text depth=0.2em
							            },
						            leaf/.style={
								            draw,
								            rounded corners=.7em,
								            text depth=0.2em
							            },
						            branch/.style={
								            fill=white,
								            font=\ttfamily\scriptsize,
								            rounded corners=.7em,
								            text depth=0.2em
							            }
					            ]
					            \useasboundingbox (-6,0.5) rectangle (6,-5);

					            \node[node, anchor=center] at (0,0) (parti) {Participation?};

					            \node[leaf] at (-4,-2.5) (high) {Yes};

					            \draw[rounded corners=5pt] (parti.south) -- (0,-1) -- node[above, draw=none] {high} (-4,-1) -- (high.north);

					            \node[node,anchor=center] at (0,-2.75)  (medium) {Age?};

					            \draw[rounded corners=5pt] (parti.south) -- (0,-1) -- node[above, draw=none, fill=white] {medium} (0,-2.5) -- (medium.north);

					            \node[leaf,anchor=center] at (4,-2.5)  (low) {No};

					            \draw[rounded corners=5pt] (parti.south) -- (0,-1) -- node[above, draw=none] {low} (4,-1) -- (low.north);

					            \node[leaf] at (-2,-4) (yes) {Yes};
					            \node[leaf] at (2,-4) (no) {No};

					            \draw[rounded corners=5pt] (medium.south) -- (0,-3.5) -- node[left, draw=none, fill=white] {$\leq 25$} (-1,-3.5) -- (-2,-3.5) -- (yes.north);

					            \draw[rounded corners=5pt] (medium.south) -- (0,-3.5) -- node[right, draw=none, fill=white] {$> 25$} (1,-3.5) -- (2,-3.5) -- (no.north);
				            \end{tikzpicture}
			            \end{center}

		      \end{enumerate}

		\item \textbf{Stop the algorithm:}

		      Since all branches are now leaf nodes, the algorithm can be stopped.

		      The final decision tree is:

		      \begin{center}
			      \begin{tikzpicture}[
					      >=latex,
					      thick,
					      node/.style={
							      draw,
							      rounded corners=.25em,
							      text depth=0.2em
						      },
					      leaf/.style={
							      draw,
							      rounded corners=.7em,
							      text depth=0.2em
						      },
					      branch/.style={
							      fill=white,
							      font=\ttfamily\scriptsize,
							      rounded corners=.7em,
							      text depth=0.2em
						      }
				      ]
				      \useasboundingbox (-6,0.5) rectangle (6,-5);

				      \node[node, anchor=center] at (0,0) (parti) {Participation?};

				      \node[leaf] at (-4,-2.5) (high) {Yes};

				      \draw[rounded corners=5pt] (parti.south) -- (0,-1) -- node[above, draw=none] {high} (-4,-1) -- (high.north);

				      \node[node,anchor=center] at (0,-2.75)  (medium) {Age?};

				      \draw[rounded corners=5pt] (parti.south) -- (0,-1) -- node[above, draw=none, fill=white] {medium} (0,-2.5) -- (medium.north);

				      \node[leaf,anchor=center] at (4,-2.5)  (low) {No};

				      \draw[rounded corners=5pt] (parti.south) -- (0,-1) -- node[above, draw=none] {low} (4,-1) -- (low.north);

				      \node[leaf] at (-2,-4) (yes) {Yes};
				      \node[leaf] at (2,-4) (no) {No};

				      \draw[rounded corners=5pt] (medium.south) -- (0,-3.5) -- node[left, draw=none, fill=white] {$\leq 25$} (-1,-3.5) -- (-2,-3.5) -- (yes.north);

				      \draw[rounded corners=5pt] (medium.south) -- (0,-3.5) -- node[right, draw=none, fill=white] {$> 25$} (1,-3.5) -- (2,-3.5) -- (no.north);
			      \end{tikzpicture}
		      \end{center}
	\end{enumerate}
\end{solution}

\subsection*{Task 2: Gini Index}

This time, the decision tree for dataset $D$ should be built using the \textbf{Gini Index} as the attribute selection method.

\textbf{Task 2.a: Root Node}

Using the algorithm for building a decision tree with the Gini Index, create the root node of the decision tree for the dataset $D$.

Write down \textbf{all} intermediate steps \textbf{up to} (and including) the point where the root node is created.

\begin{solution}
	\begin{enumerate}
		\item \textbf{Create the root node:}

		      To create the root node, we need to calculate the Gini Index for each attribute and select the one with the lowest Gini Index.

		      \begin{enumerate}
			      \item \textbf{Calculate the impurity of the whole Dataset $D$:}

			            \begin{alignat*}{2}
				            \text{Gini}(D) & = 1 - \sum_{i=1}^{m}p_i^2                                                      \\
				                           & = 1 - p\OfSpecificValue{Passed}{=}{Yes}^2 - p\OfSpecificValue{Passed}{=}{No}^2 \\
				                           & = 1 - \left(\frac{3}{6}\right)^2 - \left(\frac{3}{6}\right)^2                  \\
				                           & = 0,5                                                                          \\
			            \end{alignat*}

			      \item \textbf{Calculate the Gini Index for all attributes:}

			            \begin{enumerate}
				            \item \textbf{Attribute \textit{Age}:}

				                  \textit{Age} is a continuous attribute. Similar to the Information Gain, we need to find the best split point.

				                  \begin{enumerate}
					                  \item \textbf{Split point $23,5$:}

					                        \begin{alignat*}{2}
						                        \text{Gini}\OfAttribute{Age}(D)       & = \frac{|D\OfSpecificValue{Age}{$\leq$}{23,5}|}{|D\OfAttribute{Age}|} \text{Gini}(D\OfSpecificValue{Age}{$\leq$}{23,5}) + \frac{|D\OfSpecificValue{Age}{$>$}{23,5}|}{|D\OfAttribute{Age}|} \text{Gini}(D\OfSpecificValue{Age}{$>$}{23,5}) \\
						                                                              & = \frac{2}{6} \left(1 - \left(\frac{1}{2}\right)^2 - \left(\frac{1}{2}\right)^2\right) + \frac{4}{6} \left(1 - \left(\frac{2}{4}\right)^2 - \left(\frac{2}{4}\right)^2\right)                                                             \\
						                                                              & = \frac{2}{6} \cdot 0,5 + \frac{4}{6} \cdot 0,5                                                                                                                                                                                           \\
						                                                              & = 0,5                                                                                                                                                                                                                                     \\
						                        \Delta\text{Gini}\OfAttribute{Age}(D) & = \text{Gini}(D) - \text{Gini}\OfAttribute{Age}(D)                                                                                                                                                                                        \\
						                                                              & = 0,5 - 0,5                                                                                                                                                                                                                               \\
						                                                              & = 0                                                                                                                                                                                                                                       \\
					                        \end{alignat*}
					                  \item \textbf{Split point $25,5$:}

					                        \begin{alignat*}{2}
						                        \text{Gini}\OfAttribute{Age}(D)       & = \frac{|D\OfSpecificValue{Age}{$\leq$}{25,5}|}{|D\OfAttribute{Age}|} \text{Gini}(D\OfSpecificValue{Age}{$\leq$}{25,5}) + \frac{|D\OfSpecificValue{Age}{$>$}{25,5}|}{|D\OfAttribute{Age}|} \text{Gini}(D\OfSpecificValue{Age}{$>$}{25,5}) \\
						                                                              & = \frac{3}{6} \left(1 - \left(\frac{2}{3}\right)^2 - \left(\frac{1}{3}\right)^2\right) + \frac{3}{6} \left(1 - \left(\frac{1}{3}\right)^2 - \left(\frac{2}{3}\right)^2\right)                                                             \\
						                                                              & = \frac{3}{6} \cdot 0,4444 + \frac{3}{6} \cdot 0,4444                                                                                                                                                                                     \\
						                                                              & = 0,4444                                                                                                                                                                                                                                  \\
						                        \Delta\text{Gini}\OfAttribute{Age}(D) & = \text{Gini}(D) - \text{Gini}\OfAttribute{Age}(D)                                                                                                                                                                                        \\
						                                                              & = 0,5 - 0,4444                                                                                                                                                                                                                            \\
						                                                              & = 0,0556                                                                                                                                                                                                                                  \\
					                        \end{alignat*}
				                  \end{enumerate}

				                  The best split point is $25,5$ since its Gini Index is the lowest ($0,4444$) and therefore the reduction of impurity ($0,0556$) is the highest.

				            \item \textbf{Attribute \textit{Major}:}

				                  \textit{Major} is a categorical attribute with two possible values: \textit{CS} and \textit{DS}.

				                  Gini Index only supports two-way splits. Therefore if we would have had more than two values, we would have needed to calculate the Gini Index for each possible split.

				                  However since we only have two values, we can directly calculate the Gini Index for the attribute \textit{Major}:

				                  \begin{alignat*}{2}
					                  \text{Gini}\OfAttribute{Major}(D)       & = \frac{|D\OfSpecificValue{Major}{=}{CS}|}{|D\OfAttribute{Major}|} \text{Gini}(D\OfSpecificValue{Major}{=}{CS}) + \frac{|D\OfSpecificValue{Major}{=}{DS}|}{|D\OfAttribute{Major}|} \text{Gini}(D\OfSpecificValue{Major}{=}{DS}) \\
					                                                          & = \frac{1}{6} \left(1 - \left(\frac{1}{1}\right)^2 - \left(\frac{0}{1}\right)^2\right) + \frac{5}{6} \left(1 - \left(\frac{2}{5}\right)^2 - \left(\frac{3}{5}\right)^2\right)                                                   \\
					                                                          & = \frac{1}{6} \cdot 0 + \frac{5}{6} \cdot 0,48                                                                                                                                                                                  \\
					                                                          & = 0,4                                                                                                                                                                                                                           \\
					                  \Delta\text{Gini}\OfAttribute{Major}(D) & = \text{Gini}(D) - \text{Gini}\OfAttribute{Major}(D)                                                                                                                                                                            \\
					                                                          & = 0,5 - 0,4                                                                                                                                                                                                                     \\
					                                                          & = 0,1                                                                                                                                                                                                                           \\
				                  \end{alignat*}

				            \item \textbf{Attribute \textit{Participation}:}

				                  \textit{Participation} is a categorical attribute with three possible values: \textit{High}, \textit{Medium} and \textit{Low}.

				                  Since we have more than two different attributes, we have to calculate the Gini Index for each possible attribute combination.

				                  \begin{enumerate}
					                  \item \textbf{Combination \textit{\{High, Medium\}} and \textit{\{Low\}}:}

					                        \begin{alignat*}{2}
						                        \text{Gini}\OfAttribute{Parti.}(D)       & = \frac{|D\OfSpecificValue{Parti.}{=}{High, Medium}|}{|D\OfAttribute{Parti.}|} \text{Gini}(D\OfSpecificValue{Major}{=}{High, Medium}) + \frac{|D\OfSpecificValue{Parti.}{=}{Low}|}{|D\OfAttribute{Parti.}|} \text{Gini}(D\OfSpecificValue{Parti.}{=}{Low}) \\
						                                                                 & = \frac{4}{6} \left(1 - \left(\frac{3}{4}\right)^2 - \left(\frac{1}{4}\right)^2\right) + \frac{2}{6} \left(1 - \left(\frac{0}{2}\right)^2 - \left(\frac{2}{2}\right)^2\right)                                                                              \\
						                                                                 & = \frac{4}{6} \cdot 0,375 + \frac{2}{6} \cdot 0                                                                                                                                                                                                            \\
						                                                                 & = 0,25                                                                                                                                                                                                                                                     \\
						                        \Delta\text{Gini}\OfAttribute{Parti.}(D) & = \text{Gini}(D) - \text{Gini}\OfAttribute{Parti.}(D)                                                                                                                                                                                                      \\
						                                                                 & = 0,5 - 0,25                                                                                                                                                                                                                                               \\
						                                                                 & = 0,25                                                                                                                                                                                                                                                     \\
					                        \end{alignat*}

					                  \item \textbf{Combination \textit{\{High, Low\}} and \textit{\{Medium\}}:}

					                        \begin{alignat*}{2}
						                        \text{Gini}\OfAttribute{Parti.}(D)       & = \frac{|D\OfSpecificValue{Parti.}{=}{High, Low}|}{|D\OfAttribute{Parti.}|} \text{Gini}(D\OfSpecificValue{Major}{=}{High, Low}) + \frac{|D\OfSpecificValue{Parti.}{=}{Medium}|}{|D\OfAttribute{Parti.}|} \text{Gini}(D\OfSpecificValue{Parti.}{=}{Medium}) \\
						                                                                 & = \frac{4}{6} \left(1 - \left(\frac{2}{4}\right)^2 - \left(\frac{2}{4}\right)^2\right) + \frac{2}{6} \left(1 - \left(\frac{1}{2}\right)^2 - \left(\frac{1}{2}\right)^2\right)                                                                              \\
						                                                                 & = \frac{4}{6} \cdot 0,5 + \frac{2}{6} \cdot 0,5                                                                                                                                                                                                            \\
						                                                                 & = 0,5                                                                                                                                                                                                                                                      \\
						                        \Delta\text{Gini}\OfAttribute{Parti.}(D) & = \text{Gini}(D) - \text{Gini}\OfAttribute{Parti.}(D)                                                                                                                                                                                                      \\
						                                                                 & = 0,5 - 0,5                                                                                                                                                                                                                                                \\
						                                                                 & = 0                                                                                                                                                                                                                                                        \\
					                        \end{alignat*}
					                  \item \textbf{Combination \textit{\{Medium, Low\}} and \textit{\{High\}}:}

					                        \begin{alignat*}{2}
						                        \text{Gini}\OfAttribute{Parti.}(D)       & = \frac{|D\OfSpecificValue{Parti.}{=}{Medium, Low}|}{|D\OfAttribute{Parti.}|} \text{Gini}(D\OfSpecificValue{Major}{=}{Medium, Low}) + \frac{|D\OfSpecificValue{Parti.}{=}{High}|}{|D\OfAttribute{Parti.}|} \text{Gini}(D\OfSpecificValue{Parti.}{=}{High}) \\
						                                                                 & = \frac{4}{6} \left(1 - \left(\frac{1}{4}\right)^2 - \left(\frac{3}{4}\right)^2\right) + \frac{2}{6} \left(1 - \left(\frac{2}{2}\right)^2 - \left(\frac{0}{2}\right)^2\right)                                                                              \\
						                                                                 & = \frac{4}{6} \cdot 0,375 + \frac{2}{6} \cdot 0                                                                                                                                                                                                            \\
						                                                                 & = 0,25                                                                                                                                                                                                                                                     \\
						                        \Delta\text{Gini}\OfAttribute{Parti.}(D) & = \text{Gini}(D) - \text{Gini}\OfAttribute{Parti.}(D)                                                                                                                                                                                                      \\
						                                                                 & = 0,5 - 0,25                                                                                                                                                                                                                                               \\
						                                                                 & = 0,25                                                                                                                                                                                                                                                     \\
					                        \end{alignat*}
				                  \end{enumerate}

				                  The best combinations are \textit{\{High, Medium\}} and \textit{\{Low\}} and \textit{\{Medium, Low\}} and \textit{\{High\}} since their Gini Index is the lowest ($0,25$) and therefore the reduction of impurity ($0,25$) is the highest.

				                  We can choose either of them as the combination \glqq representing\grqq\ the attribute \textit{Participation}. For the simplicity of this sample solution, we will choose the combination \textit{\{High, Medium\}} and \textit{\{Low\}}.
			            \end{enumerate}

			      \item \textbf{Create the root node based on the lowest Gini Index:}

			            The attribute with the lowest Gini Index is \textit{Participation} with a value of $0,25$. It will therefore become the splitting attribute for the root node.

			            The resulting tree will look like this:

			            \begin{center}
				            \begin{tikzpicture}[
						            >=latex,
						            thick,
						            node/.style={
								            draw,
								            rounded corners=.25em,
								            text depth=0.2em
							            },
						            leaf/.style={
								            draw,
								            rounded corners=.7em,
								            text depth=0.2em
							            },
						            branch/.style={
								            fill=white,
								            font=\ttfamily\scriptsize,
								            rounded corners=.7em,
								            text depth=0.2em
							            }
					            ]
					            \useasboundingbox (-6,0.5) rectangle (6,-6);

					            \node[node, anchor=center] at (0,0) (parti) {Participation?};

					            \node[draw=none,anchor=north] at (-4,-1.5) (highmedium) {
						            \resizebox{6.5cm}{!}{%
							            \begin{tabular}{|c|c|c|c|c|}
								            \hline
								            \textbf{Age} & \textbf{Major} & \textbf{Participation} & \textbf{Passed} \\ \hline
								            23           & CS             & High                   & Yes             \\ \hline
								            26           & DS             & High                   & Yes             \\ \hline
								            24           & DS             & Medium                 & Yes             \\ \hline
								            26           & DS             & Medium                 & No              \\ \hline
							            \end{tabular}
						            }
					            };

					            \draw[rounded corners=5pt] (parti.south) -- (0,-1) -- node[above, draw=none] {high, medium} (-4,-1) -- (highmedium.north);


					            \node[draw=none,anchor=north] at (4,-1.5)  (low) {
						            \resizebox{6.5cm}{!}{%
							            \begin{tabular}{|c|c|c|c|c|}
								            \hline
								            \textbf{Age} & \textbf{Major} & \textbf{Participation} & \textbf{Passed} \\ \hline
								            23           & DS             & Low                    & No              \\ \hline
								            26           & DS             & Low                    & No              \\ \hline
							            \end{tabular}
						            }
					            };

					            \draw[rounded corners=5pt] (parti.south) -- (0,-1) -- node[above, draw=none] {low} (4,-1) -- (low.north);

				            \end{tikzpicture}
			            \end{center}

			            Since the table for the branch \textit{Low} only contains tuples with the class label \textit{No}, we can already create a leaf node for this branch:

			            \begin{center}
				            \begin{tikzpicture}[
						            >=latex,
						            thick,
						            node/.style={
								            draw,
								            rounded corners=.25em,
								            text depth=0.2em
							            },
						            leaf/.style={
								            draw,
								            rounded corners=.7em,
								            text depth=0.2em
							            },
						            branch/.style={
								            fill=white,
								            font=\ttfamily\scriptsize,
								            rounded corners=.7em,
								            text depth=0.2em
							            }
					            ]
					            \useasboundingbox (-6,0.5) rectangle (6,-6);

					            \node[node, anchor=center] at (0,0) (parti) {Participation?};

					            \node[draw=none,anchor=north] at (-4,-1.5) (highmedium) {
						            \resizebox{6.5cm}{!}{%
							            \begin{tabular}{|c|c|c|c|c|}
								            \hline
								            \textbf{Age} & \textbf{Major} & \textbf{Participation} & \textbf{Passed} \\ \hline
								            23           & CS             & High                   & Yes             \\ \hline
								            26           & DS             & High                   & Yes             \\ \hline
								            24           & DS             & Medium                 & Yes             \\ \hline
								            26           & DS             & Medium                 & No              \\ \hline
							            \end{tabular}
						            }
					            };

					            \draw[rounded corners=5pt] (parti.south) -- (0,-1) -- node[above, draw=none] {high, medium} (-4,-1) -- (highmedium.north);


					            \node[leaf,anchor=north] at (4,-1.5)  (low) {No};

					            \draw[rounded corners=5pt] (parti.south) -- (0,-1) -- node[above, draw=none] {low} (4,-1) -- (low.north);

				            \end{tikzpicture}
			            \end{center}

		      \end{enumerate}
	\end{enumerate}
\end{solution}

\textbf{Task 2.b: Splitting attribute candidates}

In the resulting tree from Task 2.a, one of the branches is already a leaf node.

Which of the \textbf{attributes} \textit{Age}, \textit{Major} and \textit{Participation} have to \textbf{be checked} for their Gini index in the next step necessary to further split the remaining branch?

\begin{solution}
	\begin{itemize}
		\item \textbf{Attribute \textit{Age}:}

		      The attribute \textit{Age} should be checked for its Gini Index in the branch \textit{High, Medium} since it still contains different values.

		\item \textbf{Attribute \textit{Major}:}

		      The attribute \textit{Major} should be checked for its Gini Index in the branch \textit{High, Medium} since it still contains different values.

		\item \textbf{Attribute \textit{Participation}:}

		      Contrary to the procedure for Information Gain - where the attribute \textit{Participation} would not be checked again - we also have to check the attribute \textit{Participation} for its Gini Index in the branch \textit{High, Medium} since it still contains different values.

		      This is due to the fact that the Gini Index does not support multi-way splits. Therefore a categorical attribute with more than two values can be splitting attribute multiple times in the same branch.


	\end{itemize}
\end{solution}

\subsection*{Task 3: Gain Ratio}

The \textbf{Gain Ratio} is a solution to a problem of the \textbf{Information Gain}.

Come up with an example \textbf{dataset} showing the problem of the Information Gain and explain how the Gain Ratio solves this problem.

\begin{solution}
	The problem of the Information Gain is that it tends to favor attributes with a large number of values.

	If we for example take a look at the following dataset $D$:

	\begin{center}
		\begin{tabular}{|c|c|c|c|c|c|}
			\hline
			\textbf{Major} & \textbf{Participation} & \textbf{Passed} \\ \hline
			CS             & High                   & Yes             \\ \hline
			DS             & Low                    & No              \\ \hline
			IIS            & High                   & Yes             \\ \hline
			AI             & Medium                 & Yes             \\ \hline
			ICT            & Medium                 & No              \\ \hline
			CME            & Low                    & No              \\ \hline
		\end{tabular}
	\end{center}

	We can see that the attribute \textit{Major} only contains unique values.

	Therefore, the Information Gain for the attribute \textit{Major} would be $1$:

	\begin{alignat*}{2}
		\text{Info}(D)                    & = -\sum_{i=1}^{m}p_i \log_2(p_i)                                                                                                                                                                                     \\
		                                  & = -p\OfSpecificValue{Passed}{=}{Yes} \log_2(p\OfSpecificValue{Passed}{=}{Yes})- p\OfSpecificValue{Passed}{=}{No} \log_2(p\OfSpecificValue{Passed}{=}{No})                                                            \\
		                                  & = -\frac{3}{6} \log_2\left(\frac{3}{6}\right) - \frac{3}{6} \log_2\left(\frac{3}{6}\right)                                                                                                                           \\
		                                  & = 1                                                                                                                                                                                                                  \\
		\text{Info}\OfAttribute{Major}(D) & = \sum_{j=1}^v \frac{|D\OfAttribute{Major$,j$}|}{|D\OfAttribute{Major}|} \text{Info}(D_{A\OfAttribute{Major$,j$}})                                                                                                   \\
		                                  & = \frac{1}{6} \left(-\frac{1}{1} \log_2\left(\frac{1}{1}\right)\right) + \frac{1}{6} \left(-\frac{1}{1} \log_2\left(\frac{1}{1}\right)\right) + \frac{1}{6} \left(-\frac{1}{1} \log_2\left(\frac{1}{1}\right)\right) \\
		                                  & + \frac{1}{6} \left(-\frac{1}{1} \log_2\left(\frac{1}{1}\right)\right) + \frac{1}{6} \left(-\frac{1}{1} \log_2\left(\frac{1}{1}\right)\right) + \frac{1}{6} \left(-\frac{1}{1} \log_2\left(\frac{1}{1}\right)\right) \\
		                                  & = 0                                                                                                                                                                                                                  \\
		\text{Gain}\OfAttribute{Major}    & = \text{Info}(D) - \text{Info}\OfAttribute{Major}(D)                                                                                                                                                                 \\
		                                  & = 1 - 0                                                                                                                                                                                                              \\
		                                  & = 1                                                                                                                                                                                                                  \\
	\end{alignat*}

	On the other hand the attribute \textit{Participation} still (see Task 1) has an Information Gain of $0,6667$:

	\begin{alignat*}{3}
		\text{Info}\OfAttribute{Parti.}(D) & = \sum_{j=1}^v \frac{|D\OfAttribute{Parti.$,j$}|}{|D\OfAttribute{Parti.}|} \text{Info}(D_{A\OfAttribute{Parti.$,j$}})                                                                                                                             \\
		                                   & = \frac{|D\OfSpecificValue{Parti.}{=}{High}|}{|D\OfAttribute{Parti.}|} \text{Info}(D\OfSpecificValue{Parti.}{=}{High}) + \frac{|D\OfSpecificValue{Parti.}{=}{Medium}|}{|D\OfAttribute{Parti.}|} \text{Info}(D\OfSpecificValue{Parti.}{=}{Medium}) \\
		                                   & + \frac{|D\OfSpecificValue{Parti.}{=}{Low}|}{|D\OfAttribute{Parti.}|} \text{Info}(D\OfSpecificValue{Parti.}{=}{Low})                                                                                                                              \\
		                                   & = \frac{2}{6} \left(-\frac{2}{2} \log_2\left(\frac{2}{2}\right) - \frac{0}{2} \log_2\left(\frac{0}{2}\right)\right) + \frac{2}{6} \left(-\frac{1}{2} \log_2\left(\frac{1}{2}\right) - \frac{1}{2} \log_2\left(\frac{1}{2}\right)\right)           \\
		                                   & + \frac{2}{6} \left(-\frac{0}{2} \log_2\left(\frac{0}{2}\right) - \frac{2}{2} \log_2\left(\frac{2}{2}\right)\right)                                                                                                                               \\
		                                   & = \frac{2}{6} \cdot 0 + \frac{2}{6} \cdot 1 + \frac{2}{6} \cdot 0                                                                                                                                                                                 \\
		                                   & =  0,3333                                                                                                                                                                                                                                         \\
		\text{Gain}\OfAttribute{Parti.}    & = \text{Info}(D) - \text{Info}\OfAttribute{Parti.}(D)                                                                                                                                                                                             \\
		                                   & = 1 - 0,3333                                                                                                                                                                                                                                      \\
		                                   & = 0,6667                                                                                                                                                                                                                                          \\
	\end{alignat*}

	With the Information Gain, we would choose the attribute \textit{Major} as the splitting attribute since it has a higher Information Gain than the attribute \textit{Participation}.

	However, since the attribute \textit{Major} only contains unique values, it is not a good splitting attribute, since we on the one hand up end up with a big multi-way split and on the other hand risk to overfit our decision tree on the training data.

	The Gain Ratio solves this problem by normalizing the Information Gain with the \textbf{Split Information}:

	\begin{alignat*}{2}
		\text{SplitInfo}\OfAttribute{A}(D) & = -\sum_{j=1}^{v} \frac{|D\OfAttribute{A$,j$}|}{|D|} \log_2\left(\frac{|D\OfAttribute{A$,j$}|}{|D|}\right) \\
		\text{GainRatio}\OfAttribute{A}    & = \frac{\text{Gain}\OfAttribute{A}}{\text{SplitInfo}\OfAttribute{A}(D)}
	\end{alignat*}

	If we apply the Gain Ratio to the Gain of the attributes \textit{Major} and \textit{Participation}, we get:

	\begin{alignat*}{2}
		\text{SplitInfo}\OfAttribute{Major}(D) & = -\frac{1}{6} \log_2\left(\frac{1}{6}\right) - \frac{1}{6} \log_2\left(\frac{1}{6}\right) - \frac{1}{6} \log_2\left(\frac{1}{6}\right) \\
		                                       & - \frac{1}{6} \log_2\left(\frac{1}{6}\right) - \frac{1}{6} \log_2\left(\frac{1}{6}\right) - \frac{1}{6} \log_2\left(\frac{1}{6}\right)  \\
		                                       & = 2,5850                                                                                                                                \\
		\text{GainRatio}\OfAttribute{Major}    & = \frac{1}{2,5850}                                                                                                                      \\
		                                       & = 0,3868                                                                                                                                \\
	\end{alignat*}
	\begin{alignat*}{2}
		\text{SplitInfo}\OfAttribute{Parti.}(D) & = -\frac{2}{6} \log_2\left(\frac{2}{6}\right) - \frac{2}{6} \log_2\left(\frac{2}{6}\right) - \frac{2}{6} \log_2\left(\frac{2}{6}\right) \\
		                                        & = 1,5849                                                                                                                                \\
		\text{GainRatio}\OfAttribute{Parti.}    & = \frac{0,6667}{1,5849}                                                                                                                 \\
		                                        & = 0,4207                                                                                                                                \\
	\end{alignat*}

	With the Gain Ratio, we would now choose the attribute \textit{Participation} as the splitting attribute since it has a higher Gain Ratio than the attribute \textit{Major}.

	We therefore avoid the problem of the Information Gain to favor attributes with a large number of values.

	\textit{\textbf{Note:} The Gain Ratio has its own problems, as it becomes unstable if SplitInfo is close to zero. In this case, the Gain Ratio can become very high. Therefore, the Gain Ratio should also be used with caution.}

\end{solution}

\newpage

\section*{Exercise 2: Naïve Bayes}

Given is a dataset $D$:

\begin{center}
	\begin{tabular}{|c|c|c|c|}
		\hline
		% Basic Idea: Submission Topic & Prior Knowledge & Hours Invested & Passed
		\textbf{Topic}    & \textbf{Knowledge} & \textbf{Hours} & \textbf{Passed} \\ \hline
		Classification    & High               & 1,0            & No              \\ \hline
		Clustering        & Low                & 4,0            & No              \\ \hline
		Frequent Patterns & High               & 5,0            & Yes             \\ \hline
		Clustering        & Medium             & 5,0            & Yes             \\ \hline
		Frequent Patterns & High               & 2,0            & No              \\ \hline
		Frequent Patterns & Medium             & 3,0            & Yes             \\ \hline
		Classification    & Low                & 6,0            & Yes             \\ \hline
		Clustering        & Low                & 5,0            & Yes             \\ \hline
		Clustering        & High               & 3,0            & Yes             \\ \hline
		Classification    & Medium             & 4,0            & Yes             \\ \hline
	\end{tabular}
\end{center}

It can be assumed that \textit{Topic}, \textit{Knowledge} and \textit{Hours} are conditionally independent of each other.

The attributes \textit{Topic} and \textit{Knowledge} are categorical attributes. \newline
The attribute \textit{Hours} is a continuous attribute. It can be assumed that the values of this attribute are distributed according to a Gaussian distribution.

\subsection*{Task 1: Classification}

Use the dataset $D$ and the Naïve Bayes algorithm to classify the following tuples:

\begin{center}
	\begin{tabular}{|c|c|c|c|}
		\hline
		\textbf{Topic}    & \textbf{Knowledge} & \textbf{Hours} & \textbf{Passed} \\ \hline
		Clustering        & Medium             & 4,0            & ?               \\ \hline
		Classification    & High               & 3,0            & ?               \\ \hline
		Frequent Patterns & Low                & 6,8            & ?               \\ \hline
	\end{tabular}
\end{center}

Write down \textbf{all} intermediate steps.

\begin{solution}
	\begin{enumerate}
		\item \textbf{Calculate the prior probabilities:}

		      \begin{alignat*}{2}
			      \PriorProbability{Passed}{Yes} & = \frac{7}{10} = 0,7 \\
			      \PriorProbability{Passed}{No}  & = \frac{3}{10} = 0,3 \\
		      \end{alignat*}

		\item \textbf{Calculate the likelihoods:}

		      \begin{enumerate}
			      \item \textbf{Attribute \textit{Topic}:}
			            \begin{alignat*}{2}
				            \Likelihood{Topic}{Clustering}{Passed}{Yes}        & = \frac{3}{7} = 0,4286 \\
				            \Likelihood{Topic}{Clustering}{Passed}{No}         & = \frac{1}{3} = 0,3333 \\
				                                                               &                        \\
				            \Likelihood{Topic}{Classification}{Passed}{Yes}    & = \frac{2}{7} = 0,2857 \\
				            \Likelihood{Topic}{Classification}{Passed}{No}     & = \frac{1}{3} = 0,3333 \\
				                                                               &                        \\
				            \Likelihood{Topic}{Frequent Patterns}{Passed}{Yes} & = \frac{2}{7} = 0,2857 \\
				            \Likelihood{Topic}{Frequent Patterns}{Passed}{No}  & = \frac{1}{3} = 0,3333 \\
			            \end{alignat*}

			      \item \textbf{Attribute \textit{Knowledge}:}
			            \begin{alignat*}{2}
				            \Likelihood{Knowledge}{High}{Passed}{Yes}   & = \frac{2}{7} = 0,2857 \\
				            \Likelihood{Knowledge}{High}{Passed}{No}    & = \frac{2}{3} = 0,6667 \\
				                                                        &                        \\
				            \Likelihood{Knowledge}{Medium}{Passed}{Yes} & = \frac{3}{7} = 0,4286 \\
				            \Likelihood{Knowledge}{Medium}{Passed}{No}  & = \frac{0}{3} = 0      \\
				                                                        &                        \\
				            \Likelihood{Knowledge}{Low}{Passed}{Yes}    & = \frac{2}{7} = 0,2857 \\
				            \Likelihood{Knowledge}{Low}{Passed}{No}     & = \frac{1}{3} = 0,3333 \\
			            \end{alignat*}

			      \item \textbf{Attribute \textit{Hours}:}

			            Since the attribute \textit{Hours} is continuous and follows a Gaussian distribution, we have to calculate the mean $\mu$ and the standard deviation $\sigma$ for each class label:

			            \begin{alignat*}{2}
				            \mu_{\texttt{Passed}=\text{"Yes"}}    & = \frac{5 + 5 + 3 + 6 + 5 + 3 + 4}{7} = \frac{31}{7} = 4,4286                                                                 \\
				            \sigma_{\texttt{Passed}=\text{"Yes"}} & = \sqrt{\frac{(5-\frac{31}{7})^2 \cdot 3 + (3-\frac{31}{7})^2 \cdot 2 + (6-\frac{31}{7})^2 + (4-\frac{31}{7})^2}{7}} = 1,0498 \\
				                                                  &                                                                                                                               \\
				            \mu_{\texttt{Passed}=\text{"No"}}     & = \frac{1 + 4 + 2}{3} = \frac{7}{3} = 2,3333                                                                                  \\
				            \sigma_{\texttt{Passed}=\text{"No"}}  & = \sqrt{\frac{(1-\frac{7}{3})^2 + (4-\frac{7}{3})^2 + (2-\frac{7}{3})^2}{3}} = 1,2472                                         \\
			            \end{alignat*}

			            We can now calculate the likelihoods for the attribute \textit{Hours}:

			            \begin{alignat*}{2}
				            \Likelihood{Hours}{4}{Passed}{Yes}   & = \frac{1}{\sqrt{2\pi} \cdot 1,0498} \cdot e^{-\frac{(4-4,4286)^2}{2 \cdot 1,0498^2}} = 0,3496   \\
				            \Likelihood{Hours}{4}{Passed}{No}    & = \frac{1}{\sqrt{2\pi} \cdot 1,2472} \cdot e^{-\frac{(4-2,3333)^2}{2 \cdot 1,2472^2}} = 0,1309   \\
				                                                 &                                                                                                  \\
				            \Likelihood{Hours}{3}{Passed}{Yes}   & = \frac{1}{\sqrt{2\pi} \cdot 1,0498} \cdot e^{-\frac{(3-4,4286)^2}{2 \cdot 1,0498^2}} = 0,1505   \\
				            \Likelihood{Hours}{3}{Passed}{No}    & = \frac{1}{\sqrt{2\pi} \cdot 1,2472} \cdot e^{-\frac{(3-2,3333)^2}{2 \cdot 1,2472^2}} = 0,2773   \\
				                                                 &                                                                                                  \\
				            \Likelihood{Hours}{6,8}{Passed}{Yes} & = \frac{1}{\sqrt{2\pi} \cdot 1,0498} \cdot e^{-\frac{(6,8-4,4286)^2}{2 \cdot 1,0498^2}} = 0,0296 \\
				            \Likelihood{Hours}{6,8}{Passed}{No}  & = \frac{1}{\sqrt{2\pi} \cdot 1,2472} \cdot e^{-\frac{(6,8-2,3333)^2}{2 \cdot 1,2472^2}} = 0,0005 \\
			            \end{alignat*}
		      \end{enumerate}

		\item \textbf{Calculate the likelihood of each tuple:}

		      \begin{enumerate}
			      \item \textbf{Tuple $T_1$ with \textit{Clustering, Medium, 4}:}
			            \begin{alignat*}{2}
				            \LikelihoodTuple{T_1}{Passed}{Yes} & =  \Likelihood{Topic}{Clustering}{Passed}{Yes}    \\
				                                               & \cdot \Likelihood{Knowledge}{Medium}{Passed}{Yes} \\
				                                               & \cdot \Likelihood{Hours}{4}{Passed}{Yes}          \\
				                                               & = 0,4286 \cdot 0,4286 \cdot 0,3496                \\
				                                               & = 0,0642                                          \\
				                                               &                                                   \\
				            \LikelihoodTuple{T_1}{Passed}{No}  & =  \Likelihood{Topic}{Clustering}{Passed}{No}     \\
				                                               & \cdot \Likelihood{Knowledge}{Medium}{Passed}{No}  \\
				                                               & \cdot \Likelihood{Hours}{4}{Passed}{No}           \\
				                                               & =  0,3333 \cdot 0 \cdot 0,1309                    \\
				                                               & = 0                                               \\
			            \end{alignat*}
			      \item \textbf{Tuple $T_2$ with \textit{Classification, High, 3}:}
			            \begin{alignat*}{2}
				            \LikelihoodTuple{T_2}{Passed}{Yes} & =  \Likelihood{Topic}{Classification}{Passed}{Yes} \\
				                                               & \cdot \Likelihood{Knowledge}{High}{Passed}{Yes}    \\
				                                               & \cdot \Likelihood{Hours}{3}{Passed}{Yes}           \\
				                                               & =  0,2857 \cdot 0,2857 \cdot 0,1505                \\
				                                               & = 0,0123                                           \\
				                                               &                                                    \\
				            \LikelihoodTuple{T_2}{Passed}{No}  & =  \Likelihood{Topic}{Classification}{Passed}{No}  \\
				                                               & \cdot \Likelihood{Knowledge}{High}{Passed}{No}     \\
				                                               & \cdot \Likelihood{Hours}{3}{Passed}{No}            \\
				                                               & =  0,3333 \cdot 0,6667 \cdot 0,2773                \\
				                                               & = 0,0616                                           \\
			            \end{alignat*}

			      \item \textbf{Tuple $T_3$ with \textit{Frequent Patterns, Low, 7}:}
			            \begin{alignat*}{2}
				            \LikelihoodTuple{T_3}{Passed}{Yes} & =  \Likelihood{Topic}{Frequent Patterns}{Passed}{Yes} \\
				                                               & \cdot \Likelihood{Knowledge}{Low}{Passed}{Yes}        \\
				                                               & \cdot \Likelihood{Hours}{7}{Passed}{Yes}              \\
				                                               & =  0,2857 \cdot 0,2857 \cdot 0,0296                   \\
				                                               & = 0,0024                                              \\
				                                               &                                                       \\
				            \LikelihoodTuple{T_3}{Passed}{No}  & =  \Likelihood{Topic}{Frequent Patterns}{Passed}{No}  \\
				                                               & \cdot \Likelihood{Knowledge}{Low}{Passed}{No}         \\
				                                               & \cdot \Likelihood{Hours}{7}{Passed}{No}               \\
				                                               & =  0,3333 \cdot 0,3333 \cdot 0,0005                   \\
				                                               & = 0,0000                                              \\
			            \end{alignat*}
		      \end{enumerate}

		\item \textbf{Determine the highest posteriori probability for each tuple:}

		      The posteriori probability according to Bayes' theorem is actually calculated as follows:

		      \begin{alignat*}{2}
			      P(C_i|X) = \frac{P(X|C_i)P(C_i)}{P(X)}.
		      \end{alignat*}

		      Where $C_i$ stands for the class to be predicted and $X$ is representing a specific tuple (resp. the attribute combination that is part of that tuple).

		      However, since $P(X)$ is the same for all classes, it is sufficient to calculate only the numerators to determine the highest $P(C_i|X)$.

		      Being able to determine the highest $P(C_i|X)$ (even without knowing its exact value) is sufficient to classify the tuple.

		      \begin{enumerate}
			      \item \textbf{Tuple $T_1$ with \textit{Clustering, Medium, 4}:}

			            Calculate the numerator of $\PosterioriProbability{T_1}{Passed}{Yes}$ and $\PosterioriProbability{T_1}{Passed}{No}$:

			            \begin{alignat*}{2}
				            \BayesNumerator{T_1}{Passed}{Yes} & = 0,0642 \cdot 0,7 = 0,0449 \\
				            \BayesNumerator{T_1}{Passed}{No}  & = 0 \cdot 0,3 = 0           \\
			            \end{alignat*}

			            Since $0,0449 > 0$ and we classify the tuple $T_1$ as $\ResultClass{Passed}{Yes}$.

			            \vspace*{3em}

			            \begin{mdframed}[linecolor=solutioncolor]
				            \color{solutioncolor}
				            \begin{em}
					            \textbf{Calculation of the posteriori probability:}

					            Even if the calculation of the full posteriori probability is not necessary, it is still possible to calculate it.

					            We first need to calculate the denominator of the posteriori probability $P(X)$:

					            \begin{alignat*}{2}
						            P(T_1) & = \BayesNumerator{T_1}{Passed}{Yes}       \\
						                   & + \BayesNumerator{T_1}{Passed}{No}        \\
						                   & = 0,0642 \cdot 0,7 + 0 \cdot 0,3 = 0,0449 \\
						                   & = 0,0449 + 0 = 0,0449
					            \end{alignat*}


					            Which can then be used to calculate the posteriori probabilities:

					            \begin{alignat*}{2}
						            \PosterioriProbability{T_1}{Passed}{Yes} & = \frac{\BayesNumerator{T_1}{Passed}{Yes}}{P(T_1)} \\
						                                                     & = \frac{0,0449}{0,0449}                            \\
						                                                     & = 1                                                \\
						                                                     &                                                    \\
						            \PosterioriProbability{T_1}{Passed}{No}  & = \frac{\BayesNumerator{T_1}{Passed}{No}}{P(T_1)}  \\
						                                                     & = \frac{0}{0,0449}                                 \\
						                                                     & = 0                                                \\
					            \end{alignat*}

					            As this calculation is not necessary for the classification, we will not calculate the posteriori probabilities for the other tuples.
				            \end{em}
			            \end{mdframed}

			            \newpage

			      \item \textbf{Tuple $T_2$ with \textit{Classification, High, 3}:}

			            Calculate the numerator of $\PosterioriProbability{T_2}{Passed}{Yes}$ and $\PosterioriProbability{T_2}{Passed}{No}$:

			            \begin{alignat*}{2}
				            \BayesNumerator{T_2}{Passed}{Yes} & = 0,0123 \cdot 0,7 = 0,0086 \\
				            \BayesNumerator{T_2}{Passed}{No}  & = 0,0616 \cdot 0,3 = 0,0185 \\
			            \end{alignat*}

			            Since $0,0086 < 0,0185$ we classify the tuple $T_2$ as $\ResultClass{Passed}{No}$.

			      \item \textbf{Tuple $T_3$ with \textit{Frequent Patterns, Low, 7}:}

			            Calculate the numerator of $\PosterioriProbability{T_3}{Passed}{Yes}$ and $\PosterioriProbability{T_3}{Passed}{No}$:

			            \begin{alignat*}{2}
				            \BayesNumerator{T_3}{Passed}{Yes} & = 0,0024 \cdot 0,7 = 0,0017 \\
				            \BayesNumerator{T_3}{Passed}{No}  & = 0,0000 \cdot 0,3 = 0,0000 \\
			            \end{alignat*}

			            Since $0,0017 > 0,0000$ we classify the tuple $T_3$ as $\ResultClass{Passed}{Yes}$.
		      \end{enumerate}



	\end{enumerate}
\end{solution}

\newpage

\subsection*{Task 2: Model Evaluation}

The classifier was also trained on a version of dataset $D$ with more tuples.

To test the quality of the resulting model, some test values were classified.

The dataset $T$ contains both the true and the predicted "Passed"-Status for each test tuple:

\begin{center}
	\begin{tabular}{|c|c|c|c|c|}
		\hline
		\textbf{Topic}    & \textbf{Knowledge} & \textbf{Hours} & \textbf{\begin{tabular}[c]{@{}c@{}}Passed\\ (True)\end{tabular}} & \textbf{\begin{tabular}[c]{@{}c@{}}Passed\\ (Pred)\end{tabular}} \\ \hline
		Classification    & Medium             & 7,5            & Yes                                                              & Yes                                                              \\ \hline
		Frequent Patterns & Low                & 1,8            & No                                                               & No                                                               \\ \hline
		Frequent Patterns & High               & 3,7            & No                                                               & Yes                                                              \\ \hline
		Frequent Patterns & Low                & 0,2            & No                                                               & No                                                               \\ \hline
		Frequent Patterns & High               & 1,4            & Yes                                                              & No                                                               \\ \hline
		Frequent Patterns & High               & 9,9            & Yes                                                              & Yes                                                              \\ \hline
		Frequent Patterns & Medium             & 7,3            & Yes                                                              & Yes                                                              \\ \hline
		Frequent Patterns & Low                & 4,3            & No                                                               & Yes                                                              \\ \hline
		Classification    & Medium             & 5,5            & Yes                                                              & Yes                                                              \\ \hline
		Clustering        & Low                & 0,1            & No                                                               & No                                                               \\ \hline
	\end{tabular}
\end{center}

Use the dataset $T$ to calculate the \textbf{sensitivity}, \textbf{specificity}, \textbf{accuracy}, \textbf{precision}, \textbf{recall}, and \textbf{F1-score} of the model.

Also state the \textbf{best possible} value for each metric.

\begin{solution}
	We need to calculate the True Positives (TP), True Negatives (TN), False Positives (FP), and False Negatives (FN) for the model evaluation.

	This is often done by creating a confusion matrix:

	\begin{center}
		\begin{tabular}{c|c|>{\centering}p{1.5cm}|>{\centering}p{1.5cm}|c|}

			\multicolumn{2}{c|}{\multirow{2}{*}{}} & \multicolumn{2}{c|}{Predicted} &                              \\\cline{3-4}
			\multicolumn{2}{c|}{}                  & $Yes$                          & $No$   & Total               \\\hline
			\multirow{2}{*}{True}                  & $Yes$                          & 4 (TP) & 1 (FN)      & 5 (P) \\\cline{2-4}
			                                       & $No$                           & 2 (FP) & 3 (TN)      & 5 (N) \\\hline
			\multicolumn{2}{r|}{Total}             & 6 (P')                         & 4 (N') & 10 (P + N)
		\end{tabular}
	\end{center}

	This confusion matrix can be used to calculate the metrics:

	\begin{itemize}
		\item \textbf{Sensitivity:}
		      \begin{alignat*}{2}
			      \text{Sensitivity} & = \frac{TP}{P} = \frac{4}{5} = 0,8
		      \end{alignat*}

		      Best possible value: 1
		\item \textbf{Specificity:}
		      \begin{alignat*}{2}
			      \text{Specificity} & = \frac{TN}{N} = \frac{3}{5} = 0,6
		      \end{alignat*}

		      Best possible value: 1
		\item \textbf{Accuracy:}
		      \begin{alignat*}{2}
			      \text{Accuracy} & = \frac{TP + TN}{P + N} = \frac{4 + 3}{5 + 5} = \frac{7}{10} = 0,7
		      \end{alignat*}

		      Best possible value: 1
		\item \textbf{Precision:}
		      \begin{alignat*}{2}
			      \text{Precision} & = \frac{TP}{TP + FP} = \frac{4}{4 + 2} = \frac{4}{6} = 0,6667
		      \end{alignat*}

		      Best possible value: 1
		\item \textbf{Recall:}
		      \begin{alignat*}{2}
			      \text{Recall} & = \frac{TP}{TP + FN} = \frac{TP}{P} = \text{Sensitivity} = 0,8
		      \end{alignat*}

		      Best possible value: 1
		\item \textbf{F1-Score:}
		      \begin{alignat*}{2}
			      \text{F1-Score} & = 2 \cdot \frac{\text{Precision} \cdot \text{Recall}}{\text{Precision} + \text{Recall}} = 2 \cdot \frac{0,6667 \cdot 0,8}{0,6667 + 0,8} = 0,7273
		      \end{alignat*}

		      Best possible value: 1
	\end{itemize}

	Of course, the question regarding the best possible value is a trick question. The best possible value for each metric is 1 (or 100\%), as this would mean 100\% of the tuples were classified correctly.


\end{solution}

\newpage

\section*{Exercise 3: TODO}

TODO


\end{document}
