\def\solutionsflag{}

\documentclass[
english,
smallborders
]{i6prcsht}
\usepackage{i6common}
\usepackage{i6lecture}

\usepackage{todonotes}
\usepackage[utf8]{inputenc}
\usepackage{textcomp}
\usepackage{pdfpages}
\usepackage{csquotes}
\usepackage{awesomebox}
\usepackage{makecell}
\usepackage{graphicx}
\usepackage{multicol}
\usepackage{amsmath}

\hyphenation{Stud-On}

\newcommand{\OfSpecificValue}[3]{_{\text{\tiny #1#2#3}}}
\newcommand{\OfAttribute}[1]{_{\text{\tiny #1}}}

\begin{document}

\title{Exercise Sheet 4: \\ Classification}
\maketitle
\vspace*{-2cm}

\section*{About this Exercise Sheet}

This exercise sheet focuses on the content of lecture \textit{7. Classification}.

It includes TODO.

The exercise sheet is designed for a three-week period, during which the tasks can be completed flexibly.

The sample solution will be published after the three weeks have elapsed.

\section*{Preparation}

Before participating in the exercise, you must prepare the following:

\begin{enumerate}
	\item \textbf{Install Python and pip on your computer}

	      \begin{itemize}
		      \item Detailed instructions can be found in \texttt{1-Introduction-Python-Pandas.pdf}.
	      \end{itemize}

	\item \textbf{Download provided additional files}

	      \begin{itemize}
		      \item Download \texttt{Additional-Files-Student.zip} from StudOn
		      \item Extract it to a folder of your choice.
	      \end{itemize}

	\item \textbf{Install required Python packages}

	      \begin{itemize}
		      \item Open a terminal and navigate to the folder where you extracted the files.
		      \item Run the command \texttt{pip install -r requirements.txt} within the extracted additional files folder to install the required Python packages.
	      \end{itemize}


\end{enumerate}

\section*{Exercise 1: Decision Trees}

\begin{comment}
% Table for Exercise 1
Age; Age1; Age2; Major; Participation; Passed?
23; <=23,5; <=25; CS; High; Yes
23; <=23,5; <=25; DS; Low; No
26; >23,5; >25; DS; High; Yes
24; >23,5; <=25; CS; Medium; Yes
26; >23,5; >25; DS; Medium; No
26; >23,5; >25; DS; Low; No

% Good site to check: https://planetcalc.com/8421/
\end{comment}

Given is a dataset $D$:

\begin{center}
	\begin{tabular}{|c|c|c|c|c|c|}
		\hline
		\textbf{Age} & \textbf{Major} & \textbf{Participation} & \textbf{Passed} \\ \hline
		23           & CS             & High                   & Yes             \\ \hline
		23           & DS             & Low                    & No              \\ \hline
		26           & DS             & High                   & Yes             \\ \hline
		24           & DS             & Medium                 & Yes             \\ \hline
		26           & DS             & Medium                 & No              \\ \hline
		26           & DS             & Low                    & No              \\ \hline
	\end{tabular}
\end{center}

$D$ is containing a continuous attribute (\textit{Age}) and two categorical attributes (\textit{Major} and \textit{Participation}) which can be used to predict the target attribute \textit{Passed}.

\subsection*{Task 1: Information Gain}

Use the algorithm for \textbf{Decision Tree Induction} known from the lecture to build a decision tree for this dataset. The decision tree should be built using \textbf{Information Gain} as the attribute selection method.

Write down \textbf{all} intermediate steps.

\begin{solution}
	\begin{enumerate}
		\item \textbf{Create the root node:}

		      To create the root node, we need to calculate the Information Gain for each attribute and select the one with the highest Information Gain.

		      \begin{enumerate}
			      \item \textbf{Calculate the Entropy of the target attribute \textit{Passed}:}
			            \begin{alignat*}{2}
				            \text{Info}(D) & = -\sum_{i=1}^{m}p_i \log_2(p_i)                                                                                                                          \\
				                           & = -p\OfSpecificValue{Passed}{=}{Yes} \log_2(p\OfSpecificValue{Passed}{=}{Yes})- p\OfSpecificValue{Passed}{=}{No} \log_2(p\OfSpecificValue{Passed}{=}{No}) \\
				                           & = -\frac{3}{6} \log_2\left(\frac{3}{6}\right) - \frac{3}{6} \log_2\left(\frac{3}{6}\right)                                                                \\
				                           & = 1                                                                                                                                                       \\
			            \end{alignat*}

			      \item \textbf{Calculate the Information Gain for all attributes:}

			            \begin{enumerate}
				            \item \textbf{Attribute \textit{Age}:}

				                  \textit{Age} is a continuous attribute. To calculate the Information Gain, we need to find the best split point.

				                  \begin{enumerate}
					                  \item \textbf{Split point $23,5$:}

					                        \begin{alignat*}{2}
						                        \text{Info}\OfAttribute{Age}(D) & = \sum_{j=1}^v \frac{|D\OfAttribute{Age$,j$}|}{|D\OfAttribute{Age}|} \text{Info}(D_{A\OfAttribute{Age$,j$}})                                                                                                                            \\
						                                                        & = \frac{|D\OfSpecificValue{Age}{$\leq$}{2,5}|}{|D\OfAttribute{Age}|} \text{Info}(D\OfSpecificValue{Age}{$\leq$}{2,5}) + \frac{|D\OfSpecificValue{Age}{$>$}{2,5}|}{|D\OfAttribute{Age}|} \text{Info}(D\OfSpecificValue{Age}{$>$}{2,5})   \\
						                                                        & = \frac{2}{6} \left(-\frac{1}{2} \log_2\left(\frac{1}{2}\right) - \frac{1}{2} \log_2\left(\frac{1}{2}\right)\right) + \frac{4}{6} \left(-\frac{2}{4} \log_2\left(\frac{2}{4}\right) - \frac{2}{4} \log_2\left(\frac{2}{4}\right)\right) \\
						                                                        & = \frac{2}{6} \cdot 1 + \frac{4}{6} \cdot 1                                                                                                                                                                                             \\
						                                                        & = 1                                                                                                                                                                                                                                     \\
						                        \text{Gain}\OfAttribute{Age}    & = \text{Info}(D) - \text{Info}\OfAttribute{Age}(D)                                                                                                                                                                                      \\
						                                                        & = 1 - 1                                                                                                                                                                                                                                 \\
						                                                        & = 0                                                                                                                                                                                                                                     \\
					                        \end{alignat*}

					                  \item \textbf{Split point $25,5$:}

					                        \begin{alignat*}{2}
						                        \text{Info}\OfAttribute{Age}(D) & = \sum_{j=1}^v \frac{|D\OfAttribute{Age$,j$}|}{|D\OfAttribute{Age}|} \text{Info}(D_{A\OfAttribute{Age$,j$}})                                                                                                                            \\
						                                                        & = \frac{|D\OfSpecificValue{Age}{$\leq$}{5}|}{|D\OfAttribute{Age}|} \text{Info}(D\OfSpecificValue{Age}{$\leq$}{5}) + \frac{|D\OfSpecificValue{Age}{$>$}{5}|}{|D\OfAttribute{Age}|} \text{Info}(D\OfSpecificValue{Age}{$>$}{5})           \\
						                                                        & = \frac{3}{6} \left(-\frac{2}{3} \log_2\left(\frac{2}{3}\right) - \frac{1}{3} \log_2\left(\frac{1}{3}\right)\right) + \frac{3}{6} \left(-\frac{1}{3} \log_2\left(\frac{1}{3}\right) - \frac{2}{3} \log_2\left(\frac{2}{3}\right)\right) \\
						                                                        & = \frac{3}{6} \cdot 0,9183 + \frac{3}{6} \cdot 0,9183                                                                                                                                                                                   \\
						                                                        & = 0,9183                                                                                                                                                                                                                                \\
						                        \text{Gain}\OfAttribute{Age}    & = \text{Info}(D) - \text{Info}\OfAttribute{Age}(D)                                                                                                                                                                                      \\
						                                                        & = 1 - 0,9183                                                                                                                                                                                                                            \\
						                                                        & = 0,0817                                                                                                                                                                                                                                \\
					                        \end{alignat*}

				                  \end{enumerate}

				                  Therefore, the Information Gain for the attribute \textit{Age} is $0,817$ (if we split at $25,5$).

				            \item \textbf{Attribute \textit{Major}:}

				                  \textit{Major} is a categorical attribute with two possible values: \textit{CS} and \textit{DS}.

				                  Since it is a categorical attribute and we are using the Information Gain, there is no need to determine a splitting criterion.

				                  \begin{alignat*}{2}
					                  \text{Info}\OfAttribute{Major}(D) & = \sum_{j=1}^v \frac{|D\OfAttribute{Major$,j$}|}{|D\OfAttribute{Major}|} \text{Info}(D_{A\OfAttribute{Major$,j$}})                                                                                                                      \\
					                                                    & = \frac{|D\OfSpecificValue{Major}{=}{CS}|}{|D\OfAttribute{Major}|} \text{Info}(D\OfSpecificValue{Major}{=}{CS}) + \frac{|D\OfSpecificValue{Major}{=}{DS}|}{|D\OfAttribute{Major}|} \text{Info}(D\OfSpecificValue{Major}{=}{DS})         \\
					                                                    & = \frac{1}{6} \left(-\frac{1}{1} \log_2\left(\frac{1}{1}\right) - \frac{0}{1} \log_2\left(\frac{0}{1}\right)\right) + \frac{5}{6} \left(-\frac{2}{5} \log_2\left(\frac{2}{5}\right) - \frac{3}{5} \log_2\left(\frac{3}{5}\right)\right) \\
					                                                    & = \frac{1}{6} \left(- 0 - 0 \cdot \text{undefined}\right) + \frac{5}{6} \left(0,9710\right) \text{\tiny \textit{\hspace{0,2cm}\textbf{Hint:} Multiplication by 0 always results in 0}}                                                  \\
					                                                    & = \frac{1}{6} \left(- 0 - 0 \right) + \frac{5}{6} \left(0,9710\right)                                                                                                                                                                   \\
					                                                    & = 0,8090                                                                                                                                                                                                                                \\
					                  \text{Gain}\OfAttribute{Major}    & = \text{Info}(D) - \text{Info}\OfAttribute{Major}(D)                                                                                                                                                                                    \\
					                                                    & = 1 - 0,8090                                                                                                                                                                                                                            \\
					                                                    & = 0,1910                                                                                                                                                                                                                                \\
				                  \end{alignat*}

				            \item \textbf{Attribute \textit{Participation}:}

				                  \textit{Participation} is a categorical attribute with three possible values: \textit{High}, \textit{Medium} and \textit{Low}.

				                  Since it is a categorical attribute and we are using the Information Gain, there is no need to determine a splitting criterion.

				                  \begin{alignat*}{3}
					                  \text{Info}\OfAttribute{Parti.}(D) & = \sum_{j=1}^v \frac{|D\OfAttribute{Parti.$,j$}|}{|D\OfAttribute{Parti.}|} \text{Info}(D_{A\OfAttribute{Parti.$,j$}})                                                                                                                             \\
					                                                     & = \frac{|D\OfSpecificValue{Parti.}{=}{High}|}{|D\OfAttribute{Parti.}|} \text{Info}(D\OfSpecificValue{Parti.}{=}{High}) + \frac{|D\OfSpecificValue{Parti.}{=}{Medium}|}{|D\OfAttribute{Parti.}|} \text{Info}(D\OfSpecificValue{Parti.}{=}{Medium}) \\
					                                                     & + \frac{|D\OfSpecificValue{Parti.}{=}{Low}|}{|D\OfAttribute{Parti.}|} \text{Info}(D\OfSpecificValue{Parti.}{=}{Low})                                                                                                                              \\
					                                                     & = \frac{2}{6} \left(-\frac{2}{2} \log_2\left(\frac{2}{2}\right) - \frac{0}{2} \log_2\left(\frac{0}{2}\right)\right) + \frac{2}{6} \left(-\frac{1}{2} \log_2\left(\frac{1}{2}\right) - \frac{1}{2} \log_2\left(\frac{1}{2}\right)\right)           \\
					                                                     & + \frac{2}{6} \left(-\frac{0}{2} \log_2\left(\frac{0}{2}\right) - \frac{2}{2} \log_2\left(\frac{2}{2}\right)\right)                                                                                                                               \\
					                                                     & = \frac{2}{6} \cdot 0 + \frac{2}{6} \cdot 1 + \frac{2}{6} \cdot 0                                                                                                                                                                                 \\
					                                                     & =  0,3333                                                                                                                                                                                                                                         \\
					                  \text{Gain}\OfAttribute{Parti.}    & = \text{Info}(D) - \text{Info}\OfAttribute{Parti.}(D)                                                                                                                                                                                             \\
					                                                     & = 1 - 0,3333                                                                                                                                                                                                                                      \\
					                                                     & = 0,6667                                                                                                                                                                                                                                          \\
				                  \end{alignat*}
			            \end{enumerate}

			      \item \textbf{Create the node based on the highest Information Gain:}

			            The attribute with the highest Information Gain is \textit{Participation} with a value of $0,6667$. It will therefore become the splitting attribute for the root node.

			            The resulting tree will look like this:

			            \begin{center}
				            \begin{tikzpicture}[
						            >=latex,
						            thick,
						            node/.style={
								            draw,
								            rounded corners=.25em,
								            text depth=0.2em
							            },
						            leaf/.style={
								            draw,
								            rounded corners=.7em,
								            text depth=0.2em
							            },
						            branch/.style={
								            fill=white,
								            font=\ttfamily\scriptsize,
								            rounded corners=.7em,
								            text depth=0.2em
							            }
					            ]
					            \useasboundingbox (-6,0.5) rectangle (6,-6);

					            \node[node, anchor=center] at (0,0) (parti) {Participation?};

					            \node[draw=none,anchor=center] at (-4,-2.5) (high) {
						            \resizebox{6.5cm}{!}{%
							            \begin{tabular}{|c|c|c|c|c|}
								            \hline
								            \textbf{Age} & \textbf{Major} & \textbf{Participation} & \textbf{Passed} \\ \hline
								            23           & CS             & High                   & Yes             \\ \hline
								            26           & DS             & High                   & Yes             \\ \hline
							            \end{tabular}
						            }
					            };

					            \draw[rounded corners=5pt] (parti.south) -- (0,-1) -- node[above, draw=none] {high} (-4,-1) -- (high.north);

					            \node[draw=none,anchor=center] at (0,-4.25)  (medium) {
						            \resizebox{6.5cm}{!}{%
							            \begin{tabular}{|c|c|c|c|c|}
								            \hline
								            \textbf{Age} & \textbf{Major} & \textbf{Participation} & \textbf{Passed} \\ \hline
								            24           & DS             & Medium                 & Yes             \\ \hline
								            26           & DS             & Medium                 & No              \\ \hline
							            \end{tabular}
						            }
					            };

					            \draw[rounded corners=5pt] (parti.south) -- (0,-1) -- node[above, draw=none, fill=white] {medium} (0,-2.5) -- (medium.north);

					            \node[draw=none,anchor=center] at (4,-2.5)  (low) {
						            \resizebox{6.5cm}{!}{%
							            \begin{tabular}{|c|c|c|c|c|}
								            \hline
								            \textbf{Age} & \textbf{Major} & \textbf{Participation} & \textbf{Passed} \\ \hline
								            23           & DS             & Low                    & No              \\ \hline
								            26           & DS             & Low                    & No              \\ \hline
							            \end{tabular}
						            }
					            };

					            \draw[rounded corners=5pt] (parti.south) -- (0,-1) -- node[above, draw=none] {low} (4,-1) -- (low.north);

				            \end{tikzpicture}
			            \end{center}
		      \end{enumerate}

		\item \textbf{Visit each branch:}

		      \begin{enumerate}
			      \item \textbf{Branch \textit{High}:}

			            All samples in the partial dataset of the branch \textit{High} have the same value for the target attribute \textit{Passed}. Therefore, the branch becomes a leaf node.

			            \begin{center}
				            \begin{tikzpicture}[
						            >=latex,
						            thick,
						            node/.style={
								            draw,
								            rounded corners=.25em,
								            text depth=0.2em
							            },
						            leaf/.style={
								            draw,
								            rounded corners=.7em,
								            text depth=0.2em
							            },
						            branch/.style={
								            fill=white,
								            font=\ttfamily\scriptsize,
								            rounded corners=.7em,
								            text depth=0.2em
							            }
					            ]
					            \useasboundingbox (-6,0.5) rectangle (6,-6);

					            \node[node, anchor=center] at (0,0) (parti) {Participation?};

					            \node[leaf] at (-4,-2.5) (high) {Yes};

					            \draw[rounded corners=5pt] (parti.south) -- (0,-1) -- node[above, draw=none] {high} (-4,-1) -- (high.north);

					            \node[draw=none,anchor=center] at (0,-4.25)  (medium) {
						            \resizebox{6.5cm}{!}{%
							            \begin{tabular}{|c|c|c|c|c|}
								            \hline
								            \textbf{Age} & \textbf{Major} & \textbf{Participation} & \textbf{Passed} \\ \hline
								            24           & DS             & Medium                 & Yes             \\ \hline
								            26           & DS             & Medium                 & No              \\ \hline
							            \end{tabular}
						            }
					            };

					            \draw[rounded corners=5pt] (parti.south) -- (0,-1) -- node[above, draw=none, fill=white] {medium} (0,-2.5) -- (medium.north);

					            \node[draw=none,anchor=center] at (4,-2.5)  (low) {
						            \resizebox{6.5cm}{!}{%
							            \begin{tabular}{|c|c|c|c|c|}
								            \hline
								            \textbf{Age} & \textbf{Major} & \textbf{Participation} & \textbf{Passed} \\ \hline
								            23           & DS             & Low                    & No              \\ \hline
								            26           & DS             & Low                    & No              \\ \hline
							            \end{tabular}
						            }
					            };

					            \draw[rounded corners=5pt] (parti.south) -- (0,-1) -- node[above, draw=none] {low} (4,-1) -- (low.north);

				            \end{tikzpicture}
			            \end{center}

			      \item \textbf{Branch \textit{Medium}:}

			            The partial dataset of the branch \textit{Medium} contains samples with different values for the target attribute \textit{Passed}. Therefore, we need to create a new node.

			            \begin{enumerate}
				            \item \textbf{Calculate the Entropy of the target attribute \textit{Passed}:}

				                  \begin{alignat*}{2}
					                  \text{Info}(D\OfSpecificValue{Parti}{=}{Medium}) & = -\sum_{i=1}^{m}p_i \log_2(p_i)                                                                                                                          \\
					                                                                   & = -p\OfSpecificValue{Passed}{=}{Yes} \log_2(p\OfSpecificValue{Passed}{=}{Yes})- p\OfSpecificValue{Passed}{=}{No} \log_2(p\OfSpecificValue{Passed}{=}{No}) \\
					                                                                   & = -\frac{1}{2} \log_2\left(\frac{1}{2}\right) - \frac{1}{2} \log_2\left(\frac{1}{2}\right)                                                                \\
					                                                                   & = 1                                                                                                                                                       \\
				                  \end{alignat*}

				            \item \textbf{Calculate the Information Gain for all attributes that are not yet a node:}

				                  \begin{enumerate}
					                  \item \textbf{Attribute \textit{Age}:}

					                        \textit{Age} is still a continuous attribute. However since there are only two different values in the partial dataset, we only have one split point.

					                        \begin{alignat*}{2}
						                        \text{Info}\OfAttribute{Age}(D\OfSpecificValue{Parti}{=}{Medium}) & = \sum_{j=1}^v \frac{|D\OfAttribute{Age$,j$}|}{|D\OfAttribute{Age}|} \text{Info}(D_{A\OfAttribute{Age$,j$}})                                                                                                                            \\
						                                                                                          & = \frac{|D\OfSpecificValue{Age}{$\leq$}{25}|}{|D\OfAttribute{Age}|} \text{Info}(D\OfSpecificValue{Age}{$\leq$}{25}) + \frac{|D\OfSpecificValue{Age}{$>$}{25}|}{|D\OfAttribute{Age}|} \text{Info}(D\OfSpecificValue{Age}{$>$}{25})       \\
						                                                                                          & = \frac{1}{2} \left(-\frac{1}{1} \log_2\left(\frac{1}{1}\right) - \frac{0}{1} \log_2\left(\frac{0}{1}\right)\right) + \frac{1}{2} \left(-\frac{1}{1} \log_2\left(\frac{1}{1}\right) - \frac{0}{1} \log_2\left(\frac{0}{1}\right)\right) \\
						                                                                                          & = \frac{1}{2} \cdot 0 + \frac{1}{2} \cdot 0                                                                                                                                                                                             \\
						                                                                                          & = 0                                                                                                                                                                                                                                     \\
						                        \text{Gain}\OfAttribute{Age}                                      & = \text{Info}(D\OfSpecificValue{Parti}{=}{Medium}) - \text{Info}\OfAttribute{Age}(D\OfSpecificValue{Parti}{=}{Medium})                                                                                                                  \\
						                                                                                          & = 1 - 0                                                                                                                                                                                                                                 \\
						                                                                                          & = 1                                                                                                                                                                                                                                     \\
					                        \end{alignat*}

					                  \item \textbf{Attribute \textit{Major}:}

					                        \textit{Major} is still a categorical attribute. This time we only have one value in the partial dataset. Therefore, the Information Gain is $0$:

					                        \begin{alignat*}{2}
						                        \text{Info}\OfAttribute{Major}(D\OfSpecificValue{Parti}{=}{Medium}) & = \sum_{j=1}^v \frac{|D\OfAttribute{Major$,j$}|}{|D\OfAttribute{Major}|} \text{Info}(D_{A\OfAttribute{Major$,j$}})       \\
						                                                                                            & = \frac{|D\OfSpecificValue{Major}{=}{DS}|}{|D\OfAttribute{Major}|} \text{Info}(D\OfSpecificValue{Major}{=}{DS})          \\
						                                                                                            & = \frac{2}{2} \left(-\frac{1}{2} \log_2\left(\frac{1}{2}\right) - \frac{1}{2} \log_2\left(\frac{1}{2}\right)\right)      \\
						                                                                                            & = \frac{2}{2} \left(1\right)                                                                                             \\
						                                                                                            & = 1                                                                                                                      \\
						                        \text{Gain}\OfAttribute{Major}                                      & = \text{Info}(D\OfSpecificValue{Parti}{=}{Medium}) - \text{Info}\OfAttribute{Major}(D\OfSpecificValue{Parti}{=}{Medium}) \\
						                                                                                            & = 1 - 1                                                                                                                  \\
						                                                                                            & = 0                                                                                                                      \\
					                        \end{alignat*}
				                  \end{enumerate}

				            \item \textbf{Create the node based on the highest Information Gain:}

				                  The attribute with the highest Information Gain is \textit{Age} with a value of $1$. It will therefore become the splitting attribute for the node.

				                  The resulting tree will look like this:

				                  \begin{center}
					                  \begin{tikzpicture}[
							                  >=latex,
							                  thick,
							                  node/.style={
									                  draw,
									                  rounded corners=.25em,
									                  text depth=0.2em
								                  },
							                  leaf/.style={
									                  draw,
									                  rounded corners=.7em,
									                  text depth=0.2em
								                  },
							                  branch/.style={
									                  fill=white,
									                  font=\ttfamily\scriptsize,
									                  rounded corners=.7em,
									                  text depth=0.2em
								                  }
						                  ]
						                  \useasboundingbox (-6,0.5) rectangle (6,-7);

						                  \node[node, anchor=center] at (0,0) (parti) {Participation?};

						                  \node[leaf] at (-4,-2.5) (high) {Yes};

						                  \draw[rounded corners=5pt] (parti.south) -- (0,-1) -- node[above, draw=none] {high} (-4,-1) -- (high.north);

						                  \node[node,anchor=center] at (0,-4.25)  (medium) {Age?};

						                  \draw[rounded corners=5pt] (parti.south) -- (0,-1) -- node[above, draw=none, fill=white] {medium} (0,-2.5) -- (medium.north);

						                  \node[draw=none,anchor=center] at (4,-2.5)  (low) {
							                  \resizebox{6.5cm}{!}{%
								                  \begin{tabular}{|c|c|c|c|c|}
									                  \hline
									                  \textbf{Age} & \textbf{Major} & \textbf{Participation} & \textbf{Passed} \\ \hline
									                  23           & DS             & Low                    & No              \\ \hline
									                  26           & DS             & Low                    & No              \\ \hline
								                  \end{tabular}
							                  }
						                  };

						                  \draw[rounded corners=5pt] (parti.south) -- (0,-1) -- node[above, draw=none] {low} (4,-1) -- (low.north);

						                  \node[leaf] at (-2,-6) (yes) {Yes};
						                  \node[leaf] at (2,-6) (no) {No};

						                  \draw[rounded corners=5pt] (medium.south) -- (0,-5) -- node[left, draw=none, fill=white] {$\leq 25$} (-1,-5) -- (-2,-5) -- (yes.north);

						                  \draw[rounded corners=5pt] (medium.south) -- (0,-5) -- node[right, draw=none, fill=white] {$> 25$} (1,-5) -- (2,-5) -- (no.north);
					                  \end{tikzpicture}
				                  \end{center}

			            \end{enumerate}
			      \item \textbf{Branch \textit{Low}:}

			            All samples in the partial dataset of the branch \textit{Low} have the same value for the target attribute \textit{Passed}. Therefore, the branch becomes a leaf node.

			            \begin{center}
				            \begin{tikzpicture}[
						            >=latex,
						            thick,
						            node/.style={
								            draw,
								            rounded corners=.25em,
								            text depth=0.2em
							            },
						            leaf/.style={
								            draw,
								            rounded corners=.7em,
								            text depth=0.2em
							            },
						            branch/.style={
								            fill=white,
								            font=\ttfamily\scriptsize,
								            rounded corners=.7em,
								            text depth=0.2em
							            }
					            ]
					            \useasboundingbox (-6,0.5) rectangle (6,-5);

					            \node[node, anchor=center] at (0,0) (parti) {Participation?};

					            \node[leaf] at (-4,-2.5) (high) {Yes};

					            \draw[rounded corners=5pt] (parti.south) -- (0,-1) -- node[above, draw=none] {high} (-4,-1) -- (high.north);

					            \node[node,anchor=center] at (0,-2.75)  (medium) {Age?};

					            \draw[rounded corners=5pt] (parti.south) -- (0,-1) -- node[above, draw=none, fill=white] {medium} (0,-2.5) -- (medium.north);

					            \node[leaf,anchor=center] at (4,-2.5)  (low) {No};

					            \draw[rounded corners=5pt] (parti.south) -- (0,-1) -- node[above, draw=none] {low} (4,-1) -- (low.north);

					            \node[leaf] at (-2,-4) (yes) {Yes};
					            \node[leaf] at (2,-4) (no) {No};

					            \draw[rounded corners=5pt] (medium.south) -- (0,-3.5) -- node[left, draw=none, fill=white] {$\leq 25$} (-1,-3.5) -- (-2,-3.5) -- (yes.north);

					            \draw[rounded corners=5pt] (medium.south) -- (0,-3.5) -- node[right, draw=none, fill=white] {$> 25$} (1,-3.5) -- (2,-3.5) -- (no.north);
				            \end{tikzpicture}
			            \end{center}

		      \end{enumerate}

		\item \textbf{Stop the algorithm:}

		      Since all branches are now leaf nodes, the algorithm can be stopped.

		      The final decision tree is:

		      \begin{center}
			      \begin{tikzpicture}[
					      >=latex,
					      thick,
					      node/.style={
							      draw,
							      rounded corners=.25em,
							      text depth=0.2em
						      },
					      leaf/.style={
							      draw,
							      rounded corners=.7em,
							      text depth=0.2em
						      },
					      branch/.style={
							      fill=white,
							      font=\ttfamily\scriptsize,
							      rounded corners=.7em,
							      text depth=0.2em
						      }
				      ]
				      \useasboundingbox (-6,0.5) rectangle (6,-5);

				      \node[node, anchor=center] at (0,0) (parti) {Participation?};

				      \node[leaf] at (-4,-2.5) (high) {Yes};

				      \draw[rounded corners=5pt] (parti.south) -- (0,-1) -- node[above, draw=none] {high} (-4,-1) -- (high.north);

				      \node[node,anchor=center] at (0,-2.75)  (medium) {Age?};

				      \draw[rounded corners=5pt] (parti.south) -- (0,-1) -- node[above, draw=none, fill=white] {medium} (0,-2.5) -- (medium.north);

				      \node[leaf,anchor=center] at (4,-2.5)  (low) {No};

				      \draw[rounded corners=5pt] (parti.south) -- (0,-1) -- node[above, draw=none] {low} (4,-1) -- (low.north);

				      \node[leaf] at (-2,-4) (yes) {Yes};
				      \node[leaf] at (2,-4) (no) {No};

				      \draw[rounded corners=5pt] (medium.south) -- (0,-3.5) -- node[left, draw=none, fill=white] {$\leq 25$} (-1,-3.5) -- (-2,-3.5) -- (yes.north);

				      \draw[rounded corners=5pt] (medium.south) -- (0,-3.5) -- node[right, draw=none, fill=white] {$> 25$} (1,-3.5) -- (2,-3.5) -- (no.north);
			      \end{tikzpicture}
		      \end{center}
	\end{enumerate}
\end{solution}


\section*{Exercise 2: TODO}

TODO

\section*{Exercise 3: TODO}

TODO


\end{document}
