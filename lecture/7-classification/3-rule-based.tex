\section{Rule-Based Classification}

\begin{frame}{Basic Concepts}
	\begin{itemize}
		\item Rule-based classification is based on a set of \textbf{IF-THEN rules}.
		\item Each \textbf{IF-THEN rule} consists of two parts:
		      \begin{itemize}
			      \item \textbf{\color{airforceblue}IF} (antecedent/precondition): a condition or set of conditions that must be satisfied.
			      \item \textbf{\color{airforceblue}THEN} (consequent): the conclusion or action that follows if the IF part is satisfied.
		      \end{itemize}
		      \visible<2->{
		\item \textbf{Example:} (one rule)
		      \begin{itemize}
			      \item \textbf{IF} \texttt{age} $\leq$ 30 AND \texttt{student} = "yes" \textbf{THEN} \texttt{buys\_computer} = "yes".
		      \end{itemize}
		      }
		      \visible<3->{
		\item Very \textbf{easy} to read and understand for humans.
		      }
	\end{itemize}
\end{frame}

\begin{frame}{Example}
	\begin{itemize}
		\item Given is a \textbf{\color{airforceblue}set of rules}:
		      \begin{itemize}
			      \item \textbf{IF} \texttt{price} $<$ 1500 \textbf{THEN} \texttt{buy} = "yes".
			      \item \textbf{IF} \texttt{price} $\geq$ 1500 AND \texttt{color} = "red" \textbf{THEN} \texttt{buy} = "no".
			      \item \textbf{IF} \texttt{price} $\geq$ 1500 AND \texttt{location} = "Erlangen" \textbf{THEN} \texttt{buy} = "yes".
		      \end{itemize} \medskip
		      \visible<2->{
		\item The set may be used to classify \textbf{\color{airforceblue}new tuples}: \medskip
		      \begin{center}
			      \scalebox{0.85}{
				      \begin{tabular}{|c|c|c|c|}
					      \hline
					      \rowcolor{faugray!62}\textbf{price}                                 & \textbf{color}                                                       & \textbf{location}                                                       & \textbf{buy}                                                                   \\\hline
					      % Scenario 1: Exactly one rule is triggered
					      \only<3-4>{\cellcolor{faugray!28}}1349                              & \only<3-4>{\cellcolor{faugray!28}}red                                & \only<3-4>{\cellcolor{faugray!28}}Nuremberg                             & \only<3-4>{\cellcolor{faugray!28}}\only<-3>{?}\only<4->{{\color{faugreen}yes}} \\\hline
					      % Scenario 2: More than one rule is triggered
					      \only<5>{\cellcolor{faugray!28}}\only<6>{\cellcolor{faured!28}}2306 & \only<5>{\cellcolor{faugray!28}}\only<6>{\cellcolor{faured!28}}red   & \only<5>{\cellcolor{faugray!28}}\only<6>{\cellcolor{faured!28}}Erlangen & \only<5>{\cellcolor{faugray!28}}\only<6>{\cellcolor{faured!28}}?
					      \\\hline
					      % Scenario 3: No rule is triggered
					      \only<7>{\cellcolor{faugray!28}}\only<8>{\cellcolor{faured!28}}1995 & \only<7>{\cellcolor{faugray!28}}\only<8>{\cellcolor{faured!28}}green & \only<7>{\cellcolor{faugray!28}}\only<8>{\cellcolor{faured!28}}Fuerth   & \only<7>{\cellcolor{faugray!28}}\only<8>{\cellcolor{faured!28}}?
					      \\\hline
				      \end{tabular}
			      }
		      \end{center}
		      } \medskip

		      \visible<6->{
		\item Some scenarios might lead to \textbf{\color{airforceblue}conflicts}:
		      \begin{enumerate}
			      \item More than one rule is triggered.
			            \visible<8->{
			      \item No rule is triggered.
			            }
		      \end{enumerate}
		      }

	\end{itemize}

\end{frame}

\begin{frame}{Potential Solutions}
	\begin{enumerate}
		\item \textbf{More than one rule is triggered: {\color{airforceblue}conflict resolution}.}
		      \begin{itemize}
			      \item \textbf{\color{airforceblue}Size ordering:}
			            \begin{itemize}
				            \item Assign the highest priority to the triggered rule that has the "toughest" requirement \\ (i.e., rule with most used attribute in condition).
			            \end{itemize}
			      \item \textbf{\color{airforceblue}Class-based ordering:}
			            \begin{itemize}
				            \item Decreasing order of prevalence or misclassification cost per class.
				            \item No order of rules within class $\rightarrow$ disjunction (logical \texttt{OR}) between rules.
			            \end{itemize}
			      \item \textbf{\color{airforceblue}Rule-based ordering} (decision list):
			            \begin{itemize}
				            \item Rules are organized into one long priority list,\\
				                  according to some measure of rule quality, or by experts.
				            \item Rules \underline{must} be applied in this particular order to avoid conflict.
			            \end{itemize}
		      \end{itemize}
		      \visible<2->{
		\item \textbf{No rule is triggered.}
		      \begin{itemize}
			      \item Use a fallback/default rule.
			      \item Always evaluated as the last rule, if and only if other rules are not covered by some tuple, i. e. no rules have been triggered.
		      \end{itemize}
		      }
	\end{enumerate}
\end{frame}

\begin{frame}{Rule Extraction from a Decision Tree}
	\begin{itemize}
		\item \textbf{Basic idea:}
		      \begin{itemize}
			      \item Rules are \textbf{\color{airforceblue}easier to understand} than large trees.
			      \item A rule can be created for \textbf{\color{airforceblue}each path from the root to a leaf.}
			      \item Each attribute-value pair along the path forms \textbf{\color{airforceblue}a condition}:
		      \end{itemize}
	\end{itemize}
	\vspace*{0em}
	\visible<2->{
		\begin{center}
			\begin{columns}
				\begin{column}{0.45\textwidth}
					\begin{figure}[h]
						\centering
						\scalebox{0.9}{
							\begin{tikzpicture}[
		overlay,
		remember picture,
		>=latex,
		thick,
		node/.style={
				draw=faugray,
				rounded corners=.25em,
				fill=faugray!62,
				text depth=0.2em
			},
		leaf/.style={
				draw,
				rounded corners=.7em,
				text depth=0.2em
			},
		branch/.style={
				fill=white,
				font=\ttfamily\scriptsize,
				rounded corners=.7em,
				text depth=0.2em
			}
	]
	\node[node] at (0,0) (age) {age?};
	\node[leaf,text=faugreen,below=6.2em of age] (age-yes) {yes};

	\node[node,below left=2em and 1.5em of age] (student) {student?};
	\node[leaf,text=faured,below right=2.5em and -1em of student] (student-no) {no};
	\node[leaf,text=faugreen,below left=2.5em and -1em of student] (student-yes) {yes};

	\node[node,below right=2em and .6em of age] (credit-rating) {credit rating?};
	\node[leaf,text=faured,below left=2.5em and -2em of credit-rating] (credit-no) {no};
	\node[leaf,text=faugreen,below right=2.5em and -2em of credit-rating] (credit-yes) {yes};

	\draw (age.south) -- (age-yes.north);
	\node[branch,below=4em of age.north] {31\dots 40};

	\draw[rounded corners=5pt]
	(age.south) -- ($(age.south) + (0,-1em)$) --
	($(student.north) + (0,1em)$) -- (student.north);
	\node[branch,above=1em of student.north] {$\leq$30};

	\path[draw,rounded corners=5pt]
	(age.south) -- ($(age.south) + (0,-1em)$) --
	($(credit-rating.north) + (0,1em)$) -- (credit-rating.north);
	\node[branch,above=1em of credit-rating.north] {>40};

	% student labels
	\draw[rounded corners=5pt]
	(student.south) -- ($(student.south) + (0,-1.5em)$) --
	($(student-no.north) + (0,1em)$) -- (student-no.north);
	\node[branch,above=1em of student-no.north] {no};

	\draw[rounded corners=5pt]
	(student.south) -- ($(student.south) + (0,-1.5em)$) --
	($(student-yes.north) + (0,1em)$) -- (student-yes.north);
	\node[branch,above=1em of student-yes.north] {yes};

	% credit-rating labels
	\draw[rounded corners=5pt]
	(credit-rating.south) -- ($(credit-rating.south) + (0,-1.5em)$) --
	($(credit-no.north) + (0,1em)$) -- (credit-no.north);
	\node[branch,above=1em of credit-no.north] {excellent};

	\draw[rounded corners=5pt]
	(credit-rating.south) -- ($(credit-rating.south) + (0,-1.5em)$) --
	($(credit-yes.north) + (0,1em)$) -- (credit-yes.north);
	\node[branch,above=1em of credit-yes.north] {fair};
\end{tikzpicture}

						}
					\end{figure}
				\end{column}
				\begin{column}{0.45\textwidth}
					\begin{enumerate}
						\visible<3->{
						\item \textbf{IF} \texttt{age} $\leq$ 30 AND \texttt{student} = "yes" \\
						      \textbf{THEN} \texttt{buys\_computer} = "yes".
						      }
						      \visible<4->{
						\item \textbf{IF} \texttt{age} $\leq$ 30 AND \texttt{student} = "no" \\
						      \textbf{THEN} \texttt{buys\_computer} = "no".
						      }
						      \visible<5->{
						\item \textbf{IF} \texttt{age}= $31\ldots40$ \\
						      \textbf{THEN} \texttt{buys\_computer} = "yes".
						      }
						      \visible<6->{
						\item \dots
						      }
					\end{enumerate}
				\end{column}
			\end{columns}
		\end{center}
	}
\end{frame}

\begin{frame}{Rule Induction: Sequential Covering Method (I)}
	\begin{itemize}
		\item Extracting rules from decision trees is not \textbf{the only} way to learn rules.
		\item Rules can be learned \textbf{\color{airforceblue}directly} from \textbf{\color{airforceblue}the training data}:
		      \begin{itemize}
			      \item Rules are learned \textbf{sequentially}.
			      \item Each rule is optimized to cover \textbf{as many tuples of a given class} as possible\\ while covering \textbf{as few tuples of other classes} as possible.
		      \end{itemize}
	\end{itemize}

	\visible<2->{
		\begin{center}
			\begin{columns}
				\begin{column}{0.45\textwidth}
					\begin{itemize}
						\item \textbf{Steps of the method:}
						      \begin{enumerate}
							      \visible<3->{
							      \item Start with an empty set of rules.
							            }
							            \visible<4->{
							      \item Find the rule $r$ with the best covering.
							            }
							            \visible<5->{
							      \item Remove all tuples covered.
							            }
							            \visible<6->{
							      \item Repeat with step 2 until:
							            \begin{itemize}
								            \item No more tuples left.
								            \item The quality of a rule is below a threshold.
							            \end{itemize}
							            }
						      \end{enumerate}
					\end{itemize}

				\end{column}
				\begin{column}{0.45\textwidth}
					\begin{center}
						\newcommand*{\xMin}{0}%
						\newcommand*{\xMax}{8}%
						\newcommand*{\yMin}{0}%
						\newcommand*{\yMax}{5}%
						\newcommand*{\colorOne}{faured}%
						\newcommand*{\colorTwo}{faugray}%
						\begin{tikzpicture}[scale=0.5]
							\foreach \i in {\xMin,...,\xMax} {
									\draw [very thin,gray] (\i,\yMin) -- (\i,\yMax)  node [below] at
									(\i,\yMin) {\tiny$\i$};
								}
							\foreach \i in {\yMin,...,\yMax} {
									\draw [very thin,gray] (\xMin,\i) -- (\xMax,\i) node [left] at
									(\xMin,\i) {\tiny$\i$};
								}
							% \colorOne Nodes (35)
							\node[circle,fill=\colorOne,scale=0.5] at (0.7,0.8) {};
							\node[circle,fill=\colorOne,scale=0.5] at (1.2,1.3) {};
							\node[circle,fill=\colorOne,scale=0.5] at (0.9,2.1) {};
							\node[circle,fill=\colorOne,scale=0.5] at (1.5,3.2) {};
							\node[circle,fill=\colorOne,scale=0.5] at (1.8,1.7) {};
							\node[circle,fill=\colorOne,scale=0.5] at (2.1,0.9) {};
							\node[circle,fill=\colorOne,scale=0.5] at (2.3,1.4) {};
							\node[circle,fill=\colorOne,scale=0.5] at (1.9,2.3) {};
							\node[circle,fill=\colorOne,scale=0.5] at (0.8,3.5) {};
							\node[circle,fill=\colorOne,scale=0.5] at (1.3,4.2) {};
							\node[circle,fill=\colorOne,scale=0.5] at (2.2,4.1) {};
							\node[circle,fill=\colorOne,scale=0.5] at (2.4,3.3) {};
							\node[circle,fill=\colorOne,scale=0.5] at (3.2,1.1) {};
							\node[circle,fill=\colorOne,scale=0.5] at (3.7,0.7) {};
							\node[circle,fill=\colorOne,scale=0.5] at (4.1,1.5) {};
							\node[circle,fill=\colorOne,scale=0.5] at (4.6,0.9) {};
							\node[circle,fill=\colorOne,scale=0.5] at (5.3,1.2) {};
							\node[circle,fill=\colorOne,scale=0.5] at (5.9,0.8) {};
							\node[circle,fill=\colorOne,scale=0.5] at (6.2,1.4) {};
							\node[circle,fill=\colorOne,scale=0.5] at (6.7,0.9) {};
							\node[circle,fill=\colorOne,scale=0.5] at (7.1,1.6) {};
							\node[circle,fill=\colorOne,scale=0.5] at (6.8,2.3) {};
							\node[circle,fill=\colorOne,scale=0.5] at (7.2,3.1) {};
							\node[circle,fill=\colorOne,scale=0.5] at (6.5,3.7) {};
							\node[circle,fill=\colorOne,scale=0.5] at (7.3,4.2) {};
							\node[circle,fill=\colorOne,scale=0.5] at (5.8,4.3) {};
							\visible<-4>{\node[circle,fill=\colorOne,scale=0.5] at (5.1,3.9) {};}
							\node[circle,fill=\colorOne,scale=0.5] at (3.1,2.3) {};
							\node[circle,fill=\colorOne,scale=0.5] at (4.4,2.4) {};
							\node[circle,fill=\colorOne,scale=0.5] at (3.8,2.1) {};
							\node[circle,fill=\colorOne,scale=0.5] at (4.2,1.8) {};
							\node[circle,fill=\colorOne,scale=0.5] at (4.9,2.2) {};
							\visible<-4>{\node[circle,fill=\colorOne,scale=0.5] at (2.4,2.7) {};}
							\visible<-4>{\node[circle,fill=\colorOne,scale=0.5] at (3.9,3.2) {};}
							\visible<-4>{\node[circle,fill=\colorOne,scale=0.5] at (5.4,2.9) {};}

							% \colorTwo Nodes (45)
							\node[circle,fill=\colorTwo,scale=0.5] at (1.1,1.9) {};
							\node[circle,fill=\colorTwo,scale=0.5] at (1.7,2.7) {};
							\node[circle,fill=\colorTwo,scale=0.5] at (2.9,2.2) {};
							\visible<-4>{\node[circle,fill=\colorTwo,scale=0.5] at (2.9,2.7) {};}
							\visible<-4>{\node[circle,fill=\colorTwo,scale=0.5] at (3.1,2.9) {};}
							\visible<-4>{\node[circle,fill=\colorTwo,scale=0.5] at (3.4,3.2) {};}
							\visible<-4>{\node[circle,fill=\colorTwo,scale=0.5] at (3.7,3.6) {};}
							\visible<-4>{\node[circle,fill=\colorTwo,scale=0.5] at (3.9,4.1) {};}
							\visible<-4>{\node[circle,fill=\colorTwo,scale=0.5] at (3.3,3.8) {};}
							\visible<-4>{\node[circle,fill=\colorTwo,scale=0.5] at (2.8,3.5) {};}
							\visible<-4>{\node[circle,fill=\colorTwo,scale=0.5] at (2.7,4.0) {};}
							\visible<-4>{\node[circle,fill=\colorTwo,scale=0.5] at (3.2,4.3) {};}
							\visible<-4>{\node[circle,fill=\colorTwo,scale=0.5] at (3.8,2.8) {};}
							\visible<-4>{\node[circle,fill=\colorTwo,scale=0.5] at (4.1,3.1) {};}
							\visible<-4>{\node[circle,fill=\colorTwo,scale=0.5] at (4.3,3.5) {};}
							\visible<-4>{\node[circle,fill=\colorTwo,scale=0.5] at (4.6,3.9) {};}
							\visible<-4>{\node[circle,fill=\colorTwo,scale=0.5] at (4.8,4.2) {};}
							\visible<-4>{\node[circle,fill=\colorTwo,scale=0.5] at (4.5,2.8) {};}
							\visible<-4>{\node[circle,fill=\colorTwo,scale=0.5] at (4.9,3.3) {};}
							\visible<-4>{\node[circle,fill=\colorTwo,scale=0.5] at (5.2,3.6) {};}
							\visible<-4>{\node[circle,fill=\colorTwo,scale=0.5] at (5.4,4.1) {};}
							\visible<-4>{\node[circle,fill=\colorTwo,scale=0.5] at (5.1,2.7) {};}
							\node[circle,fill=\colorTwo,scale=0.5] at (2.6,1.2) {};
							\node[circle,fill=\colorTwo,scale=0.5] at (2.9,1.8) {};
							\node[circle,fill=\colorTwo,scale=0.5] at (3.4,1.9) {};
							\node[circle,fill=\colorTwo,scale=0.5] at (4.3,2.1) {};
							\node[circle,fill=\colorTwo,scale=0.5] at (4.8,1.8) {};
							\node[circle,fill=\colorTwo,scale=0.5] at (5.7,1.9) {};
							\node[circle,fill=\colorTwo,scale=0.5] at (6.1,2.1) {};
							\node[circle,fill=\colorTwo,scale=0.5] at (6.3,2.8) {};
							\node[circle,fill=\colorTwo,scale=0.5] at (6.4,3.3) {};
							\node[circle,fill=\colorTwo,scale=0.5] at (5.7,3.4) {};
							\node[circle,fill=\colorTwo,scale=0.5] at (0.7,1.4) {};
							\node[circle,fill=\colorTwo,scale=0.5] at (0.9,3.1) {};
							\node[circle,fill=\colorTwo,scale=0.5] at (1.4,3.8) {};
							\node[circle,fill=\colorTwo,scale=0.5] at (7.0,1.0) {};
							\node[circle,fill=\colorTwo,scale=0.5] at (6.9,3.4) {};
							\visible<-4>{\node[circle,fill=\colorTwo,scale=0.5] at (4.2,2.6) {};}
							\visible<-4>{\node[circle,fill=\colorTwo,scale=0.5] at (3.6,2.6) {};}
							\visible<-4>{\node[circle,fill=\colorTwo,scale=0.5] at (4.7,3.6) {};}
							\visible<-4>{\node[circle,fill=\colorTwo,scale=0.5] at (3.5,4.0) {};}
							\visible<-4>{\node[circle,fill=\colorTwo,scale=0.5] at (4.1,3.7) {};}
							\visible<-4>{\node[circle,fill=\colorTwo,scale=0.5] at (4.4,4.1) {};}
							\visible<-4>{\node[circle,fill=\colorTwo,scale=0.5] at (5.0,3.1) {};}
							\visible<-4>{\node[circle,fill=\colorTwo,scale=0.5] at (3.8,3.3) {};}
							\visible<-4>{\node[circle,fill=\colorTwo,scale=0.5] at (2.8,3.1) {};}

							\node[very thin,black,right] at (\xMax + 0.5, \yMax - 0.5) (l) {\scriptsize Legend:};

							\node[circle,fill=\colorOne,scale=0.5,below=1mm of l] (ptw) {};
							\node[very thin,black,right=0.25mm of ptw] {\scriptsize Class 1};
							\node[circle,fill=\colorTwo,scale=0.5,below=1mm of ptw] (pth) {};
							\node[very thin,black,right=0.25mm of pth] {\scriptsize Class 2};


							\visible<4->{
								% Major Class 2 Area
								\draw[thick, dashed, faugraydark] (2.5, 2.5) rectangle (5.5, 4.5);
							}

							\visible<6->{
								% Major Class 1 Area
								\draw[thick, dashed, faureddark] (3, 0.5) rectangle (6.8, 1.7);
							}
						\end{tikzpicture}
					\end{center}
				\end{column}
			\end{columns}
		\end{center}
	}
\end{frame}

\begin{frame}{Rule Induction: Sequential Covering Method (II)}
	\begin{itemize}
		\item \textbf{Typical sequential covering algorithms:}
		      \begin{itemize}
			      \item {\only<2->{\color{airforceblue}}FOIL\only<1>{\footnote{\fullcite{quinlan1990}}}}, AQ\only<1>{\footnote{\fullcite{michalski1986}}}, CN2\only<1>{\footnote{\fullcite{clark1989}}}, RIPPER\only<1>{\footnote{\fullcite{cohen1995}}}.
		      \end{itemize}
		      \only<2->{
		\item \textbf{\color{airforceblue}FOIL (First-Order Inductive Learner):}
		      \begin{itemize}
			      \item Based on Information Gain
			      \item Suppose we have two rules:
			            \begin{align*}
				            R  & : \text{IF condition THEN class} = c  \\
				            R' & : \text{IF condition' THEN class} = c
			            \end{align*}
			      \item $pos/neg$ are $\#$ of positive/negative tuples covered by $R$, $pos'/neg'$ respectively for $R'$.
			      \item FOIL assesses the information gained by extending \textit{condition'} as
			            \begin{align*}
				            \text{FOIL\_Gain} = \text{pos}' \left( \log_2 \frac{\text{pos}'}{\text{pos}' + \text{neg}'} - \log_2 \frac{\text{pos}}{\text{pos}+\text{neg}} \right).
			            \end{align*}
			      \item FOIL favors rules that have high accuracy and cover many positive tuples.
		      \end{itemize}
		      }
	\end{itemize}
\end{frame}
