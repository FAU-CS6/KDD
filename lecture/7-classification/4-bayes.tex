\section{Bayes Classification Methods}

\begin{frame}{Bayesian Classification: Concepts}
	\begin{itemize}
		\item \textbf{Bayesian Classification:}
		      \begin{itemize}
			      \item Statistical classification method.
			      \item Predict class membership probabilities.
			      \item Based on \textbf{\color{airforceblue}Bayes' Theorem}.
		      \end{itemize}
		      \visible<3->{
		\item \textbf{Our Focus:}
		      \begin{itemize}
			      \item \textbf{\color{airforceblue}Naïve Bayesian Classification}
			            \begin{itemize}
				            \item Assumes conditional independence of attributes ("naïve").
				            \item Simplification of Bayesian classification.
			            \end{itemize}
		      \end{itemize}
		      }
	\end{itemize}

	\vspace*{0.75cm}

	\visible<2->{
		\begin{block}{Bayes' Theorem\footnote{\fullcite{bayes1763}}}
			\textbf{Bayes' Theorem} describes the probability of an event based on prior knowledge of conditions that might be related to the event.
		\end{block}
	}
\end{frame}

\begin{frame}{Bayesian Classification: Basic Terms}
	\begin{itemize}
		\item \textbf{Let ...}
		      \begin{itemize}
			      \item \textbf{... $X$ be a {\color{airforceblue}data sample} ("evidence").}
			            \begin{itemize}
				            \item The class label shall be unknown.
			            \end{itemize}
			      \item \textbf{... $C_i$ be the {\color{airforceblue}hypothesis} that $X$ belongs to class $i$.}
		      \end{itemize}
		      \visible<2->{
		\item \textbf{Then our goal is to determine the {\color{airforceblue}Posteriori Probability} $P(C_i|X)$:}
		      \begin{itemize}
			      \item The probability that the hypothesis holds given the observed data sample $X$.
		      \end{itemize}
		      }
		      \visible<3->{
		\item \textbf{To determine $P(C_i|X)$, we need to know:}
		      \begin{itemize}
			      \item \textbf{The {\color{airforceblue}Prior Probability} $P(C_i)$:}
			            \begin{itemize}
				            \item The overall probability of the class $i$.
				            \item E.g., $X$ will buy computer, regardless of age, income, $\ldots$.
			            \end{itemize}
			      \item \textbf{The {\color{airforceblue}Likelihood} $P(X|C_j)$:}
			            \begin{itemize}
				            \item The probability of observing the sample $X$ given that the hypothesis holds.
				            \item E.g., given that $X$ buys computer, the probability that $X$ is $31\ldots40$, medium income.
			            \end{itemize}
			      \item \textbf{The {\color{airforceblue}Probability} of the sample data $P(X)$:}
			            \begin{itemize}
				            \item The probability that sample data is observed.
			            \end{itemize}
		      \end{itemize}
		      }
	\end{itemize}
\end{frame}

\begin{frame}{Bayesian Classification: Posteriori Probability}
	\begin{itemize}
		\item \textbf{The {\color{airforceblue}Posteriori Probability} $P(C_i|X)$ follows from the Bayes' Theorem:}
		      \begin{align*}
			      P(C_i|X) = \frac{P(X|C_i)P(C_i)}{P(X)}.
		      \end{align*}

		      \vspace*{0.25cm}

		      \visible<2->{
			      \begin{block}{The Maximum Posteriori Probability}
				      Since the posteriori probability $P(C_i|X)$ is the basically the probability that $X$ belongs to class $C_i$, we want to find the class $C_i$ that \textbf{maximizes} this probability. $X$ \textbf{is classified} as belonging to this class.
			      \end{block}
		      }

		      \vspace*{0.5cm}

		      \visible<3->{
		\item \textbf{Since $P(X)$ is constant for all classes, we only have to maximize:}
		      \begin{align*}
			      P(X|C_i)P(C_i).
		      \end{align*}
		      }
	\end{itemize}
\end{frame}

\begin{frame}{Bayesian Classification: Likelihood}
	\begin{itemize}
		\item \textbf{{\color{airforceblue}Naïve} Bayesian Classification:}
		      \begin{itemize}
			      \item \textbf{Assumption:} All attributes are \textit{conditionally independent}.
			      \item I.e. no dependence relation between attributes (which is "naïve").
			            \begin{align*}
				            \resizebox{7cm}{!}{%
					            $P(X|C_i) = \prod_{k=1}^{n} P(x_k|C_i) = P(x_1|C_i)P(x_2|C_i)\cdots P(x_n|C_i).$}
			            \end{align*}
			      \item Greatly reduces the computation cost.
		      \end{itemize}
	\end{itemize}

	\vspace*{-1.5em}


	\begin{columns}
		\visible<3->{
			\begin{column}{0.45\textwidth}
				\begin{center}
					\textbf{Categorical Attribute}
				\end{center}
				\vspace*{-1em}
				\begin{itemize}
					\item $P(x_k|C_i)$ is the number of tuples in $C_i$ having value $x_k$ for $A_k$
					      divided by $|C_{i,D}|$ (the number of tuples of $C_i$ in $D$):
				\end{itemize}
			\end{column}
		}
		\visible<4->{
			\begin{column}{0.45\textwidth}
				\begin{center}
					\textbf{Continuous-valued Attribute}
				\end{center}
				\vspace*{-1em}

				\begin{itemize}
					\item $P(x_k|C_i)$ is usually computed based on Gaussian distribution with a mean $\mu$ and standard deviation $\sigma$:
				\end{itemize}

			\end{column}
		}
	\end{columns}

	\vspace*{0.2cm}

	\only<-4>{
		\begin{columns}
			\visible<3->{
				\begin{column}{0.5\textwidth}
					{
						\footnotesize
						\begin{align*}
							P(x_k|C_i) & = \frac{|\{t \in C_i: t.A_k = x_k\}|}{|C_{i,D}|} \
						\end{align*}
					}
				\end{column}
			}
			\visible<4->{
				\begin{column}{0.5\textwidth}
					{
						\footnotesize
						\begin{align*}
							G(x,\mu,\sigma) & = \frac{1}{\sqrt{2\pi}\sigma}e^{-\frac{(x-\mu)^2}{2\sigma^2}} \\
							P(x_k|C_i)      & = G(x_k,\mu_{C_i},\sigma_{C_i})
						\end{align*}
					}
				\end{column}
			}
		\end{columns}
	}
\end{frame}

\begin{frame}{Bayesian Classification: Example (I)}
	\vspace*{-0.5cm}
	\begin{columns}
		\begin{column}{0.45\textwidth}
			\begin{itemize}
				\item \textbf{Target attribute:} buys\_computer \medskip
				\item \textbf{Data sample:}
			\end{itemize}
			{
			\footnotesize
			\begin{align*}
				X = ( & \texttt{age} \leq 30,             \\
				      & \texttt{income} = "medium",       \\
				      & \texttt{student} = "yes",         \\
				      & \texttt{credit\_rating} = "fair")
			\end{align*}
			}
			\vspace*{-0.35cm}
			\visible<2->{
				\begin{itemize}
					\item \textbf{Prior Probability $P(C_i)$:}
				\end{itemize}
				\visible<3->{
					{
							\vspace*{-0.15cm}
							\scriptsize
							\begin{align*}
								P({\color{faugreen}yes}) & = \visible<4->{\frac{9}{14} \approx 0.643} \\
								P({\color{faured}no})    & = \visible<5->{\frac{5}{14} \approx 0.357}
							\end{align*}
						}
				}
			}
		\end{column}
		\begin{column}{0.45\textwidth}
			\vspace*{0.25cm}
			\begin{center}
				\scalebox{0.7}{
					\begin{tabular}{|l|l|c|c|c|}
	\hline
	\rowcolor{faugray!62}\textbf{age} & \textbf{income} & \textbf{student} & \textbf{credit\_rating} & \textbf{buys\_computer} \\\hline
	$\leq 30$                         & high            & no               & fair                    & {\color{faured}no}      \\\hline
	$\leq 30$                         & high            & no               & excellent               & {\color{faured}no}      \\\hline
	$31\ldots40$                      & high            & no               & fair                    & {\color{faugreen}yes}   \\\hline
	$>40$                             & medium          & no               & fair                    & {\color{faugreen}yes}   \\\hline
	$>40$                             & low             & yes              & fair                    & {\color{faugreen}yes}   \\\hline
	$>40$                             & low             & yes              & excellent               & {\color{faured}no}      \\\hline
	$31\ldots40$                      & low             & yes              & excellent               & {\color{faugreen}yes}   \\\hline
	$\leq 30$                         & medium          & no               & fair                    & {\color{faured}no}      \\\hline
	$\leq 30$                         & low             & no               & fair                    & {\color{faugreen}yes}   \\\hline
	$>40$                             & medium          & yes              & fair                    & {\color{faugreen}yes}   \\\hline
	$\leq 30$                         & medium          & yes              & excellent               & {\color{faugreen}yes}   \\\hline
	$31\ldots40$                      & medium          & no               & excellent               & {\color{faugreen}yes}   \\\hline
	$31\ldots40$                      & high            & yes              & fair                    & {\color{faugreen}yes}   \\\hline
	$>40$                             & medium          & no               & excellent               & {\color{faured}no}      \\\hline
\end{tabular}

				}
			\end{center}
		\end{column}
	\end{columns}
\end{frame}

\begin{frame}{Bayesian Classification: Example (II)}
	\vspace*{-0.5cm}
	\begin{columns}
		\begin{column}{0.45\textwidth}
			\begin{itemize}
				\item \textbf{Likelihood(s) $P(X_k|C_i)$:}
			\end{itemize}
			{
			\vspace*{-0.15cm}
			\scriptsize
			\visible<2->{
				\begin{align*}
					P(\texttt{age} \leq 30 | {\color{faugreen}yes})             & = \visible<3->{\frac{2}{9} \approx 0.222} \\
					P(\texttt{age} \leq 30 | {\color{faured}no})                & = \visible<4->{\frac{3}{5} = 0.6}         \\
					P(\texttt{income} = "medium" | {\color{faugreen}yes})       & = \visible<5->{\frac{4}{9} \approx 0.444} \\
					P(\texttt{income} = "medium" | {\color{faured}no})          & = \visible<6->{\frac{2}{5} = 0.4}         \\ % After four iterations, the lecturer-student interaction tends to diminish — starting with this formula, we immediately present the results.
					P(\texttt{student} = "yes" | {\color{faugreen}yes})         & = \visible<7->{\frac{5}{9} \approx 0.556} \\
					P(\texttt{student} = "yes" | {\color{faured}no})            & = \visible<7->{\frac{1}{5} = 0.2}         \\
					P(\texttt{credit\_rating} = "fair" | {\color{faugreen}yes}) & = \visible<7->{\frac{6}{9} \approx 0.667} \\
					P(\texttt{credit\_rating} = "fair" | {\color{faured}no})    & = \visible<7->{\frac{2}{5} = 0.4}         \\
				\end{align*}
			}
			}
		\end{column}
		\begin{column}{0.45\textwidth}
			\only<-7>{
				\vspace*{0.25cm}
				\begin{center}
					\scalebox{0.7}{
						\begin{tabular}{|l|l|c|c|c|}
	\hline
	\rowcolor{faugray!62}\textbf{age} & \textbf{income} & \textbf{student} & \textbf{credit\_rating} & \textbf{buys\_computer} \\\hline
	$\leq 30$                         & high            & no               & fair                    & {\color{faured}no}      \\\hline
	$\leq 30$                         & high            & no               & excellent               & {\color{faured}no}      \\\hline
	$31\ldots40$                      & high            & no               & fair                    & {\color{faugreen}yes}   \\\hline
	$>40$                             & medium          & no               & fair                    & {\color{faugreen}yes}   \\\hline
	$>40$                             & low             & yes              & fair                    & {\color{faugreen}yes}   \\\hline
	$>40$                             & low             & yes              & excellent               & {\color{faured}no}      \\\hline
	$31\ldots40$                      & low             & yes              & excellent               & {\color{faugreen}yes}   \\\hline
	$\leq 30$                         & medium          & no               & fair                    & {\color{faured}no}      \\\hline
	$\leq 30$                         & low             & no               & fair                    & {\color{faugreen}yes}   \\\hline
	$>40$                             & medium          & yes              & fair                    & {\color{faugreen}yes}   \\\hline
	$\leq 30$                         & medium          & yes              & excellent               & {\color{faugreen}yes}   \\\hline
	$31\ldots40$                      & medium          & no               & excellent               & {\color{faugreen}yes}   \\\hline
	$31\ldots40$                      & high            & yes              & fair                    & {\color{faugreen}yes}   \\\hline
	$>40$                             & medium          & no               & excellent               & {\color{faured}no}      \\\hline
\end{tabular}

					}
				\end{center}
			}
			\only<8->{
				\begin{itemize}
					\item \textbf{Likelihood(s) $P(X|C_i)$:}
				\end{itemize}
				\vspace*{-0.15cm}
				{
					\scriptsize
					\begin{align*}
						P(X | {\color{faugreen}yes}) & = \visible<9->{0.222 \cdot 0.444 \cdot 0.556 \cdot 0.667 \approx 0.037} \\
						P(X | {\color{faured}no})    & = \visible<10->{0.6 \cdot 0.4 \cdot 0.2 \cdot 0.4 = 0.019}
					\end{align*}
				}

				\visible<11->{
					\begin{itemize}
						\item \textbf{Calculate $P(X | C_i) \cdot P(C_i)$\only<11->{\footnote{\textbf{\color{airforceblue}Reminder:} Calculating the numerator of the posterior probability is sufficient to classify $X$}}:}
					\end{itemize}
					\vspace*{-0.15cm}
					{
						\scriptsize
						\visible<12->{
							\begin{align*}
								P(X | {\color{faugreen}yes}) \cdot  P({\color{faugreen}yes}) & = 0.024. \\
								P(X | {\color{faured}no}) \cdot  P({\color{faured}no})       & = 0.007.
							\end{align*}
						}
					}
				}
				\vspace*{-0.35cm}
				\visible<13->{
					\begin{itemize}
						\item \textbf{Classification Result:}
						      \visible<14->{
							      \begin{itemize}
								      \item $X$ belongs to class $C_1$ \\ (\texttt{buys\_computer} = {\color{faugreen}yes})
							      \end{itemize}
						      }

					\end{itemize}
				}
				\vspace*{0.5cm}
			}

		\end{column}
	\end{columns}
\end{frame}


\begin{frame}{Bayesian Classification: Problem}
	\begin{alertblock}{Zero-Probability Problem}
		Our naïve Bayes classifier performs poorly if any of the conditional probabilities \textbf{any} of the conditional probabilities $P(x_k|C_i)$ is zero, as this causes the \textbf{entire product} of probabilities to become zero.
	\end{alertblock}


	\begin{itemize}
		\visible<2->{
		\item \textbf{Example:}
		      \begin{itemize}
			      \item \texttt{income} = "low" (0 tuples), "medium" (990 tuples), "high" (10 tuples).
		      \end{itemize}
		      }
		      \visible<3->{
		\item \textbf{Solution:}
		      \begin{itemize}
			      \item \textbf{Use {\color{airforceblue}Laplacian correction} (or Laplacian estimator):}
			            \begin{itemize}
				            \item Add $1$ to each case:
			            \end{itemize}
		      \end{itemize}
		      }
	\end{itemize}
	\visible<3->{
		{
				\scriptsize
				\begin{align*}
					P(\texttt{income} = "low")    & = \frac{1}{1003}   \\
					P(\texttt{income} = "medium") & = \frac{991}{1003} \\
					P(\texttt{income} = "high")   & = \frac{11}{1003}
				\end{align*}
			}
	}
\end{frame}

\begin{frame}{Bayesian Classification: Summary}
	\begin{itemize}
		\item \textbf{Advantages of the Naïve Bayes Classifier:}
		      \begin{itemize}
			      \item Easy to implement.
			      \item Good results obtained in most of the cases.
		      \end{itemize}
		\item \textbf{Disadvantages of the Naïve Bayes Classifier:}
		      \begin{itemize}
			      \item Assumption: class conditional independence, therefore loss of accuracy.
			      \item Practically, \textbf{dependencies} exist among most variables.
			            \begin{itemize}
				            \item Cannot be modeled by Naïve Bayesian Classifier.
				            \item \textbf{\color{airforceblue}How to deal with these dependencies?} \\ $\rightarrow$ Bayesian Belief Networks (not part of KDD\footnote{\textbf{More info in the reference book (Chapter 9.1):} \fullcite{han2011}}).
			            \end{itemize}
		      \end{itemize}
	\end{itemize}
\end{frame}
