\section{Ensemble Methods: Increasing the Accuracy}

\begin{frame}{Ensemble Methods}
	\begin{block}{Ensemble Method}
		An \textit{ensemble method} creates a composite model that consists of several models such as to form one model.
	\end{block}
	\begin{itemize}
		\item Most of the time \textit{weak} learners are combined to mitigate their respective individual shortcomings.
		\item Data set is partitioned into $k$ training sets.
		\item Train a classifier on each training set.
		\item Every individual classifier returns its prediction.
		\item Overall prediction is determined for instance by \textit{majority voting}.
		\item Prediction typically more accurate than each individual model.\\
		      $\rightarrow$ Returns better results when diversity among models is great.
		\item \textit{Each classifier should perform better than random guessing.}
		\item \textbf{Popular methods} include Bagging, Boosting, and Random Forests.
	\end{itemize}
\end{frame}

\begin{frame}{Bagging: Boostrap Aggregation}
	\begin{itemize}
		\item \textbf{Analogy:}
		      \begin{itemize}
			      \item Diagnosis based on multiple doctors' majority vote.
		      \end{itemize}
		\item \textbf{Training:}
		      \begin{itemize}
			      \item Given a set $D$ of d tuples, at each iteration $i$, a training set $D_i$ of $d$ tuples is sampled with replacement from $D$ (i.e., bootstrap).
			      \item A classifier model $M_i$ is learned for each training set $D_i$.
		      \end{itemize}
		\item \textbf{Prediction in the case of classification: classify an unknown sample $X$.}
		      \begin{itemize}
			      \item Each classifier $M_i$ returns its class prediction.
			      \item The bagged classifier $M^*$ counts the votes and assigns the class with the most votes to $X$.
		      \end{itemize}
		\item \textbf{Prediction of a real number} (i. e. regression or time series forecast):
		      \begin{itemize}
			      \item Can be applied to the prediction of continuous values by taking the average value of each prediction for a given test tuple.
		      \end{itemize}
		\item \textbf{Accuracy:}
		      \begin{itemize}
			      \item Often significantly better than a single classifier derived from $D$.
			      \item For noisy data: not considerably worse, more robust.
			      \item Proved improved accuracy in prediction.
		      \end{itemize}
	\end{itemize}
\end{frame}


\begin{frame}{Boosting}
	\begin{itemize}
		\item \textbf{Analogy:}
		      \begin{itemize}
			      \item Consult several doctors, based on a combination of weighted diagnoses -- weight assigned based on the previous diagnosis accuracy
		      \end{itemize}
		\item \textbf{How boosting works:}
		      \begin{itemize}
			      \item Weights are assigned to each training tuple.
			      \item A series of $k$ classifiers is iteratively learned.
			      \item After a classifier $M_i$ is learned, the weights are updated to allow the subsequent classifier, $M_{i+1}$ to pay more attention to the training tuples that were misclassified by $M_i$.
			      \item The final $M^*$ combines the votes of each individual classifier, where the weight of each classifier's vote is a function of its accuracy.
		      \end{itemize}
		\item \textbf{Classification:}
		      \begin{itemize}
			      \item Each classifier $M_i$ returns its class prediction.
			      \item The bagged classifier $M^*$ counts the votes and assigns the class with the most votes to $X$.
		      \end{itemize}
		\item \textbf{Boosting algorithm can be extended for numeric prediction.}
	\end{itemize}
\end{frame}


\begin{frame}{AdaBoost ("Adaptive Boosting" \footfullcite[Algorithm AdaBoost.M1 on p. 131]{freund1997}): Training}
	\vspace*{-1em}
	\begin{itemize}
		\item \textbf{Given a data set $D$ of $d$ class-labeled tuples: $(x_1 , y_1), \ldots, (x_d, y_d)$} with $y_d \in Y = \{1, \dots, c\}$.
		\item \textbf{Initialize emtpy lists to hold information per classifier:} $\textbf{w}, \boldsymbol{\beta}, \textbf{M} \leftarrow $ empty list.
		\item \textbf{Initialize weights for first classifier to hold same probability for each tuple:} $w_i^1 \leftarrow \frac{1}{d}$
		\item \textbf{Generate $k$ classifiers in $k$ iterations. At iteration $i$,}
		      \begin{enumerate}
			      \item Calculate ``normalized'' weights: $p^i = \frac{\textbf{w}^i}{\sum_{j=1}^N w_j^i}$
			      \item Sample dataset with replacement according to previously assigned weights to form training set $D_i$.
			      \item Derive classification model $M_i$ from $D_i$.
			      \item Calculate error by using $D_i$ as a test set as follows: $\varepsilon_i = \text{error}(M_i) = \sum_{j=1}^{d} w_j \cdot \text{err}(x_j)$.\\
			            $\text{err}(x_j)$ refers to the missclassification error on tuple $x_j$. If missclassified $\text{err}(x_j)=1$, else $0$.
			      \item If $\text{error}(M_i)>0.5$: Abandon this classifier and go back to step 1.
			      \item Calculate $\beta_i = \frac{\varepsilon_i}{1 - \varepsilon_i}$.
			      \item Update weights for next iteration: If a tuple has been correctly classified, multiply its ``normalized'' weight with $\beta_i$:  $w^{i+1}_i=w^i_j \beta^{1-\text{err}(x_j)}_i$. \textit{If a tuple is misclassified, its weight is increased, otherwise it is decreased.}
			      \item Add $w^{i+1}$, $M_i$, and $\beta_i$ to their respective lists.
		      \end{enumerate}
	\end{itemize}
\end{frame}

\begin{frame}{AdaBoost ("Adaptive Boosting" \footfullcite[Algorithm AdaBoost.M1 on p. 131]{freund1997}): Prediction}
	\begin{itemize}
		\item \textbf{Initialize weight of each class to zero.}
		\item \textbf{For each classifier $i$ in $k$ classifiers:}
		      \begin{enumerate}
			      \item Calculate the weight of this classifier's vote: $w_i = \log(\frac{1}{\beta_i})$.
			      \item Get class prediction $c$ for (single) tuple $x$ from current weak classifier $M_i$: $c = M_i(x)$.
			      \item Add $w_i$ to weight for class $c$.
		      \end{enumerate}
		\item \textbf{Return predicted class with the largest weight.}
		\item Mathematically, this can be formulated as:
		      \begin{align*}
			      \textstyle M(x)= \arg \max_{y\in Y} \sum_{i=1}^k (\log \frac{1}{\beta_i})M_i(x)
		      \end{align*}
	\end{itemize}
\end{frame}

\begin{frame}{Random Forest\footfullcite{breiman2001}}
	\begin{itemize}
		\item Ensemble method consisting only of decision trees where each tree has been generated using random selection of attributes at each node.
		\item Classification: Each tree votes and the most popular class is returned.
		\item \textbf{Two methods to construct random forests:} (each builds $k$ trees)
		      \begin{enumerate}
			      \item \underline{Forest-RI} (random input selection):
			            \begin{itemize}
				            \item Random sampling with replacement to obtain training data from $D$.
				            \item Set $F$ as the number of attributes to determine split at each node. $F$ is smaller than the number of available attributes.
				            \item Construct decision tree $M_i$ by randomly select candidates at each node. Use CART to grow tree to maximum size without pruning.
			            \end{itemize}
			      \item \underline{Forest-RC}: Similar to Forest-RI but new attributes (features) are generated by linear combinations of existing attributes to reduce correlation between individual classifiers. At each node, attributes are randomly selected.

		      \end{enumerate}
		\item \textbf{Comparable in accuracy to AdaBoost, but more robust to errors and outliers.}
		\item \textbf{Insensitive to the number of attributes selected for consideration at each split, and faster than bagging or boosting.}
	\end{itemize}
\end{frame}

\begin{frame}{Classification of Class-imbalanced Data Sets}
	\vspace*{-0.5em}
	\begin{block}{Class-Imbalanced Data}
		\textit{Class-Imbalanced Data} refers to data where the main class of interest (positive labeled) is only represented by a small number of tuples. E.g., medical diagnosis and fraud detection.
	\end{block}
	\begin{itemize}
		\item Problem because traditional methods assume \textit{equality between
			      classes},\\ i. e. a balanced distribution of classes and equal error
		      costs.
		\item \textbf{Typical methods for imbalanced data in binary classification:}
		      \begin{enumerate}
			      \item \textbf{\color{airforceblue}Undersampling/Oversampling:} Changes distribution of tuples in training data.
			            \begin{itemize}
				            \item \textit{Undersampling:} Randomly eliminate tuples from negative class.
				            \item \textit{Oversampling:} Re-samples data from positive class.\\ For instance, method SMOTE generates synthetic data that is similar to existing data using nearest neighbor.
			            \end{itemize}
			      \item \textbf{\color{airforceblue}Threshold-moving:} Moves the decision threshold, $t$, so that the rare-class tuples are easier to classify, and hence, less chance of costly false-negative errors. Works when class returns a probability.
			      \item \textbf{\color{airforceblue}Ensemble techniques}.
		      \end{enumerate}
		      Threshold-moving and ensemble methods work well on extremely imbalanced data.
		\item \textbf{Still difficult on multi-class tasks.}
	\end{itemize}
\end{frame}
