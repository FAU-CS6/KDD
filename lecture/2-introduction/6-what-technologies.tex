\section{What technologies are used?}

\begin{frame}{Data Mining: Confluence of Multiple Disciplines}
	\centering
	\resizebox{5.5cm}{5.5cm}{%
		\begin{tikzpicture}[mindmap, minimum size=4cm, every 
		node/.style=concept, concept color=black!40, grow cyclic]
			\node[concept] {\Huge Data mining}
			child [concept color=gray!40, minimum size=3cm]{node {\Large 
			Machine Learning}}
			child [concept color=gray!40, minimum size=3cm]{node {\Large 
			Pattern recognition}}
			child [concept color=gray!40, minimum size=3cm]{node {\Large 
			Statistics}}
			child [concept color=gray!40, minimum size=3cm]{node {\Large 
			Visualization}}
			child [concept color=gray!40, minimum size=3cm]{node {\Large 
			Database technology}}
			child [concept color=gray!40, minimum size=3cm]{node {\Large 
			Algorithms}}
			child [concept color=gray!40, minimum size=3cm]{node {\Large 
			Machine Learning}}
			child [concept color=gray!40, minimum size=3cm]{node {\Large 
			High-performance computing}};
	\end{tikzpicture}}
\end{frame}

\begin{frame}{Why Confluence of Multiple Disciplines?}
	\textbf{Tremendous amount of data:}
	\begin{itemize}
		\item Algorithms must be highly scalable to handle also terabytes of 
		data.
	\end{itemize}
	
	\textbf{High dimensionality of data:}
	\begin{itemize}
		\item DNA microarrays may have tens of thousands of dimensions.\\
		Collections of microscopic DNA spots attached to a solid surface.
	\end{itemize}
	
	\textbf{High complexity of data:}
	\begin{itemize}
		\item Data streams and sensor data.
		\item Time-series data, temporal data, sequence data.
		\item Structure data, graphs, social networks, and multi-linked data.
		\item Heterogeneous databases and legacy databases.
		\item Spatial, spatiotemporal, multimedia, text and web data.
		\item Software programs, scientific simulations.
	\end{itemize}
	\textbf{New and sophisticated applications.}
\end{frame}