\subsection{What Technologies Are Used?}

\begin{frame}{Confluence of Multiple Disciplines}
	\centering
	\resizebox{6.5cm}{6.5cm}{%
		\begin{tikzpicture}[mindmap, minimum size=4cm,  every
				node/.style=concept, concept color=faugraydark!60, grow cyclic,
				level 1/.style={level distance=6cm, sibling angle=360/10}]
			\node[concept] {\Huge Data mining}
			child [concept color=faugray!40, minimum size=3cm]{node {\Large
							Machine Learning}}
			child [concept color=faugray!40, minimum size=3cm]{node {\Large
							Statistics}}
			child [concept color=faugray!40, minimum size=3cm]{node {\Large
							Pattern Recognition}}
			child [concept color=faugray!40, minimum size=3cm]{node {\Large
							Visualization and Human-Computer Interaction}}
			child [concept color=faugray!40, minimum size=3cm]{node {\Large
							Algorithms}}
			child [concept color=faugray!40, minimum size=3cm]{node {\Large
							High-Performance Computing}}
			child [concept color=faugray!40, minimum size=3cm]{node {\Large
							Applications}}
			child [concept color=faugray!40, minimum size=3cm]{node {\Large
							Social Sciences}}
			child [concept color=faugray!40, minimum size=3cm]{node {\Large
							Database Technology}}
			child [concept color=faugray!40, minimum size=3cm]{node {\Large
							Natural Language Processing}};
		\end{tikzpicture}}
\end{frame}

\begin{frame}{Why Confluence of Multiple Disciplines?}
	\textbf{Each discipline contributes something different. E.g.:}
	\begin{itemize}
		\item \textbf{Algorithms:} \\ Basic algorithms to get started.
		\item \textbf{Machine Learning:} \\ Specialized algorithms for learning from data.
		\item \textbf{High-Performance computing:} \\ Parallel and distributed computing to handle large datasets.
		\item \textbf{Database Technologies:} \\ Efficent storage and retrieval of data.
		\item \textbf{Etc.:} \\ $\cdots$
	\end{itemize}
\end{frame}
