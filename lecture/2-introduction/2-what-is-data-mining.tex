\section{What is data mining?}

\begin{frame}{What is Data Mining?}
	\textbf{Data mining or knowledge discovery from data}:
	\begin{itemize}
		\item Extraction of interesting (\textbf{non-trivial, implicit,
			      previously unknown and potentially useful}) patterns from huge amounts
		      of data.
		\item Is \textbf{data mining} a misnomer?
	\end{itemize}
	Alternative names:
	\begin{itemize}
		\item Knowledge discovery/mining in databases (KDD).
		\item Knowledge extraction.
		\item Data/pattern analysis.
		\item Data archeology/dredging.
		\item Information harvesting.
		\item Business intelligence.
	\end{itemize}
\end{frame}

\begin{frame}{Examples: Is everything Data Mining?}
	Considered to be data mining:
	\begin{itemize}
		\item Analysis of customer behavior for user-related advertising.
		\item Analysis of payment histories for fraud detection.
		\item Analysis of infection behavior for better understanding of a
		      pandemic.
	\end{itemize}
	\textbf{NOT} considered to be data mining:
	\begin{itemize}
		\item Simple search for females in a customer database.
		\item Simple join of two database tables.
		\item Simple deductive database validating a new tuple with regards to
		      predefined constraints.
	\end{itemize}
\end{frame}

\begin{frame}{Data Mining in the Database-Systems Community}
	\begin{itemize}
		\item \textbf{Knowledge discovery pipeline} is a typical view from
		      the database-systems and data-warehousing community.
		\item Data mining plays an essential role in the knowledge-discovery
		      process.
	\end{itemize}
	\vspace{0.15cm}
	\centering
	\scalebox{0.95}{%
		\begin{tikzpicture}
			% Dialectics
			\node[draw] (Database) at (0,0) {Database};
			\node[draw,fill=black,text=white] (Data warehouse) at (2.8,1) {Data
				warehouse};
			\node[draw] (Task-relevant data) at (6,2) {Task-relevant data};
			\node[draw,fill=black,text=white] (Data mining) at (8.9,3)
			{Visualization};
			\node[draw] (Knowledge discovery) at (11.9,4) {Knowledge discovery};

			\node (Data Integration) at (3,0.25) {Data Integration};
			\node (Selection) at (5.3,1.3) {Selection};
			\node (Data Mining) at (8.7,2.25) {Data Mining};

			\draw node[vertex] (Joint1) at (1.3,0.46) {};
			\draw node[vertex] (Joint2) at (4.5,1.52) {};

			\draw node[vertex] (Joint3) at (4.5,0.46) {};
			\draw node[vertex] (Joint7) at (4.5,0.46) {};
			\draw node[vertex] (Joint8) at (7.6,0.46) {};
			\draw node[vertex] (Joint9) at (10.3,0.46) {};
			\draw node[vertex] (Joint10) at (10.3,-0.2) {};
			\draw node[vertex] (Joint11) at (1.3,-0.2) {};

			\draw node[vertex] (Joint5) at (7.6,2.55) {};
			\draw node[vertex] (Joint6) at (10.3,3.47) {};


			\draw[->,draw=black] (Database) to (Data warehouse);
			\draw[->,draw=black] (Data warehouse) to (Task-relevant data);
			\draw[->,draw=black] (Task-relevant data) to (Data mining);
			\draw[->,draw=black] (Data mining) to (Knowledge discovery);
			\draw[-,draw=black, dashed] (Joint1) to (Joint3);
			\draw[->,draw=black, dashed] (Joint3) to (Joint2);
			\draw[->,draw=black] (Joint6) to (Joint9);
			\draw[-,draw=black] (Joint9) to (Joint8);
			\draw[->,draw=black] (Joint8) to (Joint7);
			\draw[->,draw=black] (Joint8) to (Joint5);
			\draw[->,draw=black,dashed] (Joint10) to (Joint11);
			\draw[->,draw=black,dashed] (Joint9) to (Joint10);
			\draw[->,draw=black,dashed] (Joint11) to (Joint1);
		\end{tikzpicture}
	}
\end{frame}

\begin{frame}{Data Mining in the Business Community}
	\centering
	\begin{tikzpicture}
		\coordinate (A) at (-6,0) {};
		\coordinate (B) at ( 6,0) {};
		\coordinate (C) at (0,6) {};
		\draw[name path=AC] (A) -- (C);
		\draw[name path=BC] (B) -- (C);

		\node (Data Integration) at (1,5) {\parbox{\linewidth}{Increasing
				potential \\ to support decisions.}};
		\node (Data Integration) at (12.9,1) {\parbox{\linewidth}{Database \\
				administration.}};
		\node (Data Integration) at (12.9,3) {\parbox{\linewidth}{Data \\
				analyst.}};
		\node (Data Integration) at (12.9,4.5) {\parbox{\linewidth}{Business \\
				analyst.}};
		\node (Data Integration) at (12.9,5.5) {\parbox{\linewidth}{End user.}};

		\foreach \y/\A in {
		0/{
		\parbox{\linewidth}{
			\centering
			\textbf{Sources of data}: \\
			\small{paper, files, web documents, scientific experiments,
				database system.}}
		},
		1/{
		\parbox{\linewidth}{
			\centering
			\textbf{Data preprocessing/integration,\\
				data warehouses.}}
		},
		2/{
		\parbox{\linewidth}{
			\centering
			\textbf{Data exploration:} \\
			\small{statistics, querying, and reporting.}}
		},
		3/{
		\parbox{\linewidth}{
			\centering
			\textbf{Data mining:} \\
			\small{information discovery.}}
		},
		4/{
		\parbox{\linewidth}{
			\centering
			\textbf{Presentation:} \\
			\small{visualization techniques.}}
		},
		5/{
		\parbox{\linewidth}{
			\centering
			\textbf{Decision}}
		}
		} {
		\path[name path=horiz] (A|-0,\y) -- (B|-0,\y);
		\draw[name intersections={of=AC and horiz,by=P},
			name intersections={of=BC and horiz,by=Q}] (P) -- (Q)
		node[midway,above] {\A};
		}
	\end{tikzpicture}
\end{frame}

\begin{frame}{Data Mining in the Machine Learning and Statistics Community}
	Machine-learning and statistics communities usually classify data mining as
	the central part of their pipeline:

	\vspace{0.5cm}
	\centering
	\scalebox{0.95}{%
		\begin{tikzpicture}
			[node distance = 1cm, auto,font=\footnotesize,
				% STYLES
				every node/.style={node distance=1cm},
				% The comment style is used to describe the characteristics of each
				%force
				comment/.style={rectangle, inner sep= 5pt, text width=3.8cm, node
						distance=0.5cm, font=\scriptsize\sffamily},
				% The force style is used to draw the forces' name
				force/.style={rectangle, draw, fill=black!10, inner sep=5pt, text
						width=1.8cm, text badly centered, minimum height=1cm,
						font=\bfseries\footnotesize\sffamily}]

			% Draw forces
			\node [force, dashed] (a) {\parbox{\linewidth}{\centering Data \\
					input.}};
			\node [force, right=1cm of a] (b) {\parbox{\linewidth}{\centering
					Data
					\\ preprocessing.}};
			\node [force, right=1cm of b] (c) {\parbox{\linewidth}{\centering
					Data
					\\ mining.}};
			\node [force, right=1cm of c] (d) {\parbox{\linewidth}{\centering
					Post
					\\ processing.}};
			\node [force, dashed, right=1cm of d] (e)
			{\parbox{\linewidth}{\centering Pattern, \\ information, \\
					knowledge.}};

			%%%%%%%%%%%%%%%
			% Change data from here

			% SUPPLIERS
			\node [comment, below=0.25cm of b] {
				\begin{itemize}
					\item Data integration.
					\item Normalization.
					\item Feature selection.
					\item Dimension reduction.
				\end{itemize}
			};

			% USERS
			\node [comment, below=0.25 of c] {
				\begin{itemize}
					\item Pattern discovery.
					\item Association/correlation.
					\item Classification.
					\item Clustering.
					\item Outlier analysis.
					\item \ldots
				\end{itemize}
			};

			% PUBLIC POLICIES
			\node [comment, below=0.25 of d] {
				\begin{itemize}
					\item Pattern evaluation.
					\item Pattern selection.
					\item Pattern interpretation.
					\item Pattern visualization.
				\end{itemize}
			};

			\path[->,thick]
			(a) edge (b)
			(b) edge (c)
			(c) edge (d)
			(d) edge (e);

		\end{tikzpicture}
	}
\end{frame}

\begin{frame}{The Data Mining Process: CRISP-DM}
	\begin{itemize}
		\item \textbf{CRoss-Industry Standard Process for Data Mining}:
	\end{itemize}
	\vspace{0.5cm}
	\centering
	\scalebox{0.95}{%
		\begin{tikzpicture}
			[node distance = 2cm, auto,font=\footnotesize,
				% STYLES
				every node/.style={node distance=2cm},
				% The comment style is used to describe the characteristics of each
				%force
				comment/.style={rectangle, inner sep= 5pt, text width=5cm, node
						distance=0.5cm, font=\scriptsize\sffamily},
				% The force style is used to draw the forces' name
				force/.style={rectangle, draw, fill=black!10, inner sep=5pt, text
						width=4cm, minimum height=2cm, font=\footnotesize\sffamily}]

			% Draw forces
			\node [force] (a) {
				\parbox{\linewidth}{
					\textbf{1. Business understanding}.\\
					- Determine business objectives.\\
					- Assess situation.\\
					- Determine data mining goals.\\
					- Produce project plan.
				}
			};
			\node [force, right=1cm of a] (b) {
				\parbox{\linewidth}{
					\textbf{2. Data understanding}.\\
					- Collect initial data.\\
					- Describe data.\\
					- Explore data.\\
					- Verify data quality.
				}
			};
			\node [force, right=1cm of b] (c) {
				\parbox{\linewidth}{
					\textbf{3. Data preparation}.\\
					- Select data.\\
					- Clean data.\\
					- Construct data.\\
					- Integrate/format data.
				}
			};
			\node [force, below=1cm of c] (d) {
				\parbox{\linewidth}{
					\textbf{4. Modeling}.\\
					- Select modeling techniques.\\
					- Generate test design.\\
					- Build model.\\
					- Assess model.
				}
			};
			\node [force, below=1cm of a] (e) {
				\parbox{\linewidth}{
					\textbf{6. Deployment}.\\
					- Plan deployment.\\
					- Plan monitoring/maintenance.\\
					- Produce final report.\\
					- Review project.
				}
			};
			\node [force, below=1cm of b] (f) {
				\parbox{\linewidth}{
					\textbf{5. Evaluation}.\\
					- Evaluate results.\\
					- Review process.\\
					- Determine next steps.
				}
			};

			\path[->,thick]
			(b) edge (c)
			(d) edge (f)
			(f) edge (e)
			(f) edge (a);

			\path[<->,thick]
			(a) edge (b)
			(c) edge (d);
		\end{tikzpicture}
	}
\end{frame}
