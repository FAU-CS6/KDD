\subsection{What Data Is Available?}

\begin{frame}{What Kind of Data Can Be Mined? (I)}
	\begin{itemize}
		\item \textbf{Any kind of data as long as meaningful for the target
			      application.}
		\item Most basic forms of data sources:
		      \begin{itemize}
			      \item \textbf{Relational database:} \\
			            \small{Collection of tables, where the tables consist of a
				            set of attributes and usually a large set of tuples.}
			      \item \textbf{Data warehouse:} \\
			            \small{Repository of information collected from multiple
				            sources, stored under a unified schema.}
			      \item \textbf{Transactional database:} \\
			            \small{Captures transactions, such as customer purchases,
				            flight bookings, or user clicks on a website.}
		      \end{itemize}
	\end{itemize}
\end{frame}
\begin{frame}{What Kind of Data Can Be Mined? (II)}
	Advanced data sets and advanced applications:
	\begin{itemize}
		\item Data streams and sensor data.
		\item Time series data, temporal data, sequence data (incl.
		      biosequences).
		\item Structure data, graphs, social networks and multi-linked data.
		\item Object-relational databases.
		\item Heterogeneous databases and legacy databases.
		\item NoSQL databases.
		\item Spatial data and spatiotemporal data.
		\item Multimedia databases.
		\item Text databases.
		\item The world wide web.
	\end{itemize}
\end{frame}
