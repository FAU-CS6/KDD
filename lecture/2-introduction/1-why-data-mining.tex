\section{Why data mining?}

\begin{frame}{Why Data Mining? (I)}
	\textbf{The explosive growth of data: from terabytes to petabytes and
		more.}\\
	\begin{itemize}
		\item Data collection and availability:
		      \begin{itemize}
			      \item Automated data collection tools.
			      \item Database systems.
			      \item World wide web.
			      \item Computerized society.
			      \item Digitization.
		      \end{itemize}
		\item Major sources of abundant data:
		      \begin{itemize}
			      \item Business: web, e-commerce, transactions, stocks \ldots
			      \item Science: remote sensing, bioinformatics, scientific
			            simulation \ldots
			      \item Society: news, digital cameras, social media \ldots
		      \end{itemize}
		\item The era of \textbf{big data} (as inflationary used buzzword).
	\end{itemize}
\end{frame}

\begin{frame}{Why Data Mining? (II)}
	\textbf{The initial situation:}
	\begin{itemize}
		\item We are drowning in  data
		\item We are starving for knowledge
	\end{itemize}
	\textbf{The basic idea behind data mining:}
	\begin{itemize}
		\item We can analyze the data to satisfy our hunger for knowledge
	\end{itemize}
\end{frame}

\begin{frame}{Evolution of Sciences (I)}
	\begin{itemize}
		\item Before $1600$, era of \textbf{empirical science}.
		\item $1600-1950$s, rise of \textbf{theoretical science}.
		      \begin{itemize}
			      \item Each discipline has grown a theoretical component.
			      \item Theoretical models often motivate experiments and generalize
			            our understanding.
		      \end{itemize}
		\item $1950-1990$s, rise of \textbf{computational science}.
		      \begin{itemize}
			      \item Over the last $50$ years most disciplines have grown a third,
			            computational branch.
			            \begin{itemize}
				            \item E.g. empirical, theoretical, and computational ecology.
				            \item E.g. physics, linguistics or biology.
			            \end{itemize}
			      \item Computational science traditionally meant simulation.
			      \item It grew out of our inability to describe reality by
			            closed-form mathematical models.
		      \end{itemize}
	\end{itemize}
\end{frame}

\begin{frame}{Evolution of Sciences (II)}
	\begin{itemize}
		\item $1990-$now, rise of \textbf{data science}.
		      \begin{itemize}
			      \item The flood of data from new instruments and modern simulations.
			      \item The ability to economically store and manage petabytes of
			            data.
			      \item The internet makes all these archives world wide accessible.
			      \item Scientific \emph{information management}, \\
			            acquisition,\\
			            organization, \\
			            query, and \\
			            visualization scale almost linearly with amount of data.
			      \item \textbf{Data mining} is a major new challenge!
		      \end{itemize}
		\item For further reading:\\
		      \small{Jim Gray and Alex Szaly: \emph{The World Wide Telescope: An
				      Archetype for Online Science}, \\ Communications of the ACM
			      45(11):
			      50-54, 2002.}
	\end{itemize}
\end{frame}

\begin{frame}{Evolution of Sciences (III)}
	\begin{itemize}
		\item $1960$s: Data collection, database creation, \\
		      \hspace{1cm} integrated management systems (IMS), and \\
		      \hspace{1cm} network database management systems (DBMS).
		\item $1970$s: Relational data model, relational DBMS implementation
		      (RDBMS).
		\item $1980$s: RDBMS products, database creation, \\
		      \hspace{1cm} advanced data models (extended relational, object
		      oriented, deductive etc.),\\
		      \hspace{1cm} application-oriented DBMS (spatial, scientific,
		      engineering etc.).
		\item $1990$s: Data mining, data warehousing, multimedia databases, web
		      databases.
		\item $2000$s: Stream data management and mining,\\
		      \hspace{1cm} data mining and applications, \\
		      \hspace{1cm} web technology (XML, data integration), and global
		      information systems.
	\end{itemize}
\end{frame}
