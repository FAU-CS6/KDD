\section{Data Cleaning}

\begin{frame}{Dirty Data}
	\begin{itemize}
		\item \textbf{Data in the real world is {\color{airforceblue}dirty}.}
		\item \textbf{Lots of different kinds of dirty data:}
		      \begin{itemize}
			      \item \textbf{Incomplete data:} lacking attributes, lacking values or containing aggregate data.
			      \item \textbf{Inconsistencies:} containing discrepancies in codes or names.
			      \item \textbf{Errors:} containing incorrect values.
			      \item \textbf{Noise:} containing small inaccuracies.
			      \item \textbf{Outliers:} containing extreme values.
		      \end{itemize}
	\end{itemize}
\end{frame}

\begin{frame}{Dirty Data: Incomplete Data}
	\begin{columns}
		\begin{column}{0.45\textwidth}
			\begin{itemize}
				\item \textbf{Potential reasons:}
				      \begin{itemize}
					      \item Data not yet available.
					      \item Technical malfunction.
					      \item Human error.
					      \item etc.
				      \end{itemize}
				\item \textbf{Potential solutions:}
				      \begin{itemize}
					      \item Ignore the tuple.
					      \item Fill in the missing value manually.
					            \begin{itemize}
						            \item Often infeasible.
					            \end{itemize}
					      \item Fill in automatically with:
					            \begin{itemize}
						            \item A global constant.
						            \item The attribute mean.
						            \item The class mean.
						            \item The most probable value.
					            \end{itemize}
				      \end{itemize}
			\end{itemize}
		\end{column}

		\begin{column}{0.45\textwidth}
			\centering

			\vspace*{1cm}

			\scalebox{0.8}{
				\begin{tabular}{|c|c|}
					\hline
					\textbf{Mat. Nr.}                 & \textbf{Age}                      \\ \hline
					12345678                          & 23                                \\ \hline
					23061995                          & 25                                \\ \hline
					21241992                          & \only<2->{\cellcolor{faugray!50}} \\ \hline
					\only<2->{\cellcolor{faugray!50}} & 23                                \\ \hline
					25052025                          & 21                                \\ \hline
					14912780                          & 24                                \\ \hline
				\end{tabular}
			}

		\end{column}

	\end{columns}
\end{frame}

\begin{frame}{Dirty Data: Inconsistencies}
	\begin{columns}
		\begin{column}{0.45\textwidth}
			\begin{itemize}
				\item \textbf{Potential reasons:}
				      \begin{itemize}
					      \item Merging of data from different sources.
					      \item Missing conventions.
					      \item Human error.
					      \item etc.
				      \end{itemize}
				\item \textbf{Potential solutions:}
				      \begin{itemize}
					      \item Manual data cleaning.
					      \item (Semi-)Automatic data cleaning.
					            \begin{itemize}
						            \item Most often common inconsistencies can be detected and solved via rule based approaches.
					            \end{itemize}
				      \end{itemize}
			\end{itemize}
		\end{column}

		\begin{column}{0.45\textwidth}
			\centering

			\vspace*{1cm}

			\scalebox{0.8}{
				\begin{tabular}{|c|c|}
					\hline
					\textbf{Applicant}                        & \textbf{Grade}                       \\ \hline
					124                                       & 1.0                                  \\ \hline
					\only<2->{\cellcolor{faugray!50}} Michael & 2.3                                  \\ \hline
					134                                       & 3.7                                  \\ \hline
					323                                       & \only<2->{\cellcolor{faugray!50}} A- \\ \hline
					174                                       & 2.0                                  \\ \hline
					123                                       & 1.6                                  \\ \hline
				\end{tabular}
			}

		\end{column}

	\end{columns}
\end{frame}


\begin{frame}{Dirty Data: Errors}
	\begin{columns}
		\begin{column}{0.45\textwidth}
			\begin{itemize}
				\item \textbf{Potential reasons:}
				      \begin{itemize}
					      \item Malfunctions.
					      \item Transmission errors.
					      \item Human error.
					      \item etc.
				      \end{itemize}
				\item \textbf{Potential solutions:}
				      \begin{itemize}
					      \item Ignore the tuple.
					      \item Manual data cleaning.
					            \begin{itemize}
						            \item A subject matter expert (SME) is often needed to
						                  identify the errors.
					            \end{itemize}
					      \item (Semi-)Automatic data cleaning.
					            \begin{itemize}
						            \item Errors are often highly case dependent and therefore there is no general solution.
					            \end{itemize}
				      \end{itemize}
			\end{itemize}
		\end{column}

		\begin{column}{0.45\textwidth}
			\centering

			\vspace*{1cm}

			\scalebox{0.8}{
				\begin{tabular}{|c|c|}
					\hline
					\textbf{Module} & \textbf{ECTS}                      \\ \hline
					EADEIS          & 5                                  \\ \hline
					MoL             & 5                                  \\ \hline
					DL              & 5                                  \\ \hline
					EDB             & 7.5                                \\ \hline
					KDDmUe          & \only<2->{\cellcolor{faugray!50}}6 \\ \hline
					POIS            & 5                                  \\ \hline
				\end{tabular}
			}
		\end{column}
	\end{columns}
\end{frame}

\begin{frame}{Dirty Data: Noise}
	\begin{columns}
		\begin{column}{0.45\textwidth}
			\begin{itemize}
				\item \textbf{Potential reasons:}
				      \begin{itemize}
					      \item Small sensor inaccuracies.
					      \item Transmission errors.
					      \item etc.
				      \end{itemize}
				\item \textbf{Potential solutions:}
				      \begin{itemize}
					      \item Data smoothing by:
					            \begin{itemize}
						            \item Binning.
						            \item Regression.
						            \item Clustering.
						            \item etc.
					            \end{itemize}
				      \end{itemize}
			\end{itemize}
		\end{column}

		\begin{column}{0.45\textwidth}
			\centering

			\vspace*{1cm}

			\scalebox{0.8}{
				\begin{tabular}{|c|c|}
					\hline
					\textbf{Time} & \textbf{Temperature}                      \\ \hline
					08:01         & \only<2->{\cellcolor{faugray!50}}14.123°C \\ \hline
					08:02         & \only<2->{\cellcolor{faugray!50}}14.153°C \\ \hline
					08:03         & \only<2->{\cellcolor{faugray!50}}14.163°C \\ \hline
					08:04         & \only<2->{\cellcolor{faugray!50}}14.723°C \\ \hline
					08:05         & \only<2->{\cellcolor{faugray!50}}14.126°C \\ \hline
					08:06         & \only<2->{\cellcolor{faugray!50}}14.463°C \\ \hline
				\end{tabular}
			}

		\end{column}
	\end{columns}

	\begin{columns}
		\begin{column}{0.9\textwidth}
			\centering

			\only<3->{
				\begin{block}{Errors $\Longleftrightarrow$ Noise}
					\begin{itemize}
						\item Noise can be referred to as a special type of error.
						\item Not every error is noise!
					\end{itemize}
				\end{block}
			}
		\end{column}
	\end{columns}
\end{frame}

\begin{frame}{Dirty Data: Outliers}
	\begin{columns}
		\begin{column}{0.45\textwidth}
			\begin{itemize}
				\item \textbf{Potential reasons:}
				      \begin{itemize}
					      \item Errors.
					      \item Very rare events.
				      \end{itemize}
				\item \textbf{Potential solutions:}
				      \begin{itemize}
					      \item If an error, treat them as one.
					      \item If a rare event, the outlier is interesting and can be
					            used for further analysis.
				      \end{itemize}
			\end{itemize}
		\end{column}

		\begin{column}{0.45\textwidth}
			\centering

			\vspace*{1cm}

			\scalebox{0.8}{
				\begin{tabular}{|c|c|}
					\hline
					\textbf{Year} & \textbf{Max. Temp.}                   \\ \hline
					2026          & 32°C                                  \\ \hline
					2027          & 34°C                                  \\ \hline
					2028          & 33°C                                  \\ \hline
					2029          & 35°C                                  \\ \hline
					2030          & \only<2->{\cellcolor{faugray!50}}61°C \\ \hline
					2031          & 36°C                                  \\ \hline
				\end{tabular}
			}

		\end{column}
	\end{columns}

	\begin{columns}
		\begin{column}{0.9\textwidth}
			\centering

			\only<3->{
				\begin{block}{Errors $\Longleftrightarrow$ Outliers}
					\begin{itemize}
						\item Outliers might indicate errors.
						\item Not every outlier is an error!
					\end{itemize}
				\end{block}
			}
		\end{column}
	\end{columns}

\end{frame}

\begin{frame}{Data Cleaning as a Process (I)}
	\begin{itemize}
		\item \textbf{Data discrepancy detection:}
		      \begin{itemize}
			      \item Use \textbf{\color{airforceblue}metadata} (e.g. domain,
			            range, dependency, distribution).
			      \item Check field overloading.
			      \item Check uniqueness rule, consecutive rule and null rule.
			      \item Use commercial tools:
			            \begin{itemize}
				            \item \textbf{\color{airforceblue}Data scrubbing:} use simple
				                  domain knowledge (e.g. postal code, spell-check) to detect
				                  errors and make corrections.
				            \item \textbf{\color{airforceblue}Data auditing:} by analyzing
				                  data to discover rules and relationships to detect violators
				                  (e.g. correlation and clustering to find outliers).
			            \end{itemize}
		      \end{itemize}
	\end{itemize}
\end{frame}

\begin{frame}{Data Cleaning as a Process (II)}
	\begin{itemize}
		\item \textbf{Data migration and integration:}
		      \begin{itemize}
			      \item Data-migration tools: allow transformations to be
			            specified.
			      \item ETL (Extraction/Transformation/Loading) tools: allow
			            users to specify transformations through a graphical user
			            interface.
		      \end{itemize}
		\item \textbf{Integration of the two processes.}
		      \begin{itemize}
			      \item Iterative and interactive (e.g. the Potter's Wheel tool).
		      \end{itemize}
	\end{itemize}
\end{frame}
