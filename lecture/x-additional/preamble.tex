% Set institute and date 
\institute[CS6]{Chair of Computer Science 6 (Data Management), Friedrich-Alexander-University Erlangen-N\"urnberg}
\date[SS\the\year{}]{Summer semester \the\year{}}

% Configure the bibliography
\defbibheading{bibliography}{}
\addbibresource{references.bib}

% Define additional colors 
\definecolor{airforceblue}{rgb}{0.36, 0.54, 0.66}
\definecolor{ForestGreen}{rgb}{0.34, 0.139, 0.34}

% Configure the template
\setbeamercovered{transparent}
\setbeamertemplate{section in toc}[sections numbered]
\setbeamertemplate{section page}{%
	\begingroup
	\begin{beamercolorbox}[sep=10pt,center,rounded=true,shadow=true]{section title}
		\usebeamerfont{section title}\thesection~\insertsection\par
	\end{beamercolorbox}
	\endgroup
}
\setlength{\skip\footins}{0.2cm}
\setlength{\footnotesep}{0.1cm}

% Configure the formatting of listings
\lstset{%
	language=Python,
	tabsize=2,
	basicstyle=\tt,
	keywordstyle=\color{blue},
	commentstyle=\color{green!50!black},
	stringstyle=\color{red},
	numbers=left,
	numbersep=0.5em,
	xleftmargin=1em,
	numberstyle=\tt
}

% Add tikz and pgfplots libraries
\usetikzlibrary{arrows,decorations.pathmorphing,backgrounds,fit,positioning,shapes.symbols,chains,intersections,snakes,positioning,matrix,mindmap,shapes.multipart,shapes,calc,shapes.geometric,shadows,shadows.blur}
\usepgfplotslibrary{groupplots}

% Define pgfplotsset
\pgfplotsset{height=4cm,width=8cm,compat=1.14}

% Define tikz sets 
\tikzset{
	every overlay node/.style={
			anchor=north west, inner sep=0pt,
		},
}
\tikzset{
	thick,
	>=latex,
	every edge/.style={draw=gray, thick, >=latex},
	vertex/.style = {
			circle,
			fill            = black,
			outer sep = 2pt,
			inner sep = 1pt,
		}
}
\tikzset{level 1/.append style={sibling angle=50,level distance = 165mm}}
\tikzset{level 2/.append style={sibling angle=20,level distance = 45mm}}
\tikzset{every node/.append style={scale=1}}
\tikzset{
	vertex/.style = {
			circle,
			fill            = black,
			outer sep = 2pt,
			inner sep = 1pt,
		}
}
\tikzset{
	mynode/.style={
			draw,
			thick,
			anchor=south west,
			minimum width=2cm,
			minimum height=1.3cm,
			align=center,
			inner sep=0.2cm,
			outer sep=0,
			rectangle split,
			rectangle split parts=2,
			rectangle split draw splits=false},
	reverseclip/.style={
			insert path={(current page.north east) --
					(current page.south east) --
					(current page.south west) --
					(current page.north west) --
					(current page.north east)}
		}
}
\tikzset{basic/.style={
			draw,
			rectangle split,
			rectangle split parts=2,
			rectangle split part fill={blue!20,white},
			minimum width=2.5cm,
			text width=2cm,
			align=left,
			font=\itshape
		},
	Diamond/.style={ diamond,
			draw,
			shape aspect=2,
			inner sep = 2pt,
			text centered,
			fill=blue!10!white,
			font=\itshape
		}
}

% Define tikzoverlay
% Usage:
% \tikzoverlay at (-1cm,-5cm) {content};
% or
% \tikzoverlay[text width=5cm] at (-1cm,-5cm) {content};
\def\tikzoverlay{%
	\tikz[remember picture, overlay]\node[every overlay node]
}%

% Define additional math operators
\DeclareMathOperator*{\argmax}{arg\,max}
\DeclareMathOperator*{\argmin}{arg\,min}

% Define pgfmath functions
\pgfmathdeclarefunction{gauss}{2}{%
	\pgfmathparse{1/(#2*sqrt(2*pi))*exp(-((x-#1)^2)/(2*#2^2))}%
}

% Define additional commands
\newcommand*{\fullref}[1]{\underline{\hyperref[{#1}]{\cref{#1} (\nameref*{#1})}}}
\newcommand{\tikzmark}[1]{\tikz[remember picture] \node[coordinate] (#1) {#1};}
\newcommand{\plots}{0.611201}
\newcommand{\plotm}{2.19882}
\newcommand{\MaxNumberX}{3}
\newcommand{\MaxNumberY}{5}
