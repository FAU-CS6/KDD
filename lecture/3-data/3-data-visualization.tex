\section{Data Visualization}

\begin{frame}{Data Visualization}
	\textbf{Why visualize data?}
	\begin{itemize}
		\item \textbf{Gain insight} into an information space by mapping data into graphical primitives.
		\item \textbf{Provide qualitative overview} of large data sets.
		\item \textbf{Search} for patterns, trends, structure, irregularities, relationships among data.
		\item \textbf{Help find interesting regions and suitable parameters} for further quantitative analysis.
		\item \textbf{Provide a visual proof} of computer representations derived.
	\end{itemize}
\end{frame}

\begin{frame}{Categorization of Visualization Methods}
	\textbf{Visualization methods can be categorized into five groups:}
	\begin{enumerate}
		\item Pixel Oriented Visualization
		\item Geometric Visualization
		\item Icon Based Visualization
		\item Hierarchical Visualization
		\item Complex Data and Relations Visualization
	\end{enumerate}
\end{frame}

\subsection{Pixel Oriented Visualization}

\begin{frame}{Pixel Oriented Visualization}
	Very simple visualization techniques based on pixels.

	\vspace*{0.2cm}

	\textbf{General Methods:}
	\begin{itemize}
		\item Heat map.
		\item \textit{Circle segment diagram.}
	\end{itemize}
\end{frame}

\begin{frame}{Heat Map}
	\vspace*{-0.5em}

	\begin{itemize}
		\item For a data set of $m$ dimensions create $m$ windows on the screen, one for each dimension.
		\item The values in dimension $m$ of a record are mapped to $m$ pixels at the corresponding positions.
		\item The color/intensity of the pixels reflect the corresponding values.
	\end{itemize}

	\vspace*{1.5em}
	\begin{center}
		\includegraphics[width=9cm]{img/pixel.jpg}\\
		\vspace*{-0.5em}
		a) Income. \hspace{0.3cm} b) Credit limit. \hspace{0.1cm} c) Transaction volume. \hspace{0.2cm} d) Age.
	\end{center}

	\note{\textbf{Drawback:} Do not help us to understand the data distribution in a multidimensional space.}
\end{frame}

\subsection{Geometric Visualization}

\begin{frame}{Geometric Visualization}
	Visualization of geometric transformations and projections of data.

	\vspace*{0.2cm}

	\textbf{General Methods:}
	\begin{itemize}
		\item Scatter plot and scatter-plot matrices.
		\item Polar plot.
		\item \textit{Parallel coordinates.}
	\end{itemize}

	\textbf{Additional Methods:}
	\begin{itemize}
		\item Projection pursuit\footnote{\fullcite{friedman1974}}.
		      Finds a linear projection (one- or two-dimensional) that are ``highly revealing''.
		\item Prosection views\footnote{\fullcite{furnas1994}}.
		\item Hyperslice\footnote{\fullcite{wijk1993}}.
	\end{itemize}
\end{frame}

\begin{frame}{Scatter Plot Matrices}
	\begin{columns}[t]
		\begin{column}{0.45\textwidth}
			\begin{itemize}
				\item Compare the values of more than two dimensions at once.
				\item Each column is drawn against each other column (twice).
				\item Makes use of scatter plots for each comparison.
			\end{itemize}
		\end{column}
		\begin{column}{0.45\textwidth}
			\vspace*{-0.3cm}
			\begin{center}
				\scalebox{0.95}{
					\includegraphics[height=6.5cm]{img/scatterplot_matrix.pdf}
				}
			\end{center}
		\end{column}
	\end{columns}

\end{frame}

\begin{frame}{Polar Plot}
	\begin{columns}[t]
		\begin{column}{0.45\textwidth}
			\begin{itemize}
				\item Shows connections among multiple dimensions for each data record.
				\item Saves space.
				\item \textbf{Downside:} Can get crowded with too much data records.
			\end{itemize}
		\end{column}
		\begin{column}{0.45\textwidth}
			%\vspace{-5em}
			\begin{center}
				\includegraphics[width=0.8\textwidth,clip, trim=0cm 0cm 0cm 2.6cm]{img/dimension-plot.pdf}
			\end{center}
		\end{column}
	\end{columns}
\end{frame}

\subsection{Icon Based Visualization}

\begin{frame}{Icon Based Visualization}
	Visualization of the data values as features of icons:

	\vspace*{0.2cm}

	\begin{itemize}
		\item \textbf{General methods:}
		      \begin{itemize}
			      \item Stick figures.
			      \item \textit{Chernoff faces.}
		      \end{itemize}
		\item \textbf{General techniques:}
		      \begin{itemize}
			      \item Shape coding: \emph{Use shape to represent certain information encoding.}
			      \item Color icons: \emph{Use color icons to encode more information.}
			      \item Tile bars: \emph{Use small icons to represent the relevant feature vectors in document retrieval.}
		      \end{itemize}
	\end{itemize}
\end{frame}


\begin{frame}{Stick Figures}
	\centering
	\includegraphics[width=0.9\textwidth,clip, trim=0cm 0cm 0cm 2.5cm]{img/fau-enrolled-students.pdf}
\end{frame}


\subsection{Hierarchical Visualization}

\begin{frame}{Hierarchical Visualization }
	Visualization of the data using a hierarchical partitioning into subspaces:

	\vspace*{0.2cm}

	\begin{itemize}
		\item \textbf{General Methods:}
		      \begin{itemize}
			      \item Tree maps.
			      \item \textit{Cone trees.}
		      \end{itemize}
	\end{itemize}
\end{frame}


\begin{frame}{Tree Maps}
	\begin{columns}[t]
		\begin{column}{0.45\textwidth}
			\begin{itemize}
				\item Uses a hierarchical partitioning of the screen space depending on the importance of the data.
				\item In this example, there is only one level of hierarchy, but it can be extended to multiple levels (e.g. departments below the faculty level).
			\end{itemize}
		\end{column}
		\begin{column}{0.45\textwidth}
			\begin{center}
				\scalebox{0.75}{
					\begin{tikzpicture}[xscale = 0.9, yscale = 0.77,
							font=\sffamily,
							mystyle/.style={draw=white, very thick, text=white, font=\sffamily\bfseries},
						]
						\pie[square,
							style={mystyle},
							color={fauyellow, faured, faucyan, faugreen, faugray},
							text=legend,
						]{24/Humanities, 26/Business, 10/Medicine, 14/Sciences, 26/Engineering}
					\end{tikzpicture}
				}
			\end{center}
		\end{column}
	\end{columns}
\end{frame}

\subsection{Complex Data and Relations Visualization}

\begin{frame}{Complex Data and Relations Visualization}
	Visualizing non-numerical data:

	\vspace*{0.2cm}

	\begin{itemize}
		\item \textbf{General Methods:}
		      \begin{itemize}
			      \item Word cloud.
			      \item \textit{Networks.}
		      \end{itemize}
	\end{itemize}
\end{frame}

\begin{frame}{Word Cloud}
	\begin{columns}[t]
		\begin{column}{0.45\textwidth}
			\begin{itemize}
				\item The importance of a word is represented by its size or color.
				\item In this example, the size of the word is proportional to its frequency in a sample text.
			\end{itemize}
		\end{column}
		\begin{column}{0.45\textwidth}
			\begin{center}
				\scalebox{0.5}{
					\begin{tikzpicture}
						% How often the word occurs in the text
						% data - 71 times
						\node[text=faugray, font=\bfseries\fontsize{50}{60}\selectfont] at (0,0) {data};

						% attributes - 24 times
						\node[text=faugray, font=\bfseries\fontsize{32}{38}\selectfont] at (3,1.5) {attributes};

						% visualization - 22 times
						\node[text=faugray, font=\bfseries\fontsize{30}{36}\selectfont] at (-3.5,1) {visualization};

						% objects - 18 times
						\node[text=faugray, font=\bfseries\fontsize{28}{34}\selectfont] at (2.5,-1.5) {objects};

						% values - 15 times
						\node[text=faugray, font=\bfseries\fontsize{26}{31}\selectfont] at (-2,-1.5) {values};

						% distance - 13 times
						\node[text=faugray, font=\fontsize{24}{29}\selectfont] at (0,2.5) {distance};

						% object - 13 times
						\node[text=faugray, font=\fontsize{24}{29}\selectfont] at (0,-2.5) {object};

						% attribute - 12 times
						\node[text=faugray, font=\fontsize{22}{26}\selectfont] at (-4,-0.5) {attribute};

						% measure - 11 times
						\node[text=faugray, font=\fontsize{20}{24}\selectfont] at (4,0) {measure};

						% binary - 11 times
						\node[text=faugray, font=\fontsize{20}{24}\selectfont] at (3.5,3) {binary};
					\end{tikzpicture}
				}
			\end{center}
		\end{column}
	\end{columns}

\end{frame}
