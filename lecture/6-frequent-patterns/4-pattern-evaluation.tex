\section{Which Patterns are Interesting?}

\begin{frame}{Interesting Patterns - Example}
	\vspace*{-0.25cm}
	\begin{center}
		\begin{columns}[c]
			\begin{column}{0.2\textwidth}
				\begin{center}
					\scalebox{0.8}{
						\begin{tabular}{|c|c|}
							\hline
							\textbf{TID} & \textbf{Items}       \\\hline
							1            & Apple, Cereal        \\\hline
							2            & Bread, Mango, Cereal \\\hline
							3            & Cereal, Bread        \\\hline
							4            & Bread                \\\hline
							$\cdots$     & $\cdots$             \\\hline
						\end{tabular}
					}
				\end{center}
			\end{column}
			\visible<2->{
				\begin{column}{0.1\textwidth}
					\begin{center}
						$\rightarrow$
					\end{center}
				\end{column}
				\begin{column}{0.4\textwidth}
					\begin{center}
						\scalebox{0.8}{
							\begin{tabular}{|c|c|c|c|}
								\hline
								           & Bread & No Bread & Sum (Row) \\\hline
								Cereal     & 2000  & 1750     & 3750      \\\hline
								No Cereal  & 1000  & 250      & 1250      \\\hline
								Sum (Col.) & 3000  & 2000     & 5000      \\\hline
							\end{tabular}
						}
					\end{center}
				\end{column}
			}
		\end{columns}
	\end{center}

	\visible<3->{
		\begin{block}{Is Bread $\implies$ Cereal a good rule?}
			\begin{itemize}
				\visible<4->{
				\item \textbf{Support:} $2000/5000 = 40\%$.
				\item \textbf{Confidence:} $2000/3000 = 66.7\%$.
				      }
				      \visible<5->{
				\item \textbf{Problem:}
				      \begin{itemize}
					      \item Overall 75\% of transactions contain cereal. \\
					            $\Rightarrow$ If bread is present, the likelihood of cereal is actually lower (66.7\%).
					      \item \textbf{Misleading due to negative correlation.}
				      \end{itemize}
				      }
			\end{itemize}
		\end{block}
	}
\end{frame}

\begin{frame}{Interesting Patterns - Lift}
	\begin{itemize}
		\item \textbf{Idea:} Check association rules for positive correlation.
		\item \textbf{Interestingness measure:} Lift
		      \begin{align*}
			      \text{Lift}(A,B) & =\frac{P(A \cup B)}{P(A) P(B)}
		      \end{align*}
		      \begin{itemize}
			      \item \textbf{Independence:} $\text{Lift}(A,B)=1$.
			      \item \textbf{Positive correlation:} $\text{Lift}(A,B)>1$.
			      \item \textbf{Negative correlation:} $\text{Lift}(A,B)<1$.
		      \end{itemize}
		\item \textbf{In our example:} \\
		      \begin{align*}                       \\
			      \text{Lift}(\text{Bread},\text{Cereal}) & = \frac{2000/5000}{3000/5000 \cdot 3750 / 5000} = 0.89
		      \end{align*}
	\end{itemize}
\end{frame}

\begin{frame}{Interesting Patterns - $\chi^2$-Test}
	\begin{itemize}
		\item With a small trick, we can also use the \textbf{$\chi^2$-test}\footnote{Known from KDDmUe - Lecture 4: Data Preprocessing.}.
		\item \textbf{In our example:} \\
		      \begin{center}
			      \scalebox{0.8}{
				      \begin{tabular}{|c|c|c|c|}
					      \hline
					                 & Bread       & No Bread    & Sum (Row) \\\hline
					      Cereal     & 2000 (2250) & 1750 (1500) & 3750      \\\hline
					      No Cereal  & 1000 (750)  & 250 (500)   & 1250      \\\hline
					      Sum (Col.) & 3000        & 2000        & 5000      \\\hline
				      \end{tabular}
			      }
		      \end{center}
		      \begin{align*}
			      \chi^2 = \frac{(2000-2250)^2}{2250} + \frac{(1750-1500)^2}{1500} +
			      \frac{(1000-750)^2}{750} + \frac{(250-500)^2}{500} = 277.78
		      \end{align*}

		      \vspace*{-0.1cm}

		      \begin{block}{Interpretation}
			      \begin{itemize}
				      \item Lookup in the $\chi^2$-table with $df=(2-1)(2-1)=1$ and $\alpha=0.005$ gives $7.879$ \\
				            $\Rightarrow$ Bread and Cereal \textbf{are correlated}.
				      \item The observed value of Bread and Cereal is $2000$, while the expected value is $2250$. \\
				            $\Rightarrow$ Hints at a \textbf{negative} correlation.
			      \end{itemize}
		      \end{block}
	\end{itemize}
\end{frame}

\begin{frame}{Interesting Patterns - Null-Invariance}
	\begin{block}{Null-Transaction}
		\begin{itemize}
			\item A transaction that does not contain any of the itemsets being examined.
			\item Can outweigh the number of individual itemsets.
		\end{itemize}
	\end{block}

	\begin{block}{Null-Invariance}
		\begin{itemize}
			\item A measure is null-invariant, if its value is free from the influence of
			      null-transactions.
		\end{itemize}
	\end{block}


	\begin{itemize}
		\item We also want interestingness measures that are \textbf{null-invariant}.
		      \begin{itemize}
			      \item Lift and $\chi^2$ are \textbf{not} null-invariant.
			      \item We will take a closer look at the \textbf{Kulczynski measure} (Kulc) and the \textbf{Imbalance Ratio} (IR) as examples for null-invariant measures\footnote{The appendix also contains a list of 20+ measures (some null-invariant, some not). This list is not exam relevant.}.
		      \end{itemize}
	\end{itemize}
\end{frame}

\begin{frame}{Interesting Patterns - Kulczynski Measure}
	\begin{itemize}
		\item \textbf{Kulczynski Measure:}
		      \begin{align*}
			      \text{Kulc}(A,B) & = \frac{\text{sup}(AB)}{2}(\frac{1}{\text{sup}(A)} + \frac{1}{\text{sup}(B)})
		      \end{align*}
		      \begin{itemize}
			      \item \textbf{Interesting rule:} $\text{Kulc}(A,B)$ close to $0$ or $1$.
			      \item \textbf{(Potentially) not very interesting rule:} $\text{Kulc}(A,B)$ close to $0.5$.
		      \end{itemize}
		\item \textbf{In our example:} \\
		      \begin{align*}                       \\
			      \text{Kulc}(\text{Bread},\text{Cereal}) & = \frac{2000}{2}(\frac{1}{3000} + \frac{1}{3750}) = 0.6
		      \end{align*}
	\end{itemize}
\end{frame}

\begin{frame}{Interesting Patterns - Imbalance Ratio}
	\begin{itemize}
		\item \textbf{Imbalance Ratio:}
		      \begin{align*}
			      \text{IR}(A,B) & = \frac{|\text{sup}(A) - \text{sup}(B)|}{\text{sup}(A) + \text{sup}(B) - \text{sup}(A \cup B)}
		      \end{align*}
		      \begin{itemize}
			      \item \textbf{(Very) balanced rule:} $\text{IR}(A,B)$ close to $0$.
			      \item \textbf{(Very) unbalanced rule:}  $\text{IR}(A,B)$ close to $1$.
		      \end{itemize}
		\item \textbf{In our example:} \\
		      \begin{align*}                       \\
			      \text{IR}(\text{Bread},\text{Cereal}) & = \frac{|3000 - 3750|}{3000 + 3750 - 2000} \approx 0.16
		      \end{align*}
	\end{itemize}
\end{frame}
