\section{Partitioning Methods}

\begin{frame}{Partitioning Algorithms: Basic Concept}
	\begin{itemize}
		\item \textbf{Partitioning method:}
		      \begin{itemize}
			      \item Partition a database $D$ of $n$ objects $o_j, j \in \{1,
				            \ldots, n\}$ into a set of $k$-clusters $C_i$, $1 \leq i \leq k$
			            such that the sum of squared distances to $c_i$ is minimized
			            (where $c_i$ is the \textbf{\color{airforceblue}centroid} or
			            \textbf{\color{airforceblue}medoid} of cluster $C_i$):
			            \begin{align*}
				            \min \sum_{i=1}^{k} \sum_{j=1}^{n} d(o_j,c_i)^2.
			            \end{align*}
		      \end{itemize}
		\item \textbf{Given $k$, find a partition of $k$ clusters that
			      optimizes the chosen partitioning criterion.}
		      \begin{itemize}
			      \item Globally optimal: exhaustively enumerate all partitions.
			      \item Heuristic methods: k-means and k-medoids algorithms.
			      \item \textbf{\color{airforceblue}k-means} (MacQueen'67,
			            Lloyd'57/'82):
			            \begin{itemize}
				            \item Each cluster is represented by the center of the cluster.
			            \end{itemize}
			      \item \textbf{\color{airforceblue}k-medoids} or PAM (Partition
			            around medoids) (Kaufman \& Rousseeuw'87):
			            \begin{itemize}
				            \item Each cluster is represented by one of the objects in the
				                  cluster.
			            \end{itemize}
		      \end{itemize}
	\end{itemize}
\end{frame}

\begin{frame}{The $k$-means Clustering Method}
	\begin{itemize}
		\item \textbf{Given $k$, the $k$-means algorithm is implemented in four
			      steps:}
		      \begin{itemize}
			      \item[1.] Partition the database into $k$ non-empty subsets.
				      \begin{itemize}
					      \item E.g. the first $\frac{n}{k}$ objects, then the next
					            $\frac{n}{k}$ objects, $\ldots$
				      \end{itemize}
			      \item[2.] Compute the \textbf{\color{airforceblue}centroids} of the clusters of the
				      current partitioning.
				      \begin{itemize}
					      \item The centroid is the center, i.e. mean point, of the
					            cluster.
					      \item For each attribute (or dimension), calculate the average
					            value.
				      \end{itemize}
			      \item[3.] Assign each object to the cluster with the nearest
				      centroid.
				      \begin{itemize}
					      \item That is, for each object calculate distance to each of
					            the $k$\\
					            centroids and pick the one with the smallest distance.
				      \end{itemize}
			      \item[4.] If any object has changed its cluster, go back to step 2.
				      Otherwise stop.
		      \end{itemize}
		\item \textbf{Variant:}
		      \begin{itemize}
			      \item Start with arbitrarily chosen $k$ objects as initial
			            centroids in step $1$.
			      \item Continue with step $3$.
		      \end{itemize}
	\end{itemize}
\end{frame}

\begin{frame}{An Example of $k$-means Clustering}
	\centering
	\newcommand*{\xMin}{0}%
	\newcommand*{\xMax}{10}%
	\newcommand*{\yMin}{0}%
	\newcommand*{\yMax}{10}%
	\centering
	\begin{tikzpicture}[scale=0.23]
		\foreach \i in {\xMin,...,\xMax} {
				\draw [very thin,gray] (\i,\yMin) -- (\i,\yMax)  node [below] at
				(\i,\yMin) {\tiny$\i$};
			}
		\foreach \i in {\yMin,...,\yMax} {
				\draw [very thin,gray] (\xMin,\i) -- (\xMax,\i) node [left] at
				(\xMin,\i) {\tiny$\i$};
			}
		\node [black] at (6,5) {\Large\textbullet};
		\node [black] at (8,5) {\Large\textbullet};
		\node [black] at (4,3) {\Large\textbullet};
		\node [black] at (5,6) {\Large\textbullet};
		\node [black] at (6,2) {\Large\textbullet};
		\node [black] at (6,3) {\Large\textbullet};
		\node [black] at (2,2) {\Large\textbullet};
		\node [black] at (3,3) {\Large\textbullet};
		\node [black] at (4,4) {\Large\textbullet};
		\node [black] at (5,5) {\Large\textbullet};
		\node [black] at (6,6) {\Large\textbullet};
		\node [black] at (7,7) {\Large\textbullet};
		\node [black] at (8,7) {\Large\textbullet};
		\node [black] at (8,4) {\Large\textbullet};
		\draw[->] (12,3)--(17,3);
		\node at (14,8) {\scriptsize Arbitrarily};
		\node at (14,7) {\scriptsize partition};
		\node at (14,6) {\scriptsize objects into};
		\node at (14,5) {\scriptsize k groups};
		\node at (14,2) {\scriptsize for $k=2$};
	\end{tikzpicture}
	\begin{tikzpicture}[scale=0.23]
		\foreach \i in {\xMin,...,\xMax} {
				\draw [very thin,gray] (\i,\yMin) -- (\i,\yMax)  node [below] at
				(\i,\yMin) {\tiny$\i$};
			}
		\foreach \i in {\yMin,...,\yMax} {
				\draw [very thin,gray] (\xMin,\i) -- (\xMax,\i) node [left] at
				(\xMin,\i) {\tiny$\i$};
			}
		\node [red] at (6,5) {\Large\textbullet};
		\node [red] at (8,5) {\Large\textbullet};
		\node [black] at (4,3) {\Large\textbullet};
		\node [red] at (5,6) {\Large\textbullet};
		\node [black] at (6,2) {\Large\textbullet};
		\node [black] at (6,3) {\Large\textbullet};
		\node [black] at (2,2) {\Large\textbullet};
		\node [black] at (3,3) {\Large\textbullet};
		\node [black] at (4,4) {\Large\textbullet};
		\node [red] at (5,5) {\Large\textbullet};
		\node [red] at (6,6) {\Large\textbullet};
		\node [red] at (7,7) {\Large\textbullet};
		\node [red] at (8,7) {\Large\textbullet};
		\node [black] at (8,4) {\Large\textbullet};
		\draw[->] (12,3)--(17,3);
		\node at (14,8) {\scriptsize Calculate};
		\node at (14,7) {\scriptsize the cluster};
		\node at (14,6) {\scriptsize centroids};
	\end{tikzpicture}
	\begin{tikzpicture}[scale=0.23]
		\foreach \i in {\xMin,...,\xMax} {
				\draw [very thin,gray] (\i,\yMin) -- (\i,\yMax)  node [below] at
				(\i,\yMin) {\tiny$\i$};
			}
		\foreach \i in {\yMin,...,\yMax} {
				\draw [very thin,gray] (\xMin,\i) -- (\xMax,\i) node [left] at
				(\xMin,\i) {\tiny$\i$};
			}
		\node [red] at (6,5) {\Large\textbullet};
		\node [red] at (8,5) {\Large\textbullet};
		\node [black] at (4,3) {\Large\textbullet};
		\node [red] at (5,6) {\Large\textbullet};
		\node [black] at (6,2) {\Large\textbullet};
		\node [black] at (6,3) {\Large\textbullet};
		\node [black] at (2,2) {\Large\textbullet};
		\node [black] at (3,3) {\Large\textbullet};
		\node [black] at (4,4) {\Large\textbullet};
		\node [red] at (5,5) {\Large\textbullet};
		\node [red] at (6,6) {\Large\textbullet};
		\node [red] at (7,7) {\Large\textbullet};
		\node [red] at (8,7) {\Large\textbullet};
		\node [black] at (8,4) {\Large\textbullet};
		\node [orange] at (6.24,5.71) {\Large\textbullet};
		\node [blue] at (4.71,3) {\Large\textbullet};
		\draw[->] (12,3) to [out=30,in=30] (12,-1);
		\node at (15,8) {\scriptsize Check if};
		\node at (15,7) {\scriptsize objects};
		\node at (15,6) {\scriptsize have to be};
		\node at (15,5) {\scriptsize reasigned};
	\end{tikzpicture}\\
	\hspace{2.65cm}
	\begin{tikzpicture}[scale=0.23]
		\foreach \i in {\xMin,...,\xMax} {
				\draw [very thin,gray] (\i,\yMin) -- (\i,\yMax)  node [below] at
				(\i,\yMin) {\tiny$\i$};
			}
		\foreach \i in {\yMin,...,\yMax} {
				\draw [very thin,gray] (\xMin,\i) -- (\xMax,\i) node [left] at
				(\xMin,\i) {\tiny$\i$};
			}
		\node [red] at (6,5) {\Large\textbullet};
		\node [red] at (8,5) {\Large\textbullet};
		\node [black] at (4,3) {\Large\textbullet};
		\node [red] at (5,6) {\Large\textbullet};
		\node [black] at (6,2) {\Large\textbullet};
		\node [black] at (6,3) {\Large\textbullet};
		\node [black] at (2,2) {\Large\textbullet};
		\node [black] at (3,3) {\Large\textbullet};
		\node [black] at (4,4) {\Large\textbullet};
		\node [red] at (5,5) {\Large\textbullet};
		\node [red] at (6,6) {\Large\textbullet};
		\node [red] at (7,7) {\Large\textbullet};
		\node [red] at (8,7) {\Large\textbullet};
		\node [red] at (8,4) {\Large\textbullet};
		\draw[->] (17,5)--(12,5);
		\draw[->] (12,6)--(17,10);
		\node at (15,4) {\scriptsize Reasign};
		\node at (15,3) {\scriptsize objects};
	\end{tikzpicture}
	\begin{tikzpicture}[scale=0.23]
		\foreach \i in {\xMin,...,\xMax} {
				\draw [very thin,gray] (\i,\yMin) -- (\i,\yMax)  node [below] at
				(\i,\yMin) {\tiny$\i$};
			}
		\foreach \i in {\yMin,...,\yMax} {
				\draw [very thin,gray] (\xMin,\i) -- (\xMax,\i) node [left] at
				(\xMin,\i) {\tiny$\i$};
			}
		\node [red] at (6,5) {\Large\textbullet};
		\node [red] at (8,5) {\Large\textbullet};
		\node [black] at (4,3) {\Large\textbullet};
		\node [red] at (5,6) {\Large\textbullet};
		\node [black] at (6,2) {\Large\textbullet};
		\node [black] at (6,3) {\Large\textbullet};
		\node [black] at (2,2) {\Large\textbullet};
		\node [black] at (3,3) {\Large\textbullet};
		\node [black] at (4,4) {\Large\textbullet};
		\node [red] at (5,5) {\Large\textbullet};
		\node [red] at (6,6) {\Large\textbullet};
		\node [red] at (7,7) {\Large\textbullet};
		\node [red] at (8,7) {\Large\textbullet};
		\node [black] at (8,4) {\Large\textbullet};
		\draw[green] (8,4.13) circle (0.35cm);
		\node [orange] at (6.24,5.71) {\Large\textbullet};
		\node [blue] at (4.71,3) {\Large\textbullet};
	\end{tikzpicture}
\end{frame}

\begin{frame}{Comments on the $k$-means Method}
	\begin{itemize}
		\item \textbf{Strength:}
		      \begin{itemize}
			      \item Efficient: $\mathcal{O}(tkn)$, where $n$ is $\#$ objects, $k$
			            is $\#$ of clusters, and $t$ is the $\#$ of iterations. \\
			            Normally: $k$, $t \ll n$.
			      \item Comparing: PAM: $\mathcal{O}(k(n-k)^2)$, CLARA:
			            $\mathcal{O}(ks^2 + k(n-k))$.
			      \item Comment: Often terminates at a local optimum.
		      \end{itemize}
		\item \textbf{Weakness:}
		      \begin{itemize}
			      \item Applicable only to objects in a continuous n-dimensional
			            space.
			            \begin{itemize}
				            \item Using the $k$-modes method for categorical data.
				            \item In comparison, $k$-medoids can be applied to a wide range
				                  of data.
			            \end{itemize}
			      \item Need to specify $k$, the number of clusters, in advance.
			            \begin{itemize}
				            \item There are ways to automatically determine the best $k$
				                  (see Hastie et al., 2009).
			            \end{itemize}
			      \item Sensitive to noisy data and outliers.
			      \item Not suitable to discover clusters with non-convex shapes.
		      \end{itemize}
	\end{itemize}
\end{frame}

\begin{frame}{Variations of the $k$-means Method}
	\begin{itemize}
		\item \textbf{Most of the variants of the $k$-means differ in:}
		      \begin{itemize}
			      \item Selection of the initial $k$ subsets (or centroids).
			      \item Dissimilarity calculations.
			      \item Strategies to calculate cluster centroids.
		      \end{itemize}
		\item \textbf{Handling categorical data: $k$-modes:}
		      \begin{itemize}
			      \item Replacing centroids with modes.
			            \begin{itemize}
				            \item See Chapter 2: mode = value that occurs most frequently
				                  in the data.
			            \end{itemize}
			      \item Using new dissimilarity measures to deal with categorical
			            objects.
			      \item Using a frequency-based method to update modes of clusters.
			      \item A mixture of categorical and numerical data: k-prototype
			            method.
		      \end{itemize}
	\end{itemize}
\end{frame}

\begin{frame}{What is the Problem of the $k$-means Method?}
	\begin{itemize}
		\item \textbf{The $k$-means algorithm is {\color{airforceblue}sensitive
					      to outliers}!}
		      \begin{itemize}
			      \item Since an object with an extremely large value may
			            substantially \\
			            distort the distribution of the data.
		      \end{itemize}
		\item \textbf{$k$-medoids:}
		      \begin{itemize}
			      \item Instead of taking the mean value of the objects in a cluster
			            as a reference point,\\
			            medoids can be used, which is the most centrally located object in
			            a cluster.
		      \end{itemize}
	\end{itemize}
	\centering
	\newcommand*{\xMin}{0}%
	\newcommand*{\xMax}{10}%
	\newcommand*{\yMin}{0}%
	\newcommand*{\yMax}{10}%
	\begin{tikzpicture}[scale=0.3]
		\foreach \i in {\xMin,...,\xMax} {
				\draw [very thin,gray] (\i,\yMin) -- (\i,\yMax)  node [below] at
				(\i,\yMin) {\tiny$\i$};
			}
		\foreach \i in {\yMin,...,\yMax} {
				\draw [very thin,gray] (\xMin,\i) -- (\xMax,\i) node [left] at
				(\xMin,\i) {\tiny$\i$};
			}
		\node[color=orange] at (4,3) {\Huge\textbullet};
		\node[color=orange] at (3,6) {\Huge\textbullet};
		\node[color=orange] at (3,8) {\Huge\textbullet};
		\node[color=orange] at (4,7) {\Huge\textbullet};
		\node[color=orange] at (4,5) {\Huge\textbullet};
		\node[color=orange] at (5,5) {\Huge\textbullet};
		\node[color=blue] at (5,1) {\Huge\textbullet};
		\node[color=blue] at (7,3) {\Huge\textbullet};
		\node[color=blue] at (7,4) {\Huge\textbullet};
		\node[color=blue] at (8,5) {\Huge\textbullet};
		\draw[->] (12,5)--(17,5);
	\end{tikzpicture}
	\begin{tikzpicture}[scale=0.3]
		\foreach \i in {\xMin,...,\xMax} {
				\draw [very thin,gray] (\i,\yMin) -- (\i,\yMax)  node [below] at
				(\i,\yMin) {\tiny$\i$};
			}
		\foreach \i in {\yMin,...,\yMax} {
				\draw [very thin,gray] (\xMin,\i) -- (\xMax,\i) node [left] at
				(\xMin,\i) {\tiny$\i$};
			}
		\node[color=orange] at (4,3) {\Large\textbullet};
		\node[color=red] at (3,6) {\Huge\textbullet};
		\node[color=orange] at (3,8) {\Large\textbullet};
		\node[color=orange] at (4,7) {\Large\textbullet};
		\node[color=orange] at (4,5) {\Large\textbullet};
		\node[color=orange] at (5,5) {\Large\textbullet};
		\node[color=blue] at (5,1) {\Large\textbullet};
		\node[color=blue] at (7,3) {\Huge\textbullet};
		\node[color=blue] at (7,4) {\Large\textbullet};
		\node[color=blue] at (8,5) {\Large\textbullet};
	\end{tikzpicture}
\end{frame}

\begin{frame}{The $k$-medoids Clustering Method}
	\begin{itemize}
		\item \textbf{$k$-medoids clustering:}
		      \begin{itemize}
			      \item Find representative objects ( \textbf{\color{airforceblue}medoids}) in clusters.
			      \item \textbf{PAM} (Partitioning Around Medoids, Kaufmann \&
			            Rousseeuw, 1987):
			            \begin{itemize}
				            \item Starts from an initial set of $k$ medoids and iteratively
				                  replaces one of the medoids \\
				                  by one of the non-medoids, if it improves the total distance of
				                  the resulting clustering.
				            \item PAM works effectively for small data sets, but does not
				                  scale well for large\\
				                  data sets (due to the computational complexity).
			            \end{itemize}
		      \end{itemize}
		\item \textbf{Efficiency improvement on PAM:}
		      \begin{itemize}
			      \item CLARA (Kaufmann \& Rousseeuw, 1990): PAM on samples.
			      \item CLARANS (Ng \& Han, 1994): Randomized re-sampling.
		      \end{itemize}
	\end{itemize}
\end{frame}

\begin{frame}{PAM: A Typical $k$-medoids Algorithm}
	\newcommand*{\xMin}{0}%
	\newcommand*{\xMax}{10}%
	\newcommand*{\yMin}{0}%
	\newcommand*{\yMax}{10}%
	\centering
	\begin{tikzpicture}[scale=0.23]
		\foreach \i in {\xMin,...,\xMax} {
				\draw [very thin,gray] (\i,\yMin) -- (\i,\yMax)  node [below] at
				(\i,\yMin) {\tiny$\i$};
			}
		\foreach \i in {\yMin,...,\yMax} {
				\draw [very thin,gray] (\xMin,\i) -- (\xMax,\i) node [left] at
				(\xMin,\i) {\tiny$\i$};
			}
		\node[color=orange] at (2,6) {\Large\textbullet};
		\node[color=orange] at (3,4) {\Large\textbullet};
		\node[color=orange] at (3,8) {\Large\textbullet};
		\node[color=orange] at (4,7) {\Large\textbullet};
		\node[color=orange] at (6,2) {\Large\textbullet};
		\node[color=orange] at (6,3) {\Large\textbullet};
		\node[color=orange] at (7,3) {\Large\textbullet};
		\node[color=orange] at (7,4) {\Large\textbullet};
		\node[color=orange] at (7,6) {\Large\textbullet};
		\node[color=orange] at (8,5) {\Large\textbullet};
		\draw[->] (12,3)--(17,3);
		\node at (14,8) {\scriptsize Arbitrarily};
		\node at (14,7) {\scriptsize choose $k$};
		\node at (14,6) {\scriptsize object as};
		\node at (14,5) {\scriptsize initial};
		\node at (14,4) {\scriptsize medoids};
		\node at (14,2) {\scriptsize for $k=2$};
	\end{tikzpicture}
	\begin{tikzpicture}[scale=0.23]
		\foreach \i in {\xMin,...,\xMax} {
				\draw [very thin,gray] (\i,\yMin) -- (\i,\yMax)  node [below] at
				(\i,\yMin) {\tiny$\i$};
			}
		\foreach \i in {\yMin,...,\yMax} {
				\draw [very thin,gray] (\xMin,\i) -- (\xMax,\i) node [left] at
				(\xMin,\i) {\tiny$\i$};
			}
		\node[color=orange] at (2,6) {\Large\textbullet};
		\node[color=orange] at (3,4) {\Large\textbullet};
		\node[color=blue] at (3,8) {\Large\textbullet};
		\node[color=orange] at (4,7) {\Large\textbullet};
		\node[color=orange] at (6,2) {\Large\textbullet};
		\node[color=blue] at (6,3) {\Large\textbullet};
		\node[color=orange] at (7,3) {\Large\textbullet};
		\node[color=orange] at (7,4) {\Large\textbullet};
		\node[color=orange] at (7,6) {\Large\textbullet};
		\node[color=orange] at (8,5) {\Large\textbullet};
		\draw[->] (12,3)--(17,3);
		\node at (14,8) {\scriptsize Assign};
		\node at (14,7) {\scriptsize each};
		\node at (14,6) {\scriptsize remaining};
		\node at (14,5) {\scriptsize object to};
		\node at (14,4) {\scriptsize nearest medoid};
	\end{tikzpicture}
	\begin{tikzpicture}[scale=0.23]
		\foreach \i in {\xMin,...,\xMax} {
				\draw [very thin,gray] (\i,\yMin) -- (\i,\yMax)  node [below] at
				(\i,\yMin) {\tiny$\i$};
			}
		\foreach \i in {\yMin,...,\yMax} {
				\draw [very thin,gray] (\xMin,\i) -- (\xMax,\i) node [left] at
				(\xMin,\i) {\tiny$\i$};
			}
		\node[color=orange] at (2,6) {\Large\textbullet};
		\node[color=orange] at (3,4) {\Large\textbullet};
		\node[color=blue] at (3,8) {\Large\textbullet};
		\node[color=orange] at (4,7) {\Large\textbullet};
		\node[color=orange] at (6,2) {\Large\textbullet};
		\node[color=blue] at (6,3) {\Large\textbullet};
		\node[color=orange] at (7,3) {\Large\textbullet};
		\node[color=orange] at (7,4) {\Large\textbullet};
		\node[color=orange] at (7,6) {\Large\textbullet};
		\node[color=orange] at (8,5) {\Large\textbullet};
		\draw[dashed] (3,6) ellipse (2cm and 3cm);
		\draw[dashed] (7,4) ellipse (2cm and 3cm);
		\draw[->] (12,3) to [out=30,in=30] (12,-1);
		\node at (7,11) {\scriptsize Total cost = 21};
		\node at (15,8) {\scriptsize Randomly};
		\node at (15,7) {\scriptsize select};
		\node at (15,6) {\scriptsize nonmedoid};
		\node at (15,5) {\scriptsize object $o_\text{random}$};
	\end{tikzpicture}\\
	\hspace{1.2cm}
	\begin{tikzpicture}[scale=0.23]
		\foreach \i in {\xMin,...,\xMax} {
				\draw [very thin,gray] (\i,\yMin) -- (\i,\yMax)  node [below] at
				(\i,\yMin) {\tiny$\i$};
			}
		\foreach \i in {\yMin,...,\yMax} {
				\draw [very thin,gray] (\xMin,\i) -- (\xMax,\i) node [left] at
				(\xMin,\i) {\tiny$\i$};
			}
		\node[color=orange] at (2,6) {\Large\textbullet};
		\node[color=orange] at (3,4) {\Large\textbullet};
		\node[color=blue] at (3,8) {\Large\textbullet};
		\node[color=orange] at (4,7) {\Large\textbullet};
		\node[color=red] at (6,2) {\Large\textbullet};
		\node[color=blue] at (6,3) {\Large\textbullet};
		\node[color=orange] at (7,3) {\Large\textbullet};
		\node[color=green] at (7,4) {\Large\textbullet};
		\node[color=orange] at (7,6) {\Large\textbullet};
		\node[color=orange] at (8,5) {\Large\textbullet};
		\draw[->] (17,5)--(12,5);
		\draw[->] (12,6)--(17,10);
		\node at (15,4) {\scriptsize Compute};
		\node at (15,3) {\scriptsize total cost of};
		\node at (15,2) {\scriptsize swapping};
		\node at (-5,4) {\scriptsize Swapping $o$};
		\node at (-5,3) {\scriptsize and $o_\text{random}$};
		\node at (-5,2) {\scriptsize if quality is};
		\node at (-5,1) {\scriptsize improved};
		\node at (3.4,11) {\scriptsize Total cost = 25};
	\end{tikzpicture}
	\begin{tikzpicture}[scale=0.23]
		\foreach \i in {\xMin,...,\xMax} {
				\draw [very thin,gray] (\i,\yMin) -- (\i,\yMax)  node [below] at
				(\i,\yMin) {\tiny$\i$};
			}
		\foreach \i in {\yMin,...,\yMax} {
				\draw [very thin,gray] (\xMin,\i) -- (\xMax,\i) node [left] at
				(\xMin,\i) {\tiny$\i$};
			}
		\node[color=orange] at (2,6) {\Large\textbullet};
		\node[color=orange] at (3,4) {\Large\textbullet};
		\node[color=blue] at (3,8) {\Large\textbullet};
		\node[color=orange] at (4,7) {\Large\textbullet};
		\node[color=red] at (6,2) {\Large\textbullet};
		\node[color=blue] at (6,3) {\Large\textbullet};
		\node[color=orange] at (7,3) {\Large\textbullet};
		\node[color=orange] at (7,4) {\Large\textbullet};
		\node[color=orange] at (7,6) {\Large\textbullet};
		\node[color=orange] at (8,5) {\Large\textbullet};
	\end{tikzpicture}
\end{frame}

\begin{frame}{PAM (Partitioning Around Medoids)}
	\begin{itemize}
		\item \textbf{Use real objects to represent the clusters.}
		\item \textbf{Algorithm:}
		      \begin{itemize}
			      \item[1.] Arbitrarily choose $k$ objects as the initial mediods.
			      \item[2.] Repeat.
			      \item[3.] \hspace{0.2cm} Assign each remaining object to the
				      cluster with the nearest mediod $o_i$.
			      \item[4.] \hspace{0.2cm} Randomly select a non-medoid object $o_h$.
			      \item[5.] \hspace{0.2cm} Compute the total cost $\text{TC}_{ih}$ of
				      swapping $o_i$ with $o_h$.
			      \item[6.] \hspace{0.2cm} If $\text{TC}_{ih} < 0$ then swap $o_i$
				      with $o_h$ to form the new set of $k$ medoids.
			      \item[7.] Until no change.
		      \end{itemize}
		\item $\text{TC}_{ih} = \sum\limits_{j}C_{jih}$
		      \begin{itemize}
			      \item with $C_{jih}$ as the cost for object $o_j$ if $o_i$ is
			            swapped with $o_h$.
			      \item That is, distance to new medoid minus distance to old medoid.
		      \end{itemize}
	\end{itemize}
\end{frame}

\begin{frame}{PAM Clustering - Cases (I)}
	\begin{itemize}
		\item \textbf{Case 1:}
		      \begin{itemize}
			      \item $o_j$ currently belongs to medoid $o_i$. If $o_i$ is replaced
			            with $o_h$ as a medoid, and $o_j$ is closest to $o_h$, then $o_j$
			            is reassigned to $o_h$ (same cluster, different distance).
			            \begin{align*}
				            C_{jih} = d(o_j,o_h) - d(o_j,o_i).
			            \end{align*}
		      \end{itemize}
	\end{itemize}
	\newcommand*{\xMin}{0}%
	\newcommand*{\xMax}{10}%
	\newcommand*{\yMin}{0}%
	\newcommand*{\yMax}{10}%
	\centering
	\begin{tikzpicture}[scale=0.3]
		\foreach \i in {\xMin,...,\xMax} {
				\draw [very thin,gray] (\i,\yMin) -- (\i,\yMax)  node [below] at
				(\i,\yMin) {\tiny$\i$};
			}
		\foreach \i in {\yMin,...,\yMax} {
				\draw [very thin,gray] (\xMin,\i) -- (\xMax,\i) node [left] at
				(\xMin,\i) {\tiny$\i$};
			}
		\draw[thick] (7,6)--(6,4);
		\draw[thick] (7,6)--(7,4);
		\node[color=orange] at (3,4) {\Huge\textbullet};
		\node[color=orange] at (2,6) {\Huge\textbullet};
		\node[color=blue] at (3,8) {\Huge\textbullet};
		\node[color=orange] at (4,7) {\Huge\textbullet};
		\node[color=orange] at (6,2) {\Huge\textbullet};
		\node[color=green] at (6,4) {\Huge\textbullet};
		\node[color=orange] at (7,3) {\Huge\textbullet};
		\node[color=lime] at (7,4) {\Huge\textbullet};
		\node[color=brown] at (7,6) {\Huge\textbullet};
		\node[color=orange] at (8,5) {\Huge\textbullet};
		\node at (3,9) {$t$};
		\node at (7,7) {$j$};
		\node at (5.4,4.3) {$i$};
		\node at (7.3,4.7) {$h$};
		\draw[dashed] (3,6) ellipse (2cm and 3cm);
		\draw[dashed] (7,4) ellipse (2cm and 3cm);
	\end{tikzpicture}
\end{frame}

\begin{frame}{PAM Clustering - Cases (II)}
	\begin{itemize}
		\item \textbf{Case 2:}
		      \begin{itemize}
			      \item $o_j$ currently belongs to medoid $o_t$, $t \neq j$. If $o_i$
			            is replaced with $o_h$ as a medoid, and $o_j$ is still closest to
			            $o_t$, then the assignment does not change.
			            \begin{align*}
				            C_{jih} = d(o_j,o_h) - d(o_j,o_t) \geq 0.
			            \end{align*}
		      \end{itemize}
	\end{itemize}
	\newcommand*{\xMin}{0}%
	\newcommand*{\xMax}{10}%
	\newcommand*{\yMin}{0}%
	\newcommand*{\yMax}{10}%
	\centering
	\begin{tikzpicture}[scale=0.3]
		\foreach \i in {\xMin,...,\xMax} {
				\draw [very thin,gray] (\i,\yMin) -- (\i,\yMax)  node [below] at
				(\i,\yMin) {\tiny$\i$};
			}
		\foreach \i in {\yMin,...,\yMax} {
				\draw [very thin,gray] (\xMin,\i) -- (\xMax,\i) node [left] at
				(\xMin,\i) {\tiny$\i$};
			}
		\draw[thick] (4,9)--(6,4);
		\draw[thick] (4,9)--(7,4);
		\node[color=orange] at (3,4) {\Huge\textbullet};
		\node[color=orange] at (2,6) {\Huge\textbullet};
		\node[color=blue] at (3,8) {\Huge\textbullet};
		\node[color=orange] at (4,7) {\Huge\textbullet};
		\node[color=orange] at (6,2) {\Huge\textbullet};
		\node[color=green] at (6,4) {\Huge\textbullet};
		\node[color=orange] at (7,3) {\Huge\textbullet};
		\node[color=lime] at (7,4) {\Huge\textbullet};
		\node[color=brown] at (4,9) {\Huge\textbullet};
		\node[color=orange] at (8,5) {\Huge\textbullet};
		\node at (3,9) {$t$};
		\node at (4,10) {$j$};
		\node at (5.4,4.3) {$i$};
		\node at (7.3,4.7) {$h$};
		\draw[dashed] (3,7) ellipse (2cm and 4cm);
		\draw[dashed] (7,4) ellipse (2cm and 3cm);
	\end{tikzpicture}
\end{frame}

\begin{frame}{PAM Clustering - Cases (III)}
	\begin{itemize}
		\item \textbf{Case 3:}
		      \begin{itemize}
			      \item $o_j$ currently belongs to medoid $o_i$. If $o_i$ is replaced
			            with $o_h$ as a medoid, and $o_j$ is closest to medoid $o_t$ of one
			            of the other clusters, then $o_j$ is reassigned to $o_t$ (new
			            cluster, different distance).
			            \begin{align*}
				            C_{jih} = d(o_j,o_t) - d(o_j,o_h) < 0.
			            \end{align*}
		      \end{itemize}
	\end{itemize}
	\newcommand*{\xMin}{0}%
	\newcommand*{\xMax}{10}%
	\newcommand*{\yMin}{0}%
	\newcommand*{\yMax}{10}%
	\centering
	\begin{tikzpicture}[scale=0.3]
		\foreach \i in {\xMin,...,\xMax} {
				\draw [very thin,gray] (\i,\yMin) -- (\i,\yMax)  node [below] at
				(\i,\yMin) {\tiny$\i$};
			}
		\foreach \i in {\yMin,...,\yMax} {
				\draw [very thin,gray] (\xMin,\i) -- (\xMax,\i) node [left] at
				(\xMin,\i) {\tiny$\i$};
			}
		\draw[thick] (5,6)--(3,8);
		\draw[thick] (5,6)--(4,5);
		\node[color=orange] at (2,6) {\Huge\textbullet};
		\node[color=orange] at (3,6) {\Huge\textbullet};
		\node[color=brown] at (5,6) {\Huge\textbullet};
		\node[color=lime] at (3,8) {\Huge\textbullet};
		\node[color=orange] at (3,4) {\Huge\textbullet};
		\node[color=green] at (4,5) {\Huge\textbullet};
		\node[color=blue] at (7,4) {\Huge\textbullet};
		\node[color=orange] at (7,3) {\Huge\textbullet};
		\node[color=orange] at (7,2) {\Huge\textbullet};
		\node[color=orange] at (8,5) {\Huge\textbullet};
		\draw[dashed] (3.5,6) ellipse (2cm and 3cm);
		\draw[dashed] (7,4) ellipse (2cm and 3cm);
		\node at (2.3,8.7) {$h$};
		\node at (5,7) {$j$};
		\node at (3.5,5) {$i$};
		\node at (6.2,4.3) {$t$};
	\end{tikzpicture}
\end{frame}

\begin{frame}{PAM Clustering - Cases (IV)}
	\begin{itemize}
		\item \textbf{Case 4:}
		      \begin{itemize}
			      \item $o_j$ currently belongs to medoid $o_t$, $t \neq j$. If $o_i$
			            is replaced with $o_h$ as a medoid (from a different cluster!), and
			            $o_j$ is closest to $o_h$, then $o_j$ is reassigned to $o_h$ (new
			            cluster, different distance).
			            \begin{align*}
				            C_{jih} = d(o_j,o_h) - d(o_j,o_t) < 0.
			            \end{align*}
		      \end{itemize}
	\end{itemize}
	\newcommand*{\xMin}{0}%
	\newcommand*{\xMax}{10}%
	\newcommand*{\yMin}{0}%
	\newcommand*{\yMax}{10}%
	\centering
	\begin{tikzpicture}[scale=0.3]
		\foreach \i in {\xMin,...,\xMax} {
				\draw [very thin,gray] (\i,\yMin) -- (\i,\yMax)  node [below] at
				(\i,\yMin) {\tiny$\i$};
			}
		\foreach \i in {\yMin,...,\yMax} {
				\draw [very thin,gray] (\xMin,\i) -- (\xMax,\i) node [left] at
				(\xMin,\i) {\tiny$\i$};
			}
		\draw[thick] (8,4)--(4,5);
		\draw[thick] (8,4)--(7,4);
		\node[color=orange] at (2,6) {\Huge\textbullet};
		\node[color=orange] at (3,8) {\Huge\textbullet};
		\node[color=orange] at (4,7) {\Huge\textbullet};
		\node[color=orange] at (3,4) {\Huge\textbullet};
		\node[color=orange] at (5,1) {\Huge\textbullet};
		\node[color=lime] at (7,4) {\Huge\textbullet};
		\node[color=brown] at (8,4) {\Huge\textbullet};
		\node[color=orange] at (8,5) {\Huge\textbullet};
		\node[color=green] at (4,5) {\Huge\textbullet};
		\node[color=blue] at (7,2) {\Huge\textbullet};
		\node at (3.3,5.3) {$i$};
		\node at (6.3,4.3) {$h$};
		\node at (6.3,2.3) {$t$};
		\node at (8.5,4) {$j$};
		\draw[dashed] (3.5,6) ellipse (2cm and 3cm);
		\draw[dashed] (7,4) ellipse (2cm and 3cm);
	\end{tikzpicture}
\end{frame}

\begin{frame}{CLARA (Clustering Large Applications)}
	\begin{itemize}
		\item \textbf{(Kaufmann and Rousseeuw, 1990)}
		\item \textbf{Draws multiple samples of the data set, applies PAM on
			      each sample,\\
			      and gives the best clustering as the output.}
		\item \textbf{Strength:}
		      \begin{itemize}
			      \item Deals with larger data sets than PAM.
		      \end{itemize}
		\item \textbf{Weakness:}
		      \begin{itemize}
			      \item Efficiency depends on the sample size.
			      \item A good clustering based on samples will not necessarily
			            represent \\
			            a good clustering of the whole data set if the sample is biased.
		      \end{itemize}
	\end{itemize}
\end{frame}

\begin{frame}{CLARANS ("Randomized" CLARA)}
	\begin{itemize}
		\item \textbf{A Clustering Algorithm based on Randomized Search} (Ng \&
		      Han, 1994)
		\item \textbf{Samples:}
		      \begin{itemize}
			      \item Drawn dynamically with some randomness in each step of the
			            search.
		      \end{itemize}
		\item \textbf{Clustering process:}
		      \begin{itemize}
			      \item Can be presented as searching a graph where each node is a
			            potential solution,\\
			            that is, a set of $k$ medoids.
		      \end{itemize}
		\item \textbf{If local optimum found,}
		      \begin{itemize}
			      \item start with new randomly selected node in search for a new
			            local optimum.
		      \end{itemize}
		\item \textbf{More efficient and scalable than both PAM and CLARA.}
		\item \textbf{Focusing techniques and spatial access structures may
			      further improve its performance.} (Ester et al., 1995)
	\end{itemize}
\end{frame}
