\section{Grid-based Methods}

\begin{frame}{Grid-based Clustering Method}
	\begin{itemize}
		\item \textbf{Using multi-resolution grid data structure.}
		\item \textbf{Several interesting methods.}
		      \begin{itemize}
			      \item STING (a STatistical INformation Grid approach) (Wang, Yang
			            \& Muntz, VLDB'97).
			      \item WaveCluster (Sheikholeslami, Chatterjee \& Zhang, VLDB'98):
			            \begin{itemize}
				            \item A multi-resolution clustering approach using wavelet
				                  method.
				            \item Not shown here.
			            \end{itemize}
			      \item CLIQUE (Agrawal et al., SIGMOD'98):
			            \begin{itemize}
				            \item Both grid-based and subspace clustering.
			            \end{itemize}
		      \end{itemize}
	\end{itemize}
\end{frame}

\begin{frame}{STING: A Statistical Information Grid Approach}
	\begin{itemize}
		\item \textbf{The spatial area is divided into rectangular cells.}
		\item \textbf{There are several levels of cells corresponding to
			      different levels of resolution.}
		\item (Wang, Yang \& Muntz, VLDB'97)
	\end{itemize}
	\vspace{0.5cm}
	\centering
	\includegraphics[width=9cm]{img/layers.pdf}
\end{frame}

\begin{frame}{The STING Clustering Method (I)}
	\begin{itemize}
		\item \textbf{Each cell at a higher level:}
		      \begin{itemize}
			      \item Partitioned into a number of smaller cells at next lower
			            level.
		      \end{itemize}
		\item \textbf{Statistical info of each cell:}
		      \begin{itemize}
			      \item Calculated and stored beforehand and used to answer queries.
			      \item Count, plus for each attribute: mean, standard dev, min, max
			            and \\
			            type of distribution: normal, uniform, etc.
		      \end{itemize}
		\item \textbf{Parameters of higher-level cells:}
		      \begin{itemize}
			      \item Can easily be calculated from parameters of lower-level cells.
		      \end{itemize}
		\item \textbf{Use a top-down approach:}
		      \begin{itemize}
			      \item To answer spatial data queries (or: cluster definitions).
		      \end{itemize}
		\item \textbf{Start from a pre-selected layer:}
		      \begin{itemize}
			      \item Typically with a small number of cells.
		      \end{itemize}
		\item \textbf{For each cell at the current level:}
		      \begin{itemize}
			      \item Compute the confidence interval.
		      \end{itemize}
	\end{itemize}
\end{frame}

\begin{frame}{The STING Clustering Method (II)}
	\begin{itemize}
		\item \textbf{Cells labeled relevant or not relevant.}
		      \begin{itemize}
			      \item At the specified confidence level.
			      \item Irrelevant cells removed from further consideration.
		      \end{itemize}
		\item \textbf{When finished examining the current layer, proceed to the
			      next lower level.}
		      \begin{itemize}
			      \item Only look at cells that are children of relevant cells.
		      \end{itemize}
		\item \textbf{Repeat this process until bottom layer is reached.}
		\item \textbf{Find regions (clusters) that satisfy the
				      {\color{airforceblue}density} specified.}
		      \begin{itemize}
			      \item Breadth-first search, at bottom layer.
			      \item Examine cells within a certain distance from center of
			            current cell (often just the neighbors).
			      \item If average density within this small area is greater than
			            density specified, mark area and put relevant cells just examined
			            into queue.
			      \item Examine next cell from queue and repeat procedure, until end
			            of queue.
			      \item Then one region has been identified.
		      \end{itemize}
	\end{itemize}
\end{frame}

\begin{frame}{STING Algorithm: Analysis}
	\begin{itemize}
		\item \textbf{Advantages:}
		      \begin{itemize}
			      \item Query-independent, easy to parallelize, incremental update.
			      \item $\mathcal{O}(K)$, where $K$ is the number of grid cells at
			            the lowest level.
		      \end{itemize}
		\item \textbf{Disadvantages:}
		      \begin{itemize}
			      \item All the cluster boundaries are either horizontal or vertical,
			            \\
			            and no diagonal boundary is detected.
		      \end{itemize}
	\end{itemize}
\end{frame}

\begin{frame}{CLIQUE (CLustering In QUEst)}
	\begin{itemize}
		\item \textbf{Using {\color{airforceblue}subspaces} (lower-dimensional)
			      of a high-dimensional data space that allow better clustering than
			      original space:}
		      \begin{itemize}
			      \item (Agrawal, Gehrke, Gunopulos \& Raghavan, SIGMOD'98).
		      \end{itemize}
		\item \textbf{CLIQUE can be considered as both density-based and
			      grid-based.}
		      \begin{itemize}
			      \item Partitions each dimension into non-overlapping intervals, \\
			            thereby partitioning the entire data space into \textbf{cells}.
			      \item Uses density threshold to identify \textbf{dense} cells.
			            \begin{itemize}
				            \item A cell is dense, if the number of data points mapped to
				                  it exceeds the density threshold.
			            \end{itemize}
		      \end{itemize}
	\end{itemize}
\end{frame}

\begin{frame}{CLIQUE: The Major Steps (I)}
	\begin{itemize}
		\item \textbf{Monotonicity of dense cells w.r.t. dimensionality:}
		      \begin{itemize}
			      \item Based on the a priori property used in frequent-pattern and
			            association-rule mining (see Chapter 5).
			      \item $k$-dimensional cell $c$ can have at least $l$ points only,
			            if every $(k-1)$-dimensional projection of $c$ (which is a cell in
			            a $(k-1)$-dimensional subspace) has at least $l$ points, too.
		      \end{itemize}
		\item \textbf{Clustering step 1:}
		      \begin{itemize}
			      \item Partition each dimension into intervals, identify intervals
			            containing at least $l$ points.
			      \item Iteratively join $k$-dimensional dense cells $c_1$ and $c_2$
			            in subspaces $(D_{i1}, \ldots, D_{ik})$ and $(D_{j1}, \ldots,
				            D_{jk})$ with $D_{i1} = D_{j1}$ and $D_{i2} = D_{j2}$ and $\ldots
				            D_{i(k-1)} = D_{j(k-1)}$ and $c_1$ and $c_2$ share the same
			            intervals to those dimensions.
			      \item New $(k+1)$-dimensional candidate cell $c$ in space $(D_{i1},
				            \ldots, D_{ik}, D_{jk})$ tested for density.
		      \end{itemize}
	\end{itemize}
\end{frame}

\begin{frame}{CLIQUE: The Major Steps (II)}
	\begin{itemize}
		\item \textbf{Clustering step 1 (cont.):}
		      \begin{itemize}
			      \item Iteration terminates when no more candidate cells can be
			            generated or no candidate cells are dense.
		      \end{itemize}
		\item \textbf{Clustering step 2:}
		      \begin{itemize}
			      \item Use dense cells in each subspace to assemble clusters.
			      \item Apply Minimum Description Length (MDL) principle to use the
			            maximal regions to cover connected dense cells.
			            \begin{itemize}
				            \item \textbf{\color{airforceblue}Maximal region}: hyper
				                  rectangle where every cell falling into the regions is dense,
				                  and region cannot be extended further in any dimension.
			            \end{itemize}
			      \item Simple greedy approach:
			            \begin{itemize}
				            \item Start with arbitrary dense cell.
				            \item Find maximum region covering that cell.
				            \item Work on remaining dense cells that have not yet been
				                  covered.
			            \end{itemize}
		      \end{itemize}
	\end{itemize}
\end{frame}

\begin{frame}{CLIQUE: Example}
	\centering
	\includegraphics[width=9cm]{img/clique.pdf}
\end{frame}

\begin{frame}{Strength and Weakness of CLIQUE}
	\begin{itemize}
		\item \textbf{Strength:}
		      \begin{itemize}
			      \item Automatically finds subspaces of the highest dimensionality \\
			            such that high-density clusters exist in those subspaces.
			      \item Insensitive to the order of records in input and does not
			            presume \\
			            any canonical data distribution.
			      \item Scales linearly with the size of input and has good
			            scalability \\
			            as the number of dimensions in the data is increased.
		      \end{itemize}
		\item \textbf{Weaknesses:}
		      \begin{itemize}
			      \item Dependent on proper grid size and density threshold.
			      \item Accuracy of clustering result may be degraded at the expense
			            \\
			            of simplicity of the method.
		      \end{itemize}
	\end{itemize}
\end{frame}
