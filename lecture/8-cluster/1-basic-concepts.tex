\section{Basic Concepts}

\begin{frame}{What is Cluster Analysis?}
	\begin{itemize}
		\item \textbf{{\color{airforceblue}Cluster}: A collection of data 
		objects within a larger set that are:}
		\begin{itemize}
			\item {\color{airforceblue}Similar (or related)} to one another 
			within the same group and,
			\item dissimilar (or unrelated) to the objects outside the group.
		\end{itemize}
		\item \textbf{{\color{airforceblue}Cluster analysis} (or clustering, 
		data segmentation, $\ldots$):}
		\begin{itemize}
			\item {\color{airforceblue}Define similarities} among data based on 
			the characteristics found in the data (input from user!).
			\item Group similar data objects into clusters.
		\end{itemize}
		\item \textbf{Unsupervised learning:}
		\begin{itemize}
			\item No predefined classes.
			\item I.e., learning by observation (vs. learning by examples: 
			supervised).
		\end{itemize}
		\item \textbf{Typical applications:}
		\begin{itemize}
			\item As a stand-alone tool to get insight into data distribution.
			\item As a preprocessing step for other algorithms.
		\end{itemize}
	\end{itemize}
\end{frame}

\begin{frame}{Clustering for Data Understanding and Applications}
	\begin{itemize}
		\item \textbf{Biology:}
		\begin{itemize}
			\item Taxonomy of living things: kingdom, phylum, class, order, 
			family, genus, and species.
		\end{itemize}
		\item \textbf{Land use:}
		\begin{itemize}
			\item Identification of areas of similar land use in an 
			earth-observation database.
		\end{itemize}
		\item \textbf{Marketing:}
		\begin{itemize}
			\item Help marketers discover distinct groups in their customer 
			bases, and then use this knowledge to develop targeted marketing 
			programs.
		\end{itemize}
		\item \textbf{City planning:}
		\begin{itemize}
			\item Identifying groups of houses according to their house type, 
			value, and geographical location.
		\end{itemize}
		\item \textbf{Earthquake studies:}
		\begin{itemize}
			\item Observed earthquake epicenters should be clustered along 
			continent faults.
		\end{itemize}
		\item \textbf{Climate:}
		\begin{itemize}
			\item Understanding earth climate, find patterns of atmosphere and 
			ocean.
		\end{itemize}
	\end{itemize}
\end{frame}

\begin{frame}{Quality: What is Good Clustering?}
	\begin{itemize}
		\item \textbf{A good clustering method will produce high-quality 
		clusters.}
		\begin{itemize}
			\item \textbf{\color{airforceblue}High intra-class similarity:}
			\begin{itemize}
				\item Cohesive within clusters.
			\end{itemize}
			\item \textbf{\color{airforceblue}Low inter-class similarity:}
			\begin{itemize}
				\item Distinctive between clusters.
			\end{itemize}
		\end{itemize}
		\item \textbf{The {\color{airforceblue}quality} of a clustering method 
		depends on:}
		\begin{itemize}
			\item the \textbf{\color{airforceblue}similarity measure} used by 
			the method,
			\item its implementation, and
			\item its ability to discover some or all of the hidden patterns.
		\end{itemize}
	\end{itemize}
\end{frame}

\begin{frame}{Measure the Quality of Clustering}
	\begin{itemize}
		\item \textbf{Dissimilarity/similarity metric:}
		\begin{itemize}
			\item Similarity is expressed in terms of a distance function, 
			typically a metric: $d(x,y)$.
			\item The definitions of distance functions are usually rather 
			different for interval-scaled, boolean, categorical, ordinal, 
			ratio, and vector variables (see chapter 2).
			\item \textbf{\color{airforceblue}Weights} should be associated 
			with different variables \\
			based on applications and data semantics.
		\end{itemize}
		\item \textbf{Quality of clustering:}
		\begin{itemize}
			\item There is usually a separate 
			\textbf{\color{airforceblue}"quality" function} that measures the 
			"goodness" of a cluster.
			\item It is hard to define "similar enough" or "good enough."
			\item The answer is typically highly subjective.
		\end{itemize}
	\end{itemize}
\end{frame}

\begin{frame}{Considerations for Cluster Analysis}
	\begin{itemize}
		\item \textbf{Partitioning criteria:}
		\begin{itemize}
			\item Single level vs. hierarchical partitioning.
			\item Often, multi-level hierarchical partitioning is desirable.
		\end{itemize}
		\item \textbf{Separation of clusters:}
		\begin{itemize}
			\item Exclusive (e.g., one customer belongs to only one region) vs.
			\item Non-exclusive (e.g., one document may belong to more than one 
			class).
		\end{itemize}
		\item \textbf{Similarity measure:}
		\begin{itemize}
			\item Distance-based (e.g., Euclidean, road network, vector) vs.
			\item Connectivity-based (e.g., density or contiguity).
		\end{itemize}
		\item \textbf{Clustering space:}
		\begin{itemize}
			\item Full space (often when low-dimensional) vs.
			\item Subspaces (often in high-dimensional clustering).
		\end{itemize}
	\end{itemize}
\end{frame}

\begin{frame}{Requirements and Challenges}
	\begin{itemize}
		\item \textbf{Scalability:}
		\begin{itemize}
			\item Clustering all the data instead of only the samples.
		\end{itemize}
		\item \textbf{Ability to deal with different types of attributes:}
		\begin{itemize}
			\item Numerical, binary, categorical, ordinal, linked, and mixture 
			of these.
		\end{itemize}
		\item \textbf{Constraint-based clustering:}
		\begin{itemize}
			\item User may give inputs on constraints.
			\item Use domain knowledge to determine input parameters.
		\end{itemize}
		\item \textbf{Interpretability and usability.}
		\item \textbf{Others:}
		\begin{itemize}
			\item Discovery of clusters with arbitrary shape.
			\item Ability to deal with noisy data.
			\item Incremental clustering and insensitivity to input order.
			\item High dimensionality.
		\end{itemize}
	\end{itemize}
\end{frame}

\begin{frame}{Major Clustering Approaches}
	\begin{itemize}
		\item \textbf{Partitioning approach:}
		\begin{itemize}
			\item Construct various partitions and then evaluate them by some 
			criterion.
			\item E.g., minimizing the sum of square errors.
			\item Typical methods: k-means, k-medoids, CLARA, CLARANS.
		\end{itemize}
		\item \textbf{Hierarchical approach:}
		\begin{itemize}
			\item Create a hierarchical decomposition of the set of data (or 
			objects) using some criterion.
			\item Typical methods: AGNES, DIANA, BIRCH, CHAMELEON.
		\end{itemize}
		\item \textbf{Density-based approach:}
		\begin{itemize}
			\item Based on connectivity and density functions.
			\item Typical methods: DBSCAN, OPTICS, DENCLUE.
		\end{itemize}
		\item \textbf{Grid-based approach:}
		\begin{itemize}
			\item Based on a multiple-level granularity structure.
			\item Typical methods: STING, WaveCluster, CLIQUE.
		\end{itemize}
	\end{itemize}
\end{frame}

\begin{frame}{Major Clustering Approaches (II)}
	\begin{itemize}
		\item \textbf{Model-based approach:}
		\begin{itemize}
			\item A model is hypothesized for each of the clusters and tries to 
			find the best fit of that model to each other.
			\item Typical methods: EM, SOM, COBWEB.
		\end{itemize}
		\item \textbf{Frequent-pattern-based approach:}
		\begin{itemize}
			\item Based on the analysis of frequent patterns.
			\item Typical methods: p-Cluster.
		\end{itemize}
		\item \textbf{User-guided or constraint-based approach:}
		\begin{itemize}
			\item Clustering by considering user-specified or 
			application-specific constraints.
			\item Typical methods: COD (obstacles), constrained clustering.
		\end{itemize}
		\item \textbf{Link-based clustering:}
		\begin{itemize}
			\item Objects are often linked together in various ways.
			\item Massive links can be used to cluster objects: SimRank, 
			LinkClus.
		\end{itemize}
	\end{itemize}
\end{frame}